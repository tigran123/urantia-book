\upaper{84}{Marriage and Family Life}
\vs p084 0:1 MATERIAL necessity founded marriage, sex hunger embellished it, religion sanctioned and exalted it, the state demanded and regulated it, while in later times evolving love is beginning to justify and glorify marriage as the ancestor and creator of civilization’s most useful and sublime institution, the home. And home building should be the center and essence of all educational effort.
\vs p084 0:2 Mating is purely an act of self\hyp{}perpetuation associated with varying degrees of self\hyp{}gratification; marriage, home building, is largely a matter of self\hyp{}maintenance, and it implies the evolution of society. Society itself is the aggregated structure of family units. Individuals are very temporary as planetary factors --- only families are continuing agencies in social evolution. The family is the channel through which the river of culture and knowledge flows from one generation to another.
\vs p084 0:3 The home is basically a sociologic institution. Marriage grew out of co\hyp{}operation in self\hyp{}maintenance and partnership in self\hyp{}perpetuation, the element of self\hyp{}gratification being largely incidental. Nevertheless, the home does embrace all three of the essential functions of human existence, while life propagation makes it the fundamental human institution, and sex sets it off from all other social activities.
\usection{1.\bibnobreakspace Primitive Pair Associations}
\vs p084 1:1 Marriage was not founded on sex relations; they were incidental thereto. Marriage was not needed by primitive man, who indulged his sex appetite freely without encumbering himself with the responsibilities of wife, children, and home.
\vs p084 1:2 Woman, because of physical and emotional attachment to her offspring, is dependent on co\hyp{}operation with the male, and this urges her into the sheltering protection of marriage. But no direct biologic urge led man into marriage --- much less held him in. It was not love that made marriage attractive to man, but food hunger which first attracted savage man to woman and the primitive shelter shared by her children.
\vs p084 1:3 \pc Marriage was not even brought about by the conscious realization of the obligations of sex relations. Primitive man comprehended no connection between sex indulgence and the subsequent birth of a child. It was once universally believed that a virgin could become pregnant. The savage early conceived the idea that babies were made in spiritland; pregnancy was believed to be the result of a woman’s being entered by a spirit, an evolving ghost. Both diet and the evil eye were also believed to be capable of causing pregnancy in a virgin or unmarried woman, while later beliefs connected the beginnings of life with the breath and with sunlight.
\vs p084 1:4 Many early peoples associated ghosts with the sea; hence virgins were greatly restricted in their bathing practices; young women were far more afraid of bathing in the sea at high tide than of having sex relations. Deformed or premature babies were regarded as the young of animals which had found their way into a woman’s body as a result of careless bathing or through malevolent spirit activity. Savages, of course, thought nothing of strangling such offspring at birth.
\vs p084 1:5 The first step in enlightenment came with the belief that sex relations opened up the way for the impregnating ghost to enter the female. Man has since discovered that father and mother are equal contributors of the living inheritance factors which initiate offspring. But even in the twentieth century many parents still endeavor to keep their children in more or less ignorance as to the origin of human life.
\vs p084 1:6 \pc A family of some simple sort was insured by the fact that the reproductive function entails the mother\hyp{}child relationship. Mother love is instinctive; it did not originate in the mores as did marriage. All mammalian mother love is the inherent endowment of the adjutant mind\hyp{}spirits of the local universe and is in strength and devotion always directly proportional to the length of the helpless infancy of the species.
\vs p084 1:7 The mother and child relation is natural, strong, and instinctive, and one which, therefore, constrained primitive women to submit to many strange conditions and to endure untold hardships. This compelling mother love is the handicapping emotion which has always placed woman at such a tremendous disadvantage in all her struggles with man. Even at that, maternal instinct in the human species is not overpowering; it may be thwarted by ambition, selfishness, and religious conviction.
\vs p084 1:8 While the mother\hyp{}child association is neither marriage nor home, it was the nucleus from which both sprang. The great advance in the evolution of mating came when these temporary partnerships lasted long enough to rear the resultant offspring, for that was homemaking.
\vs p084 1:9 Regardless of the antagonisms of these early pairs, notwithstanding the looseness of the association, the chances for survival were greatly improved by these male\hyp{}female partnerships. A man and a woman, co\hyp{}operating, even aside from family and offspring, are vastly superior in most ways to either two men or two women. This pairing of the sexes enhanced survival and was the very beginning of human society. The sex division of labor also made for comfort and increased happiness.
\usection{2.\bibnobreakspace The Early Mother\hyp{}Family}
\vs p084 2:1 The woman’s periodic hemorrhage and her further loss of blood at childbirth early suggested blood as the creator of the child (even as the seat of the soul) and gave origin to the blood\hyp{}bond concept of human relationships. In early times all descent was reckoned in the female line, that being the only part of inheritance which was at all certain.
\vs p084 2:2 The primitive family, growing out of the instinctive biologic blood bond of mother and child, was inevitably a mother\hyp{}family; and many tribes long held to this arrangement. The mother\hyp{}family was the only possible transition from the stage of group marriage in the horde to the later and improved home life of the polygamous and monogamous father\hyp{}families. The mother\hyp{}family was natural and biologic; the father\hyp{}family is social, economic, and political. The persistence of the mother\hyp{}family among the North American red men is one of the chief reasons why the otherwise progressive Iroquois never became a real state.
\vs p084 2:3 Under the mother\hyp{}family mores the wife’s mother enjoyed virtually supreme authority in the home; even the wife’s brothers and their sons were more active in family supervision than was the husband. Fathers were often renamed after their own children.
\vs p084 2:4 The earliest races gave little credit to the father, looking upon the child as coming altogether from the mother. They believed that children resembled the father as a result of association, or that they were “marked” in this manner because the mother desired them to look like the father. Later on, when the switch came from the mother\hyp{}family to the father\hyp{}family, the father took all credit for the child, and many of the taboos on a pregnant woman were subsequently extended to include her husband. The prospective father ceased work as the time of delivery approached, and at childbirth he went to bed, along with the wife, remaining at rest from three to eight days. The wife might arise the next day and engage in hard labor, but the husband remained in bed to receive congratulations; this was all a part of the early mores designed to establish the father’s right to the child.
\vs p084 2:5 At first, it was the custom for the man to go to his wife’s people, but in later times, after a man had paid or worked out the bride price, he could take his wife and children back to his own people. The transition from the mother\hyp{}family to the father\hyp{}family explains the otherwise meaningless prohibitions of some types of cousin marriages while others of equal kinship are approved.
\vs p084 2:6 With the passing of the hunter mores, when herding gave man control of the chief food supply, the mother\hyp{}family came to a speedy end. It failed simply because it could not successfully compete with the newer father\hyp{}family. Power lodged with the male relatives of the mother could not compete with power concentrated in the husband\hyp{}father. Woman was not equal to the combined tasks of childbearing and of exercising continuous authority and increasing domestic power. The oncoming of wife stealing and later wife purchase hastened the passing of the mother\hyp{}family.
\vs p084 2:7 The stupendous change from the mother\hyp{}family to the father\hyp{}family is one of the most radical and complete right\hyp{}about\hyp{}face adjustments ever executed by the human race. This change led at once to greater social expression and increased family adventure.
\usection{3.\bibnobreakspace The Family under Father Dominance}
\vs p084 3:1 It may be that the instinct of motherhood led woman into marriage, but it was man’s superior strength, together with the influence of the mores, that virtually compelled her to remain in wedlock. Pastoral living tended to create a new system of mores, the patriarchal type of family life; and the basis of family unity under the herder and early agricultural mores was the unquestioned and arbitrary authority of the father. All society, whether national or familial, passed through the stage of the autocratic authority of a patriarchal order.
\vs p084 3:2 The scant courtesy paid womankind during the Old Testament era is a true reflection of the mores of the herdsmen. The Hebrew patriarchs were all herdsmen, as is witnessed by the saying, “The Lord is my Shepherd.”
\vs p084 3:3 But man was no more to blame for his low opinion of woman during past ages than was woman herself. She failed to get social recognition during primitive times because she did not function in an emergency; she was not a spectacular or crisis hero. Maternity was a distinct disability in the existence struggle; mother love handicapped women in the tribal defense.
\vs p084 3:4 Primitive women also unintentionally created their dependence on the male by their admiration and applause for his pugnacity and virility. This exaltation of the warrior elevated the male ego while it equally depressed that of the female and made her more dependent; a military uniform still mightily stirs the feminine emotions.
\vs p084 3:5 Among the more advanced races, women are not so large or so strong as men. Woman, being the weaker, therefore became the more tactful; she early learned to trade upon her sex charms. She became more alert and conservative than man, though slightly less profound. Man was woman’s superior on the battlefield and in the hunt; but at home woman has usually outgeneraled even the most primitive of men.
\vs p084 3:6 \pc The herdsman looked to his flocks for sustenance, but throughout these pastoral ages woman must still provide the vegetable food. Primitive man shunned the soil; it was altogether too peaceful, too unadventuresome. There was also an old superstition that women could raise better plants; they were mothers. In many backward tribes today, the men cook the meat, the women the vegetables, and when the primitive tribes of Australia are on the march, the women never attack game, while a man would not stoop to dig a root.
\vs p084 3:7 Woman has always had to work; at least right up to modern times the female has been a real producer. Man has usually chosen the easier path, and this inequality has existed throughout the entire history of the human race. Woman has always been the burden bearer, carrying the family property and tending the children, thus leaving the man’s hands free for fighting or hunting.
\vs p084 3:8 Woman’s first liberation came when man consented to till the soil, consented to do what had theretofore been regarded as woman’s work. It was a great step forward when male captives were no longer killed but were enslaved as agriculturists. This brought about the liberation of woman so that she could devote more time to homemaking and child culture.
\vs p084 3:9 The provision of milk for the young led to earlier weaning of babies, hence to the bearing of more children by the mothers thus relieved of their sometimes temporary barrenness, while the use of cow’s milk and goat’s milk greatly reduced infant mortality. Before the herding stage of society, mothers used to nurse their babies until they were four and five years old.
\vs p084 3:10 Decreasing primitive warfare greatly lessened the disparity between the division of labor based on sex. But women still had to do the real work while men did picket duty. No camp or village could be left unguarded day or night, but even this task was alleviated by the domestication of the dog. In general, the coming of agriculture has enhanced woman’s prestige and social standing; at least this was true up to the time man himself turned agriculturist. And as soon as man addressed himself to the tilling of the soil, there immediately ensued great improvement in methods of agriculture, extending on down through successive generations. In hunting and war man had learned the value of organization, and he introduced these techniques into industry and later, when taking over much of woman’s work, greatly improved on her loose methods of labor.
\usection{4.\bibnobreakspace Woman’s Status in Early Society}
\vs p084 4:1 Generally speaking, during any age woman’s status is a fair criterion of the evolutionary progress of marriage as a social institution, while the progress of marriage itself is a reasonably accurate gauge registering the advances of human civilization.
\vs p084 4:2 \pc Woman’s status has always been a social paradox; she has always been a shrewd manager of men; she has always capitalized man’s stronger sex urge for her own interests and to her own advancement. By trading subtly upon her sex charms, she has often been able to exercise dominant power over man, even when held by him in abject slavery.
\vs p084 4:3 Early woman was not to man a friend, sweetheart, lover, and partner but rather a piece of property, a servant or slave and, later on, an economic partner, plaything, and childbearer. Nonetheless, proper and satisfactory sex relations have always involved the element of choice and co\hyp{}operation by woman, and this has always given intelligent women considerable influence over their immediate and personal standing, regardless of their social position as a sex. But man’s distrust and suspicion were not helped by the fact that women were all along compelled to resort to shrewdness in the effort to alleviate their bondage.
\vs p084 4:4 \pc The sexes have had great difficulty in understanding each other. Man found it hard to understand woman, regarding her with a strange mixture of ignorant mistrust and fearful fascination, if not with suspicion and contempt. Many tribal and racial traditions relegate trouble to Eve, Pandora, or some other representative of womankind. These narratives were always distorted so as to make it appear that the woman brought evil upon man; and all this indicates the onetime universal distrust of woman. Among the reasons cited in support of a celibate priesthood, the chief was the baseness of woman. The fact that most supposed witches were women did not improve the olden reputation of the sex.
\vs p084 4:5 Men have long regarded women as peculiar, even abnormal. They have even believed that women did not have souls; therefore were they denied names. During early times there existed great fear of the first sex relation with a woman; hence it became the custom for a priest to have initial intercourse with a virgin. Even a woman’s shadow was thought to be dangerous.
\vs p084 4:6 Childbearing was once generally looked upon as rendering a woman dangerous and unclean. And many tribal mores decreed that a mother must undergo extensive purification ceremonies subsequent to the birth of a child. Except among those groups where the husband participated in the lying\hyp{}in, the expectant mother was shunned, left alone. The ancients even avoided having a child born in the house. Finally, the old women were permitted to attend the mother during labor, and this practice gave origin to the profession of midwifery. During labor, scores of foolish things were said and done in an effort to facilitate delivery. It was the custom to sprinkle the newborn with holy water to prevent ghost interference.
\vs p084 4:7 Among the unmixed tribes, childbirth was comparatively easy, occupying only two or three hours; it is seldom so easy among the mixed races. If a woman died in childbirth, especially during the delivery of twins, she was believed to have been guilty of spirit adultery. Later on, the higher tribes looked upon death in childbirth as the will of heaven; such mothers were regarded as having perished in a noble cause.
\vs p084 4:8 The so\hyp{}called modesty of women respecting their clothing and the exposure of the person grew out of the deadly fear of being observed at the time of a menstrual period. To be thus detected was a grievous sin, the violation of a taboo. Under the mores of olden times, every woman, from adolescence to the end of the childbearing period, was subjected to complete family and social quarantine one full week each month. Everything she might touch, sit upon, or lie upon was “defiled.” It was for long the custom to brutally beat a girl after each monthly period in an effort to drive the evil spirit out of her body. But when a woman passed beyond the childbearing age, she was usually treated more considerately, being accorded more rights and privileges. In view of all this it was not strange that women were looked down upon. Even the Greeks held the menstruating woman as one of the three great causes of defilement, the other two being pork and garlic.
\vs p084 4:9 However foolish these olden notions were, they did some good since they gave overworked females, at least when young, one week each month for welcome rest and profitable meditation. Thus could they sharpen their wits for dealing with their male associates the rest of the time. This quarantine of women also protected men from over\hyp{}sex indulgence, thereby indirectly contributing to the restriction of population and to the enhancement of self\hyp{}control.
\vs p084 4:10 \pc A great advance was made when a man was denied the right to kill his wife at will. Likewise, it was a forward step when a woman could own the wedding gifts. Later, she gained the legal right to own, control, and even dispose of property, but she was long deprived of the right to hold office in either church or state. Woman has always been treated more or less as property, right up to and in the twentieth century after Christ. She has not yet gained world\hyp{}wide freedom from seclusion under man’s control. Even among advanced peoples, man’s attempt to protect woman has always been a tacit assertion of superiority.
\vs p084 4:11 But primitive women did not pity themselves as their more recently liberated sisters are wont to do. They were, after all, fairly happy and contented; they did not dare to envision a better or different mode of existence.
\usection{5.\bibnobreakspace Woman under the Developing Mores}
\vs p084 5:1 In self\hyp{}perpetuation woman is man’s equal, but in the partnership of self\hyp{}maintenance she labors at a decided disadvantage, and this handicap of enforced maternity can only be compensated by the enlightened mores of advancing civilization and by man’s increasing sense of acquired fairness.
\vs p084 5:2 As society evolved, the sex standards rose higher among women because they suffered more from the consequences of the transgression of the sex mores. Man’s sex standards are only tardily improving as a result of the sheer sense of that fairness which civilization demands. Nature knows nothing of fairness --- makes woman alone suffer the pangs of childbirth.
\vs p084 5:3 The modern idea of sex equality is beautiful and worthy of an expanding civilization, but it is not found in nature. When might is right, man lords it over woman; when more justice, peace, and fairness prevail, she gradually emerges from slavery and obscurity. Woman’s social position has generally varied inversely with the degree of militarism in any nation or age.
\vs p084 5:4 But man did not consciously nor intentionally seize woman’s rights and then gradually and grudgingly give them back to her; all this was an unconscious and unplanned episode of social evolution. When the time really came for woman to enjoy added rights, she got them, and all quite regardless of man’s conscious attitude. Slowly but surely the mores change so as to provide for those social adjustments which are a part of the persistent evolution of civilization. The advancing mores slowly provided increasingly better treatment for females; those tribes which persisted in cruelty to them did not survive.
\vs p084 5:5 \pc The Adamites and Nodites accorded women increased recognition, and those groups which were influenced by the migrating Andites have tended to be influenced by the Edenic teachings regarding women’s place in society.
\vs p084 5:6 The early Chinese and the Greeks treated women better than did most surrounding peoples. But the Hebrews were exceedingly distrustful of them. In the Occident woman has had a difficult climb under the Pauline doctrines which became attached to Christianity, although Christianity did advance the mores by imposing more stringent sex obligations upon man. Woman’s estate is little short of hopeless under the peculiar degradation which attaches to her in Mohammedanism, and she fares even worse under the teachings of several other Oriental religions.
\vs p084 5:7 \pc Science, not religion, really emancipated woman; it was the modern factory which largely set her free from the confines of the home. Man’s physical abilities became no longer a vital essential in the new maintenance mechanism; science so changed the conditions of living that man power was no longer so superior to woman power.
\vs p084 5:8 These changes have tended toward woman’s liberation from domestic slavery and have brought about such a modification of her status that she now enjoys a degree of personal liberty and sex determination that practically equals man’s. Once a woman’s value consisted in her food\hyp{}producing ability, but invention and wealth have enabled her to create a new world in which to function --- spheres of grace and charm. Thus has industry won its unconscious and unintended fight for woman’s social and economic emancipation. And again has evolution succeeded in doing what even revelation failed to accomplish.
\vs p084 5:9 \pc The reaction of enlightened peoples from the inequitable mores governing woman’s place in society has indeed been pendulumlike in its extremeness. Among industrialized races she has received almost all rights and enjoys exemption from many obligations, such as military service. Every easement of the struggle for existence has redounded to the liberation of woman, and she has directly benefited from every advance toward monogamy. The weaker always makes disproportionate gains in every adjustment of the mores in the progressive evolution of society.
\vs p084 5:10 In the ideals of pair marriage, woman has finally won recognition, dignity, independence, equality, and education; but will she prove worthy of all this new and unprecedented accomplishment? Will modern woman respond to this great achievement of social liberation with idleness, indifference, barrenness, and infidelity? Today, in the twentieth century, woman is undergoing the crucial test of her long world existence!
\vs p084 5:11 Woman is man’s equal partner in race reproduction, hence just as important in the unfolding of racial evolution; therefore has evolution increasingly worked toward the realization of women’s rights. But women’s rights are by no means men’s rights. Woman cannot thrive on man’s rights any more than man can prosper on woman’s rights.
\vs p084 5:12 Each sex has its own distinctive sphere of existence, together with its own rights within that sphere. If woman aspires literally to enjoy all of man’s rights, then, sooner or later, pitiless and emotionless competition will certainly replace that chivalry and special consideration which many women now enjoy, and which they have so recently won from men.
\vs p084 5:13 Civilization never can obliterate the behavior gulf between the sexes. From age to age the mores change, but instinct never. Innate maternal affection will never permit emancipated woman to become man’s serious rival in industry. Forever each sex will remain supreme in its own domain, domains determined by biologic differentiation and by mental dissimilarity.
\vs p084 5:14 Each sex will always have its own special sphere, albeit they will ever and anon overlap. Only socially will men and women compete on equal terms.
\usection{6.\bibnobreakspace The Partnership of Man and Woman}
\vs p084 6:1 The reproductive urge unfailingly brings men and women together for self\hyp{}perpetuation but, alone, does not insure their remaining together in mutual co\hyp{}operation --- the founding of a home.
\vs p084 6:2 Every successful human institution embraces antagonisms of personal interest which have been adjusted to practical working harmony, and homemaking is no exception. Marriage, the basis of home building, is the highest manifestation of that antagonistic co\hyp{}operation which so often characterizes the contacts of nature and society. The conflict is inevitable. Mating is inherent; it is natural. But marriage is not biologic; it is sociologic. Passion insures that man and woman will come together, but the weaker parental instinct and the social mores hold them together.
\vs p084 6:3 \pc Male and female are, practically regarded, two distinct varieties of the same species living in close and intimate association. Their viewpoints and entire life reactions are essentially different; they are wholly incapable of full and real comprehension of each other. Complete understanding between the sexes is not attainable.
\vs p084 6:4 Women seem to have more intuition than men, but they also appear to be somewhat less logical. Woman, however, has always been the moral standard\hyp{}bearer and the spiritual leader of mankind. The hand that rocks the cradle still fraternizes with destiny.
\vs p084 6:5 \pc The differences of nature, reaction, viewpoint, and thinking between men and women, far from occasioning concern, should be regarded as highly beneficial to mankind, both individually and collectively. Many orders of universe creatures are created in dual phases of personality manifestation. Among mortals, Material Sons, and midsoniters, this difference is described as male and female; among seraphim, cherubim, and Morontia Companions, it has been denominated positive or aggressive and negative or retiring. Such dual associations greatly multiply versatility and overcome inherent limitations, even as do certain triune associations in the Paradise\hyp{}Havona system.
\vs p084 6:6 Men and women need each other in their morontial and spiritual as well as in their mortal careers. The differences in viewpoint between male and female persist even beyond the first life and throughout the local and superuniverse ascensions. And even in Havona, the pilgrims who were once men and women will still be aiding each other in the Paradise ascent. Never, even in the Corps of the Finality, will the creature metamorphose so far as to obliterate the personality trends that humans call male and female; always will these two basic variations of humankind continue to intrigue, stimulate, encourage, and assist each other; always will they be mutually dependent on co\hyp{}operation in the solution of perplexing universe problems and in the overcoming of manifold cosmic difficulties.
\vs p084 6:7 \pc While the sexes never can hope fully to understand each other, they are effectively complementary, and though co\hyp{}operation is often more or less personally antagonistic, it is capable of maintaining and reproducing society. Marriage is an institution designed to compose sex differences, meanwhile effecting the continuation of civilization and insuring the reproduction of the race.
\vs p084 6:8 Marriage is the mother of all human institutions, for it leads directly to home founding and home maintenance, which is the structural basis of society. The family is vitally linked to the mechanism of self\hyp{}maintenance; it is the sole hope of race perpetuation under the mores of civilization, while at the same time it most effectively provides certain highly satisfactory forms of self\hyp{}gratification. The family is man’s greatest purely human achievement, combining as it does the evolution of the biologic relations of male and female with the social relations of husband and wife.
\usection{7.\bibnobreakspace The Ideals of Family Life}
\vs p084 7:1 Sex mating is instinctive, children are the natural result, and the family thus automatically comes into existence. As are the families of the race or nation, so is its society. If the families are good, the society is likewise good. The great cultural stability of the Jewish and of the Chinese peoples lies in the strength of their family groups.
\vs p084 7:2 Woman’s instinct to love and care for children conspired to make her the interested party in promoting marriage and primitive family life. Man was only forced into home building by the pressure of the later mores and social conventions; he was slow to take an interest in the establishment of marriage and home because the sex act imposes no biologic consequences upon him.
\vs p084 7:3 Sex association is natural, but marriage is social and has always been regulated by the mores. The mores (religious, moral, and ethical), together with property, pride, and chivalry, stabilize the institutions of marriage and family. Whenever the mores fluctuate, there is fluctuation in the stability of the home\hyp{}marriage institution. Marriage is now passing out of the property stage into the personal era. Formerly man protected woman because she was his chattel, and she obeyed for the same reason. Regardless of its merits this system did provide stability. Now, woman is no longer regarded as property, and new mores are emerging designed to stabilize the marriage\hyp{}home institution:
\vs p084 7:4 \ublistelem{1.}\bibnobreakspace The new role of religion --- the teaching that parental experience is essential, the idea of procreating cosmic citizens, the enlarged understanding of the privilege of procreation --- giving sons to the Father.
\vs p084 7:5 \pc \ublistelem{2.}\bibnobreakspace The new role of science --- procreation is becoming more and more voluntary, subject to man’s control. In ancient times lack of understanding insured the appearance of children in the absence of all desire therefor.
\vs p084 7:6 \pc \ublistelem{3.}\bibnobreakspace The new function of pleasure lures --- this introduces a new factor into racial survival; ancient man exposed undesired children to die; moderns refuse to bear them.
\vs p084 7:7 \pc \ublistelem{4.}\bibnobreakspace The enhancement of parental instinct --- each generation now tends to eliminate from the reproductive stream of the race those individuals in whom parental instinct is insufficiently strong to insure the procreation of children, the prospective parents of the next generation.
\vs p084 7:8 \pc But the home as an institution, a partnership between one man and one woman, dates more specifically from the days of Dalamatia, about one\hyp{}half million years ago, the monogamous practices of Andon and his immediate descendants having been abandoned long before. Family life, however, was not much to boast of before the days of the Nodites and the later Adamites. Adam and Eve exerted a lasting influence on all mankind; for the first time in the history of the world men and women were observed working side by side in the Garden. The Edenic ideal, the whole family as gardeners, was a new idea on Urantia.
\vs p084 7:9 The early family embraced a related working group, including the slaves, all living in one dwelling. Marriage and family life have not always been identical but have of necessity been closely associated. Woman always wanted the individual family, and eventually she had her way.
\vs p084 7:10 \pc Love of offspring is almost universal and is of distinct survival value. The ancients always sacrificed the mother’s interests for the welfare of the child; an Eskimo mother even yet licks her baby in lieu of washing. But primitive mothers only nourished and cared for their children when very young; like the animals, they discarded them as soon as they grew up. Enduring and continuous human associations have never been founded on biologic affection alone. The animals love their children; man --- civilized man --- loves his children’s children. The higher the civilization, the greater the joy of parents in the children’s advancement and success; thus the new and higher realization of \bibemph{name} pride comes into existence.
\vs p084 7:11 The large families among ancient peoples were not necessarily affectional. Many children were desired because:
\vs p084 7:12 \ublistelem{1.}\bibnobreakspace They were valuable as laborers.
\vs p084 7:13 \ublistelem{2.}\bibnobreakspace They were old\hyp{}age insurance.
\vs p084 7:14 \ublistelem{3.}\bibnobreakspace Daughters were salable.
\vs p084 7:15 \ublistelem{4.}\bibnobreakspace Family pride required extension of name.
\vs p084 7:16 \ublistelem{5.}\bibnobreakspace Sons afforded protection and defense.
\vs p084 7:17 \ublistelem{6.}\bibnobreakspace Ghost fear produced a dread of being alone.
\vs p084 7:18 \ublistelem{7.}\bibnobreakspace Certain religions required offspring.
\vs p084 7:19 \pc Ancestor worshipers view the failure to have sons as the supreme calamity for all time and eternity. They desire above all else to have sons to officiate in the post\hyp{}mortem feasts, to offer the required sacrifices for the ghost’s progress through spiritland.
\vs p084 7:20 Among ancient savages, discipline of children was begun very early; and the child early realized that disobedience meant failure or even death just as it did to the animals. It is civilization’s protection of the child from the natural consequences of foolish conduct that contributes so much to modern insubordination.
\vs p084 7:21 Eskimo children thrive on so little discipline and correction simply because they are naturally docile little animals; the children of both the red and the yellow men are almost equally tractable. But in races containing Andite inheritance, children are not so placid; these more imaginative and adventurous youths require more training and discipline. Modern problems of child culture are rendered increasingly difficult by:
\vs p084 7:22 \ublistelem{1.}\bibnobreakspace The large degree of race mixture.
\vs p084 7:23 \ublistelem{2.}\bibnobreakspace Artificial and superficial education.
\vs p084 7:24 \ublistelem{3.}\bibnobreakspace Inability of the child to gain culture by imitating parents --- the parents are absent from the family picture so much of the time.
\vs p084 7:25 \pc The olden ideas of family discipline were biologic, growing out of the realization that parents were creators of the child’s being. The advancing ideals of family life are leading to the concept that bringing a child into the world, instead of conferring certain parental rights, entails the supreme responsibility of human existence.
\vs p084 7:26 Civilization regards the parents as assuming all duties, the child as having all the rights. Respect of the child for his parents arises, not in knowledge of the obligation implied in parental procreation, but naturally grows as a result of the care, training, and affection which are lovingly displayed in assisting the child to win the battle of life. The true parent is engaged in a continuous service\hyp{}ministry which the wise child comes to recognize and appreciate.
\vs p084 7:27 \pc In the present industrial and urban era the marriage institution is evolving along new economic lines. Family life has become more and more costly, while children, who used to be an asset, have become economic liabilities. But the security of civilization itself still rests on the growing willingness of one generation to invest in the welfare of the next and future generations. And any attempt to shift parental responsibility to state or church will prove suicidal to the welfare and advancement of civilization.
\vs p084 7:28 \pc Marriage, with children and consequent family life, is stimulative of the highest potentials in human nature and simultaneously provides the ideal avenue for the expression of these quickened attributes of mortal personality. The family provides for the biologic perpetuation of the human species. The home is the natural social arena wherein the ethics of blood brotherhood may be grasped by the growing children. The family is the fundamental unit of fraternity in which parents and children learn those lessons of patience, altruism, tolerance, and forbearance which are so essential to the realization of brotherhood among all men.
\vs p084 7:29 Human society would be greatly improved if the civilized races would more generally return to the family\hyp{}council practices of the Andites. They did not maintain the patriarchal or autocratic form of family government. They were very brotherly and associative, freely and frankly discussing every proposal and regulation of a family nature. They were ideally fraternal in all their family government. In an ideal family filial and parental affection are both augmented by fraternal devotion.
\vs p084 7:30 Family life is the progenitor of true morality, the ancestor of the consciousness of loyalty to duty. The enforced associations of family life stabilize personality and stimulate its growth through the compulsion of necessitous adjustment to other and diverse personalities. But even more, a true family --- a good family --- reveals to the parental procreators the attitude of the Creator to his children, while at the same time such true parents portray to their children the first of a long series of ascending disclosures of the love of the Paradise parent of all universe children.
\usection{8.\bibnobreakspace Dangers of Self\hyp{}Gratification}
\vs p084 8:1 The great threat against family life is the menacing rising tide of self\hyp{}gratification, the modern pleasure mania. The prime incentive to marriage used to be economic; sex attraction was secondary. Marriage, founded on self\hyp{}maintenance, led to self\hyp{}perpetuation and concomitantly provided one of the most desirable forms of self\hyp{}gratification. It is the only institution of human society which embraces all three of the great incentives for living.
\vs p084 8:2 Originally, property was the basic institution of self\hyp{}maintenance, while marriage functioned as the unique institution of self\hyp{}perpetuation. Although food satisfaction, play, and humor, along with periodic sex indulgence, were means of self\hyp{}gratification, it remains a fact that the evolving mores have failed to build any distinct institution of self\hyp{}gratification. And it is due to this failure to evolve specialized techniques of pleasurable enjoyment that all human institutions are so completely shot through with this pleasure pursuit. Property accumulation is becoming an instrument for augmenting all forms of self\hyp{}gratification, while marriage is often viewed only as a means of pleasure. And this overindulgence, this widely spread pleasure mania, now constitutes the greatest threat that has ever been leveled at the social evolutionary institution of family life, the home.
\vs p084 8:3 The violet race introduced a new and only imperfectly realized characteristic into the experience of humankind --- the play instinct coupled with the sense of humor. It was there in measure in the Sangiks and Andonites, but the Adamic strain elevated this primitive propensity into the \bibemph{potential of pleasure,} a new and glorified form of self\hyp{}gratification. The basic type of self\hyp{}gratification, aside from appeasing hunger, is sex gratification, and this form of sensual pleasure was enormously heightened by the blending of the Sangiks and the Andites.
\vs p084 8:4 There is real danger in the combination of restlessness, curiosity, adventure, and pleasure\hyp{}abandon characteristic of the post\hyp{}Andite races. The hunger of the soul cannot be satisfied with physical pleasures; the love of home and children is not augmented by the unwise pursuit of pleasure. Though you exhaust the resources of art, color, sound, rhythm, music, and adornment of person, you cannot hope thereby to elevate the soul or to nourish the spirit. Vanity and fashion cannot minister to home building and child culture; pride and rivalry are powerless to enhance the survival qualities of succeeding generations.
\vs p084 8:5 Advancing celestial beings all enjoy rest and the ministry of the reversion directors. All efforts to obtain wholesome diversion and to engage in uplifting play are sound; refreshing sleep, rest, recreation, and all pastimes which prevent the boredom of monotony are worth while. Competitive games, storytelling, and even the taste of good food may serve as forms of self\hyp{}gratification. (When you use salt to savor food, pause to consider that, for almost a million years, man could obtain salt only by dipping his food in ashes.)
\vs p084 8:6 \pc Let man enjoy himself; let the human race find pleasure in a thousand and one ways; let evolutionary mankind explore all forms of legitimate self\hyp{}gratification, the fruits of the long upward biologic struggle. Man has well earned some of his present\hyp{}day joys and pleasures. But look you well to the goal of destiny! Pleasures are indeed suicidal if they succeed in destroying property, which has become the institution of self\hyp{}maintenance; and self\hyp{}gratifications have indeed cost a fatal price if they bring about the collapse of marriage, the decadence of family life, and the destruction of the home --- man’s supreme evolutionary acquirement and civilization’s only hope of survival.
\vsetoff
\vs p084 8:7 [Presented by the Chief of Seraphim stationed on Urantia.]
