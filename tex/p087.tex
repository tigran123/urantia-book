\upaper{87}{The Ghost Cults}
\vs p087 0:1 THE ghost cult evolved as an offset to the hazards of bad luck; its primitive religious observances were the outgrowth of anxiety about bad luck and of the inordinate fear of the dead. None of these early religions had much to do with the recognition of Deity or with reverence for the superhuman; their rites were mostly negative, designed to avoid, expel, or coerce ghosts. The ghost cult was nothing more nor less than insurance against disaster; it had nothing to do with investment for higher and future returns.
\vs p087 0:2 Man has had a long and bitter struggle with the ghost cult. Nothing in human history is designed to excite more pity than this picture of man’s abject slavery to ghost\hyp{}spirit fear. With the birth of this very fear mankind started on the upgrade of religious evolution. Human imagination cast off from the shores of self and will not again find anchor until it arrives at the concept of a true Deity, a real God.
\usection{1.\bibnobreakspace Ghost Fear}
\vs p087 1:1 Death was feared because death meant the liberation of another ghost from its physical body. The ancients did their best to prevent death, to avoid the trouble of having to contend with a new ghost. They were always anxious to induce the ghost to leave the scene of death, to embark on the journey to deadland. The ghost was feared most of all during the supposed transition period between its emergence at the time of death and its later departure for the ghost homeland, a vague and primitive concept of pseudo heaven.
\vs p087 1:2 Though the savage credited ghosts with supernatural powers, he hardly conceived of them as having supernatural intelligence. Many tricks and stratagems were practiced in an effort to hoodwink and deceive the ghosts; civilized man still pins much faith on the hope that an outward manifestation of piety will in some manner deceive even an omniscient Deity.
\vs p087 1:3 The primitives feared sickness because they observed it was often a harbinger of death. If the tribal medicine man failed to cure an afflicted individual, the sick man was usually removed from the family hut, being taken to a smaller one or left in the open air to die alone. A house in which death had occurred was usually destroyed; if not, it was always avoided, and this fear prevented early man from building substantial dwellings. It also militated against the establishment of permanent villages and cities.
\vs p087 1:4 The savages sat up all night and talked when a member of the clan died; they feared they too would die if they fell asleep in the vicinity of a corpse. Contagion from the corpse substantiated the fear of the dead, and all peoples, at one time or another, have employed elaborate purification ceremonies designed to cleanse an individual after contact with the dead. The ancients believed that light must be provided for a corpse; a dead body was never permitted to remain in the dark. In the twentieth century, candles are still burned in death chambers, and men still sit up with the dead. So\hyp{}called civilized man has hardly yet completely eliminated the fear of dead bodies from his philosophy of life.
\vs p087 1:5 But despite all this fear, men still sought to trick the ghost. If the death hut was not destroyed, the corpse was removed through a hole in the wall, never by way of the door. These measures were taken to confuse the ghost, to prevent its tarrying, and to insure against its return. Mourners also returned from a funeral by a different road, lest the ghost follow. Backtracking and scores of other tactics were practiced to insure that the ghost would not return from the grave. The sexes often exchanged clothes in order to deceive the ghost. Mourning costumes were designed to disguise survivors; later on, to show respect for the dead and thus appease the ghosts.
\usection{2.\bibnobreakspace Ghost Placation}
\vs p087 2:1 In religion the negative program of ghost placation long preceded the positive program of spirit coercion and supplication. The first acts of human worship were phenomena of defense, not reverence. Modern man deems it wise to insure against fire; so the savage thought it the better part of wisdom to provide insurance against ghost bad luck. The effort to secure this protection constituted the techniques and rituals of the ghost cult.
\vs p087 2:2 \pc It was once thought that the great desire of a ghost was to be quickly “laid” so that it might proceed undisturbed to deadland. Any error of commission or omission in the acts of the living in the ritual of laying the ghost was sure to delay its progress to ghostland. This was believed to be displeasing to the ghost, and an angered ghost was supposed to be a source of calamity, misfortune, and unhappiness.
\vs p087 2:3 The funeral service originated in man’s effort to induce the ghost soul to depart for its future home, and the funeral sermon was originally designed to instruct the new ghost how to get there. It was the custom to provide food and clothes for the ghost’s journey, these articles being placed in or near the grave. The savage believed that it required from three days to a year to “lay the ghost” --- to get it away from the vicinity of the grave. The Eskimos still believe that the soul stays with the body three days.
\vs p087 2:4 Silence or mourning was observed after a death so that the ghost would not be attracted back home. Self\hyp{}torture --- wounds --- was a common form of mourning. Many advanced teachers tried to stop this, but they failed. Fasting and other forms of self\hyp{}denial were thought to be pleasing to the ghosts, who took pleasure in the discomfort of the living during the transition period of lurking about before their actual departure for deadland.
\vs p087 2:5 Long and frequent periods of mourning inactivity were one of the great obstacles to civilization’s advancement. Weeks and even months of each year were literally wasted in this nonproductive and useless mourning. The fact that professional mourners were hired for funeral occasions indicates that mourning was a ritual, not an evidence of sorrow. Moderns may mourn the dead out of respect and because of bereavement, but the ancients did this because of \bibemph{fear.}
\vs p087 2:6 The names of the dead were never spoken. In fact, they were often banished from the language. These names became taboo, and in this way the languages were constantly impoverished. This eventually produced a multiplication of symbolic speech and figurative expression, such as “the name or day one never mentions.”
\vs p087 2:7 \pc The ancients were so anxious to get rid of a ghost that they offered it everything which might have been desired during life. Ghosts wanted wives and servants; a well\hyp{}to\hyp{}do savage expected that at least one slave wife would be buried alive at his death. It later became the custom for a widow to commit suicide on her husband’s grave. When a child died, the mother, aunt, or grandmother was often strangled in order that an adult ghost might accompany and care for the child ghost. And those who thus gave up their lives usually did so willingly; indeed, had they lived in violation of custom, their fear of ghost wrath would have denuded life of such few pleasures as the primitives enjoyed.
\vs p087 2:8 It was customary to dispatch a large number of subjects to accompany a dead chief; slaves were killed when their master died that they might serve him in ghostland. The Borneans still provide a courier companion; a slave is speared to death to make the ghost journey with his deceased master. Ghosts of murdered persons were believed to be delighted to have the ghosts of their murderers as slaves; this notion motivated men to head hunting.
\vs p087 2:9 Ghosts supposedly enjoyed the smell of food; food offerings at funeral feasts were once universal. The primitive method of saying grace was, before eating, to throw a bit of food into the fire for the purpose of appeasing the spirits, while mumbling a magic formula.
\vs p087 2:10 The dead were supposed to use the ghosts of the tools and weapons that were theirs in life. To break an article was to “kill it,” thus releasing its ghost to pass on for service in ghostland. Property sacrifices were also made by burning or burying. Ancient funeral wastes were enormous. Later races made paper models and substituted drawings for real objects and persons in these death sacrifices. It was a great advance in civilization when the inheritance of kin replaced the burning and burying of property. The Iroquois Indians made many reforms in funeral waste. And this conservation of property enabled them to become the most powerful of the northern red men. Modern man is not supposed to fear ghosts, but custom is strong, and much terrestrial wealth is still consumed on funeral rituals and death ceremonies.
\usection{3.\bibnobreakspace Ancestor Worship}
\vs p087 3:1 The advancing ghost cult made ancestor worship inevitable since it became the connecting link between common ghosts and the higher spirits, the evolving gods. The early gods were simply glorified departed humans.
\vs p087 3:2 Ancestor worship was originally more of a fear than a worship, but such beliefs did definitely contribute to the further spread of ghost fear and worship. Devotees of the early ancestor\hyp{}ghost cults even feared to yawn lest a malignant ghost enter their bodies at such a time.
\vs p087 3:3 The custom of adopting children was to make sure that someone would provide offerings after death for the peace and progress of the soul. The savage lived in fear of the ghosts of his fellows and spent his spare time planning for the safe conduct of his own ghost after death.\fnc{The custom of adopting children was to make sure that \bibtextul{some one} would provide offerings after death\ldots{} \bibexpl{The two-word form is appropriate when referring to some one member of a particular group, as “Some one of you will go with me\ldots{}” The compound form is used when the group of which the ‘one’ is a member is not specified. \bibemph{Fowler} (1926) clarifies the differentiation by stating that ‘someone’ should be used when ‘somebody’ could be substituted for it; ‘some one’ should be used in all other cases.}}
\vs p087 3:4 Most tribes instituted an all\hyp{}souls’ feast at least once a year. The Romans had twelve ghost feasts and accompanying ceremonies each year. Half the days of the year were dedicated to some sort of ceremony associated with these ancient cults. One Roman emperor tried to reform these practices by reducing the number of feast days to 135 a year.
\vs p087 3:5 \pc The ghost cult was in continuous evolution. As ghosts were envisioned as passing from the incomplete to the higher phase of existence, so did the cult eventually progress to the worship of spirits, and even gods. But regardless of varying beliefs in more advanced spirits, all tribes and races once believed in ghosts.
\usection{4.\bibnobreakspace Good and Bad Spirit Ghosts}
\vs p087 4:1 Ghost fear was the fountainhead of all world religion; and for ages many tribes clung to the old belief in one class of ghosts. They taught that man had good luck when the ghost was pleased, bad luck when he was angered.
\vs p087 4:2 As the cult of ghost fear expanded, there came about the recognition of higher types of spirits, spirits not definitely identifiable with any individual human. They were graduate or glorified ghosts who had progressed beyond the domain of ghostland to the higher realms of spiritland.
\vs p087 4:3 The notion of two kinds of spirit ghosts made slow but sure progress throughout the world. This new dual spiritism did not have to spread from tribe to tribe; it sprang up independently all over the world. In influencing the expanding evolutionary mind, the power of an idea lies not in its reality or reasonableness but rather in its \bibemph{vividness} and the universality of its ready and simple application.
\vs p087 4:4 Still later the imagination of man envisioned the concept of both good and bad supernatural agencies; some ghosts never evolved to the level of good spirits. The early monospiritism of ghost fear was gradually evolving into a dual spiritism, a new concept of the invisible control of earthly affairs. At last good luck and bad luck were pictured as having their respective controllers. And of the two classes, the group that brought bad luck were believed to be the more active and numerous.
\vs p087 4:5 \pc When the doctrine of good and bad spirits finally matured, it became the most widespread and persistent of all religious beliefs. This dualism represented a great religio\hyp{}philosophic advance because it enabled man to account for both good luck and bad luck while at the same time believing in supermortal beings who were to some extent consistent in their behavior. The spirits could be counted on to be either good or bad; they were not thought of as being completely temperamental as the early ghosts of the monospiritism of most primitive religions had been conceived to be. Man was at last able to conceive of supermortal forces that were consistent in behavior, and this was one of the most momentous discoveries of truth in the entire history of the evolution of religion and in the expansion of human philosophy.
\vs p087 4:6 Evolutionary religion has, however, paid a terrible price for the concept of dual spiritism. Man’s early philosophy was able to reconcile spirit constancy with the vicissitudes of temporal fortune only by postulating two kinds of spirits, one good and the other bad. And while this belief did enable man to reconcile the variables of chance with a concept of unchanging supermortal forces, this doctrine has ever since made it difficult for religionists to conceive of cosmic unity. The gods of evolutionary religion have generally been opposed by the forces of darkness.
\vs p087 4:7 The tragedy of all this lies in the fact that, when these ideas were taking root in the primitive mind of man, there really were no bad or disharmonious spirits in all the world. Such an unfortunate situation did not develop until after the Caligastic rebellion and only persisted until Pentecost. The concept of good and evil as cosmic co\hyp{}ordinates is, even in the twentieth century, very much alive in human philosophy; most of the world’s religions still carry this cultural birthmark of the long\hyp{}gone days of the emerging ghost cults.
\usection{5.\bibnobreakspace The Advancing Ghost Cult}
\vs p087 5:1 Primitive man viewed the spirits and ghosts as having almost unlimited rights but no duties; the spirits were thought to regard man as having manifold duties but no rights. The spirits were believed to look down upon man as constantly failing in the discharge of his spiritual duties. It was the general belief of mankind that ghosts levied a continuous tribute of service as the price of noninterference in human affairs, and the least mischance was laid to ghost activities. Early humans were so afraid they might overlook some honor due the gods that, after they had sacrificed to all known spirits, they did another turn to the “unknown gods,” just to be thoroughly safe.
\vs p087 5:2 And now the simple ghost cult is followed by the practices of the more advanced and relatively complex spirit\hyp{}ghost cult, the service and worship of the higher spirits as they evolved in man’s primitive imagination. Religious ceremonial must keep pace with spirit evolution and progress. The expanded cult was but the art of self\hyp{}maintenance practiced in relation to belief in supernatural beings, self\hyp{}adjustment to spirit environment. Industrial and military organizations were adjustments to natural and social environments. And as marriage arose to meet the demands of bisexuality, so did religious organization evolve in response to the belief in higher spirit forces and spiritual beings. Religion represents man’s adjustment to his illusions of the mystery of chance. Spirit fear and subsequent worship were adopted as insurance against misfortune, as prosperity policies.
\vs p087 5:3 The savage visualizes the good spirits as going about their business, requiring little from human beings. It is the bad ghosts and spirits who must be kept in good humor. Accordingly, primitive peoples paid more attention to their malevolent ghosts than to their benign spirits.
\vs p087 5:4 Human prosperity was supposed to be especially provocative of the envy of evil spirits, and their method of retaliation was to strike back through a human agency and by the technique of the \bibemph{evil eye.} That phase of the cult which had to do with spirit avoidance was much concerned with the machinations of the evil eye. The fear of it became almost world\hyp{}wide. Pretty women were veiled to protect them from the evil eye; subsequently many women who desired to be considered beautiful adopted this practice. Because of this fear of bad spirits, children were seldom allowed out after dark, and the early prayers always included the petition, “deliver us from the evil eye.”
\vs p087 5:5 The Koran contains a whole chapter devoted to the evil eye and magic spells, and the Jews fully believed in them. The whole phallic cult grew up as a defense against the evil eye. The organs of reproduction were thought to be the only fetish which could render it powerless. The evil eye gave origin to the first superstitions respecting prenatal marking of children, maternal impressions, and the cult was at one time well\hyp{}nigh universal.\fnc{The whole phallic cult grew up as a defense against \bibtextul{evil eye.} \bibexpl{The phrase “evil eye” without an article seems extremely stilted, while such forms may have been used somewhere by some author, the committee could find no instances of such usage --- even in texts of the nineteenth and early twentieth centuries --- and certainly could not find a reason not to amend the text here to conform with normal practice.}}
\vs p087 5:6 Envy is a deep\hyp{}seated human trait; therefore did primitive man ascribe it to his early gods. And since man had once practiced deception upon the ghosts, he soon began to deceive the spirits. Said he, “If the spirits are jealous of our beauty and prosperity, we will disfigure ourselves and speak lightly of our success.” Early humility was not, therefore, debasement of ego but rather an attempt to foil and deceive the envious spirits.
\vs p087 5:7 The method adopted to prevent the spirits from becoming jealous of human prosperity was to heap vituperation upon some lucky or much loved thing or person. The custom of depreciating complimentary remarks regarding oneself or family had its origin in this way, and it eventually evolved into civilized modesty, restraint, and courtesy. In keeping with the same motive, it became the fashion to look ugly. Beauty aroused the envy of spirits; it betokened sinful human pride. The savage sought for an ugly name. This feature of the cult was a great handicap to the advancement of art, and it long kept the world somber and ugly.
\vs p087 5:8 \pc Under the spirit cult, life was at best a gamble, the result of spirit control. One’s future was not the result of effort, industry, or talent except as they might be utilized to influence the spirits. The ceremonies of spirit propitiation constituted a heavy burden, rendering life tedious and virtually unendurable. From age to age and from generation to generation, race after race has sought to improve this superghost doctrine, but no generation has ever yet dared to wholly reject it.
\vs p087 5:9 The intention and will of the spirits were studied by means of omens, oracles, and signs. And these spirit messages were interpreted by divination, soothsaying, magic, ordeals, and astrology. The whole cult was a scheme designed to placate, satisfy, and buy off the spirits through this disguised bribery.
\vs p087 5:10 And thus there grew up a new and expanded world philosophy consisting in:
\vs p087 5:11 \ublistelem{1.}\bibnobreakspace \bibemph{Duty ---} those things which must be done to keep the spirits favorably disposed, at least neutral.
\vs p087 5:12 \ublistelem{2.}\bibnobreakspace \bibemph{Right ---} the correct conduct and ceremonies designed to win the spirits actively to one’s interests.
\vs p087 5:13 \ublistelem{3.}\bibnobreakspace \bibemph{Truth ---} the correct understanding of, and attitude toward, spirits, and hence toward life and death.
\vs p087 5:14 \pc It was not merely out of curiosity that the ancients sought to know the future; they wanted to dodge ill luck. Divination was simply an attempt to avoid trouble. During these times, dreams were regarded as prophetic, while everything out of the ordinary was considered an omen. And even today the civilized races are cursed with the belief in signs, tokens, and other superstitious remnants of the advancing ghost cult of old. Slow, very slow, is man to abandon those methods whereby he so gradually and painfully ascended the evolutionary scale of life.
\usection{6.\bibnobreakspace Coercion and Exorcism}
\vs p087 6:1 When men believed in ghosts only, religious ritual was more personal, less organized, but the recognition of higher spirits necessitated the employment of “higher spiritual methods” in dealing with them. This attempt to improve upon, and to elaborate, the technique of spirit propitiation led directly to the creation of defenses against the spirits. Man felt helpless indeed before the uncontrollable forces operating in terrestrial life, and his feeling of inferiority drove him to attempt to find some compensating adjustment, some technique for evening the odds in the one\hyp{}sided struggle of man versus the cosmos.
\vs p087 6:2 In the early days of the cult, man’s efforts to influence ghost action were confined to propitiation, attempts by bribery to buy off ill luck. As the evolution of the ghost cult progressed to the concept of good as well as bad spirits, these ceremonies turned toward attempts of a more positive nature, efforts to win good luck. Man’s religion no longer was completely negativistic, nor did he stop with the effort to win good luck; he shortly began to devise schemes whereby he could compel spirit co\hyp{}operation. No longer does the religionist stand defenseless before the unceasing demands of the spirit phantasms of his own devising; the savage is beginning to invent weapons wherewith he may coerce spirit action and compel spirit assistance.
\vs p087 6:3 Man’s first efforts at defense were directed against the ghosts. As the ages passed, the living began to devise methods of resisting the dead. Many techniques were developed for frightening ghosts and driving them away, among which may be cited the following:
\vs p087 6:4 \ublistelem{1.}\bibnobreakspace Cutting off the head and tying up the body in the grave.
\vs p087 6:5 \ublistelem{2.}\bibnobreakspace Stoning the death house.
\vs p087 6:6 \ublistelem{3.}\bibnobreakspace Castration or breaking the legs of the corpse.
\vs p087 6:7 \ublistelem{4.}\bibnobreakspace Burying under stones, one origin of the modern tombstone.
\vs p087 6:8 \ublistelem{5.}\bibnobreakspace Cremation, a later\hyp{}day invention to prevent ghost trouble.
\vs p087 6:9 \ublistelem{6.}\bibnobreakspace Casting the body into the sea.
\vs p087 6:10 \ublistelem{7.}\bibnobreakspace Exposure of the body to be eaten by wild animals.
\vs p087 6:11 \pc Ghosts were supposed to be disturbed and frightened by noise; shouting, bells, and drums drove them away from the living; and these ancient methods are still in vogue at “wakes” for the dead. Foul\hyp{}smelling concoctions were utilized to banish unwelcome spirits. Hideous images of the spirits were constructed so that they would flee in haste when they beheld themselves. It was believed that dogs could detect the approach of ghosts, and that they gave warning by howling; that cocks would crow when they were near. The use of a cock as a weather vane is in perpetuation of this superstition.
\vs p087 6:12 Water was regarded as the best protection against ghosts. Holy water was superior to all other forms, water in which the priests had washed their feet. Both fire and water were believed to constitute impassable barriers to ghosts. The Romans carried water three times around the corpse; in the twentieth century the body is sprinkled with holy water, and hand washing at the cemetery is still a Jewish ritual. Baptism was a feature of the later water ritual; primitive bathing was a religious ceremony. Only in recent times has bathing become a sanitary practice.
\vs p087 6:13 But man did not stop with ghost coercion; through religious ritual and other practices he was soon attempting to compel spirit action. Exorcism was the employment of one spirit to control or banish another, and these tactics were also utilized for frightening ghosts and spirits. The dual\hyp{}spiritism concept of good and bad forces offered man ample opportunity to attempt to pit one agency against another, for, if a powerful man could vanquish a weaker one, then certainly a strong spirit could dominate an inferior ghost. Primitive cursing was a coercive practice designed to overawe minor spirits. Later this custom expanded into the pronouncing of curses upon enemies.
\vs p087 6:14 It was long believed that by reverting to the usages of the more ancient mores the spirits and demigods could be forced into desirable action. Modern man is guilty of the same procedure. You address one another in common, everyday language, but when you engage in prayer, you resort to the older style of another generation, the so\hyp{}called solemn style.
\vs p087 6:15 This doctrine also explains many religious\hyp{}ritual reversions of a sex nature, such as temple prostitution. These reversions to primitive customs were considered sure guards against many calamities. And with these simple\hyp{}minded peoples all such performances were entirely free from what modern man would term promiscuity.
\vs p087 6:16 Next came the practice of ritual vows, soon to be followed by religious pledges and sacred oaths. Most of these oaths were accompanied by self\hyp{}torture and self\hyp{}mutilation; later on, by fasting and prayer. Self\hyp{}denial was subsequently looked upon as being a sure coercive; this was especially true in the matter of sex suppression. And so primitive man early developed a decided austerity in his religious practices, a belief in the efficacy of self\hyp{}torture and self\hyp{}denial as rituals capable of coercing the unwilling spirits to react favorably toward all such suffering and deprivation.
\vs p087 6:17 \pc Modern man no longer attempts openly to coerce the spirits, though he still evinces a disposition to bargain with Deity. And he still swears, knocks on wood, crosses his fingers, and follows expectoration with some trite phrase; once it was a magical formula.
\usection{7.\bibnobreakspace Nature of Cultism}
\vs p087 7:1 The cult type of social organization persisted because it provided a symbolism for the preservation and stimulation of moral sentiments and religious loyalties. The cult grew out of the traditions of “old families” and was perpetuated as an established institution; all families have a cult of some sort. Every inspiring ideal grasps for some perpetuating symbolism --- seeks some technique for cultural manifestation which will insure survival and augment realization --- and the cult achieves this end by fostering and gratifying emotion.
\vs p087 7:2 From the dawn of civilization every appealing movement in social culture or religious advancement has developed a ritual, a symbolic ceremonial. The more this ritual has been an unconscious growth, the stronger it has gripped its devotees. The cult preserved sentiment and satisfied emotion, but it has always been the greatest obstacle to social reconstruction and spiritual progress.
\vs p087 7:3 Notwithstanding that the cult has always retarded social progress, it is regrettable that so many modern believers in moral standards and spiritual ideals have no adequate symbolism --- no cult of mutual support --- nothing to \bibemph{belong} to. But a religious cult cannot be manufactured; it must grow. And those of no two groups will be identical unless their rituals are arbitrarily standardized by authority.
\vs p087 7:4 The early Christian cult was the most effective, appealing, and enduring of any ritual ever conceived or devised, but much of its value has been destroyed in a scientific age by the destruction of so many of its original underlying tenets. The Christian cult has been devitalized by the loss of many fundamental ideas.
\vs p087 7:5 \pc In the past, truth has grown rapidly and expanded freely when the cult has been elastic, the symbolism expansile. Abundant truth and an adjustable cult have favored rapidity of social progression. A meaningless cult vitiates religion when it attempts to supplant philosophy and to enslave reason; a genuine cult grows.
\vs p087 7:6 \pc Regardless of the drawbacks and handicaps, every new revelation of truth has given rise to a new cult, and even the restatement of the religion of Jesus must develop a new and appropriate symbolism. Modern man must find some adequate symbolism for his new and expanding ideas, ideals, and loyalties. This enhanced symbol must arise out of religious living, spiritual experience. And this higher symbolism of a higher civilization must be predicated on the concept of the Fatherhood of God and be pregnant with the mighty ideal of the brotherhood of man.
\vs p087 7:7 The old cults were too egocentric; the new must be the outgrowth of applied love. The new cult must, like the old, foster sentiment, satisfy emotion, and promote loyalty; but it must do more: It must facilitate spiritual progress, enhance cosmic meanings, augment moral values, encourage social development, and stimulate a high type of personal religious living. The new cult must provide supreme goals of living which are both temporal and eternal --- social and spiritual.
\vs p087 7:8 No cult can endure and contribute to the progress of social civilization and individual spiritual attainment unless it is based on the biologic, sociologic, and religious significance of the \bibemph{home.} A surviving cult must symbolize that which is permanent in the presence of unceasing change; it must glorify that which unifies the stream of ever\hyp{}changing social metamorphosis. It must recognize true meanings, exalt beautiful relations, and glorify the good values of real nobility.
\vs p087 7:9 But the great difficulty of finding a new and satisfying symbolism is because modern men, as a group, adhere to the scientific attitude, eschew superstition, and abhor ignorance, while as individuals they all crave mystery and venerate the unknown. No cult can survive unless it embodies some masterful mystery and conceals some worthful unattainable. Again, the new symbolism must not only be significant for the group but also meaningful to the individual. The forms of any serviceable symbolism must be those which the individual can carry out on his own initiative, and which he can also enjoy with his fellows. If the new cult could only be dynamic instead of static, it might really contribute something worth while to the progress of mankind, both temporal and spiritual.
\vs p087 7:10 But a cult --- a symbolism of rituals, slogans, or goals --- will not function if it is too complex. And there must be the demand for devotion, the response of loyalty. Every effective religion unerringly develops a worthy symbolism, and its devotees would do well to prevent the crystallization of such a ritual into cramping, deforming, and stifling stereotyped ceremonials which can only handicap and retard all social, moral, and spiritual progress. No cult can survive if it retards moral growth and fails to foster spiritual progress. The cult is the skeletal structure around which grows the living and dynamic body of personal spiritual experience --- true religion.
\vsetoff
\vs p087 7:11 [Presented by a Brilliant Evening Star of Nebadon.]
