\upaper{86}{Early Evolution of Religion}
\vs p086 0:1 THE evolution of religion from the preceding and primitive worship urge is not dependent on revelation. The normal functioning of the human mind under the directive influence of the sixth and seventh mind\hyp{}adjutants of universal spirit bestowal is wholly sufficient to insure such development.
\vs p086 0:2 Man’s earliest prereligious fear of the forces of nature gradually became religious as nature became personalized, spiritized, and eventually deified in human consciousness. Religion of a primitive type was therefore a natural biologic consequence of the psychologic inertia of evolving animal minds after such minds had once entertained concepts of the supernatural.
\usection{1.\bibnobreakspace Chance: Good Luck and Bad Luck}
\vs p086 1:1 Aside from the natural worship urge, early evolutionary religion had its roots of origin in the human experiences of chance --- so\hyp{}called luck, commonplace happenings. Primitive man was a food hunter. The results of hunting must ever vary, and this gives certain origin to those experiences which man interprets as \bibemph{good luck} and \bibemph{bad luck.} Mischance was a great factor in the lives of men and women who lived constantly on the ragged edge of a precarious and harassed existence.
\vs p086 1:2 The limited intellectual horizon of the savage so concentrates the attention upon chance that luck becomes a constant factor in his life. Primitive Urantians struggled for existence, not for a standard of living; they lived lives of peril in which chance played an important role. The constant dread of unknown and unseen calamity hung over these savages as a cloud of despair which effectively eclipsed every pleasure; they lived in constant dread of doing something that would bring bad luck. Superstitious savages always feared a run of good luck; they viewed such good fortune as a certain harbinger of calamity.
\vs p086 1:3 This ever\hyp{}present dread of bad luck was paralyzing. Why work hard and reap bad luck --- nothing for something --- when one might drift along and encounter good luck --- something for nothing? Unthinking men forget good luck --- take it for granted --- but they painfully remember bad luck.
\vs p086 1:4 Early man lived in uncertainty and in constant fear of chance --- bad luck. Life was an exciting game of chance; existence was a gamble. It is no wonder that partially civilized people still believe in chance and evince lingering predispositions to gambling. Primitive man alternated between two potent interests: the passion of getting something for nothing and the fear of getting nothing for something. And this gamble of existence was the main interest and the supreme fascination of the early savage mind.
\vs p086 1:5 The later herders held the same views of chance and luck, while the still later agriculturists were increasingly conscious that crops were immediately influenced by many things over which man had little or no control. The farmer found himself the victim of drought, floods, hail, storms, pests, and plant diseases, as well as heat and cold. And as all of these natural influences affected individual prosperity, they were regarded as good luck or bad luck.
\vs p086 1:6 This notion of chance and luck strongly pervaded the philosophy of all ancient peoples. Even in recent times in the Wisdom of Solomon it is said: “I returned and saw that the race is not to the swift, nor the battle to the strong, neither bread to the wise, nor riches to men of understanding, nor favor to men of skill; but fate and chance befall them all. For man knows not his fate; as fishes are taken in an evil net, and as birds are caught in a snare, so are the sons of men snared in an evil time when it falls suddenly upon them.”
\usection{2.\bibnobreakspace The Personification of Chance}
\vs p086 2:1 Anxiety was a natural state of the savage mind. When men and women fall victims to excessive anxiety, they are simply reverting to the natural estate of their far\hyp{}distant ancestors; and when anxiety becomes actually painful, it inhibits activity and unfailingly institutes evolutionary changes and biologic adaptations. Pain and suffering are essential to progressive evolution.
\vs p086 2:2 The struggle for life is so painful that certain backward tribes even yet howl and lament over each new sunrise. Primitive man constantly asked, “Who is tormenting me?” Not finding a material source for his miseries, he settled upon a spirit explanation. And so was religion born of the fear of the mysterious, the awe of the unseen, and the dread of the unknown. Nature fear thus became a factor in the struggle for existence first because of chance and then because of mystery.
\vs p086 2:3 \pc The primitive mind was logical but contained few ideas for intelligent association; the savage mind was uneducated, wholly unsophisticated. If one event followed another, the savage considered them to be cause and effect. What civilized man regards as superstition was just plain ignorance in the savage. Mankind has been slow to learn that there is not necessarily any relationship between purposes and results. Human beings are only just beginning to realize that the reactions of existence appear between acts and their consequences. The savage strives to personalize everything intangible and abstract, and thus both nature and chance become personalized as ghosts --- spirits --- and later on as gods.
\vs p086 2:4 \pc Man naturally tends to believe that which he deems best for him, that which is in his immediate or remote interest; self\hyp{}interest largely obscures logic. The difference between the minds of savage and civilized men is more one of content than of nature, of degree rather than of quality.
\vs p086 2:5 But to continue to ascribe things difficult of comprehension to supernatural causes is nothing less than a lazy and convenient way of avoiding all forms of intellectual hard work. Luck is merely a term coined to cover the inexplicable in any age of human existence; it designates those phenomena which men are unable or unwilling to penetrate. Chance is a word which signifies that man is too ignorant or too indolent to determine causes. Men regard a natural occurrence as an accident or as bad luck only when they are destitute of curiosity and imagination, when the races lack initiative and adventure. Exploration of the phenomena of life sooner or later destroys man’s belief in chance, luck, and so\hyp{}called accidents, substituting therefor a universe of law and order wherein all effects are preceded by definite causes. Thus is the fear of existence replaced by the joy of living.
\vs p086 2:6 The savage looked upon all nature as alive, as possessed by something. Civilized man still kicks and curses those inanimate objects which get in his way and bump him. Primitive man never regarded anything as accidental; always was everything intentional. To primitive man the domain of fate, the function of luck, the spirit world, was just as unorganized and haphazard as was primitive society. Luck was looked upon as the whimsical and temperamental reaction of the spirit world; later on, as the humor of the gods.
\vs p086 2:7 But all religions did not develop from animism. Other concepts of the supernatural were contemporaneous with animism, and these beliefs also led to worship. Naturalism is not a religion --- it is the offspring of religion.
\usection{3.\bibnobreakspace Death --- The Inexplicable}
\vs p086 3:1 Death was the supreme shock to evolving man, the most perplexing combination of chance and mystery. Not the sanctity of life but the shock of death inspired fear and thus effectively fostered religion. Among savage peoples death was ordinarily due to violence, so that nonviolent death became increasingly mysterious. Death as a natural and expected end of life was not clear to the consciousness of primitive people, and it has required age upon age for man to realize its inevitability.
\vs p086 3:2 \pc Early man accepted life as a fact, while he regarded death as a visitation of some sort. All races have their legends of men who did not die, vestigial traditions of the early attitude toward death. Already in the human mind there existed the nebulous concept of a hazy and unorganized spirit world, a domain whence came all that is inexplicable in human life, and death was added to this long list of unexplained phenomena.
\vs p086 3:3 All human disease and natural death was at first believed to be due to spirit influence. Even at the present time some civilized races regard disease as having been produced by “the enemy” and depend upon religious ceremonies to effect healing. Later and more complex systems of theology still ascribe death to the action of the spirit world, all of which has led to such doctrines as original sin and the fall of man.
\vs p086 3:4 It was the realization of impotency before the mighty forces of nature, together with the recognition of human weakness before the visitations of sickness and death, that impelled the savage to seek for help from the supermaterial world, which he vaguely visualized as the source of these mysterious vicissitudes of life.
\usection{4.\bibnobreakspace The Death\hyp{}Survival Concept}
\vs p086 4:1 The concept of a supermaterial phase of mortal personality was born of the unconscious and purely accidental association of the occurrences of everyday life plus the ghost dream. The simultaneous dreaming about a departed chief by several members of his tribe seemed to constitute convincing evidence that the old chief had really returned in some form. It was all very real to the savage who would awaken from such dreams reeking with sweat, trembling, and screaming.
\vs p086 4:2 The dream origin of the belief in a future existence explains the tendency always to imagine unseen things in the terms of things seen. And presently this new dream\hyp{}ghost\hyp{}future\hyp{}life concept began effectively to antidote the death fear associated with the biologic instinct of self\hyp{}preservation.
\vs p086 4:3 Early man was also much concerned about his breath, especially in cold climates, where it appeared as a cloud when exhaled. The \bibemph{breath of life} was regarded as the one phenomenon which differentiated the living and the dead. He knew the breath could leave the body, and his dreams of doing all sorts of queer things while asleep convinced him that there was something immaterial about a human being. The most primitive idea of the human soul, the ghost, was derived from the breath\hyp{}dream idea\hyp{}system.
\vs p086 4:4 Eventually the savage conceived of himself as a double --- body and breath. The breath minus the body equaled a spirit, a ghost. While having a very definite human origin, ghosts, or spirits, were regarded as superhuman. And this belief in the existence of disembodied spirits seemed to explain the occurrence of the unusual, the extraordinary, the infrequent, and the inexplicable.
\vs p086 4:5 \pc The primitive doctrine of survival after death was not necessarily a belief in immortality. Beings who could not count over twenty could hardly conceive of infinity and eternity; they rather thought of recurring incarnations.
\vs p086 4:6 The orange race was especially given to belief in transmigration and reincarnation. This idea of reincarnation originated in the observance of hereditary and trait resemblance of offspring to ancestors. The custom of naming children after grandparents and other ancestors was due to belief in reincarnation. Some later\hyp{}day races believed that man died from three to seven times. This belief (residual from the teachings of Adam about the mansion worlds), and many other remnants of revealed religion, can be found among the otherwise absurd doctrines of twentieth\hyp{}century barbarians.
\vs p086 4:7 \pc Early man entertained no ideas of hell or future punishment. The savage looked upon the future life as just like this one, minus all ill luck. Later on, a separate destiny for good ghosts and bad ghosts --- heaven and hell --- was conceived. But since many primitive races believed that man entered the next life just as he left this one, they did not relish the idea of becoming old and decrepit. The aged much preferred to be killed before becoming too infirm.
\vs p086 4:8 Almost every group had a different idea regarding the destiny of the ghost soul. The Greeks believed that weak men must have weak souls; so they invented Hades as a fit place for the reception of such anemic souls; these unrobust specimens were also supposed to have shorter shadows. The early Andites thought their ghosts returned to the ancestral homelands. The Chinese and Egyptians once believed that soul and body remained together. Among the Egyptians this led to careful tomb construction and efforts at body preservation. Even modern peoples seek to arrest the decay of the dead. The Hebrews conceived that a phantom replica of the individual went down to Sheol; it could not return to the land of the living. They did make that important advance in the doctrine of the evolution of the soul.
\usection{5.\bibnobreakspace The Ghost\hyp{}Soul Concept}
\vs p086 5:1 The nonmaterial part of man has been variously termed ghost, spirit, shade, phantom, specter, and latterly \bibemph{soul.} The soul was early man’s dream double; it was in every way exactly like the mortal himself except that it was not responsive to touch. The belief in dream doubles led directly to the notion that all things animate and inanimate had souls as well as men. This concept tended long to perpetuate the nature\hyp{}spirit beliefs; the Eskimos still conceive that everything in nature has a spirit.
\vs p086 5:2 The ghost soul could be heard and seen, but not touched. Gradually the dream life of the race so developed and expanded the activities of this evolving spirit world that death was finally regarded as “giving up the ghost.” All primitive tribes, except those little above animals, have developed some concept of the soul. As civilization advances, this superstitious concept of the soul is destroyed, and man is wholly dependent on revelation and personal religious experience for his new idea of the soul as the joint creation of the God\hyp{}knowing mortal mind and its indwelling divine spirit, the Thought Adjuster.
\vs p086 5:3 Early mortals usually failed to differentiate the concepts of an indwelling spirit and a soul of evolutionary nature. The savage was much confused as to whether the ghost soul was native to the body or was an external agency in possession of the body. The absence of reasoned thought in the presence of perplexity explains the gross inconsistencies of the savage view of souls, ghosts, and spirits.
\vs p086 5:4 The soul was thought of as being related to the body as the perfume to the flower. The ancients believed that the soul could leave the body in various ways, as in:
\vs p086 5:5 \ublistelem{1.}\bibnobreakspace Ordinary and transient fainting.
\vs p086 5:6 \ublistelem{2.}\bibnobreakspace Sleeping, natural dreaming.
\vs p086 5:7 \ublistelem{3.}\bibnobreakspace Coma and unconsciousness associated with disease and accidents.
\vs p086 5:8 \ublistelem{4.}\bibnobreakspace Death, permanent departure.
\vs p086 5:9 \pc The savage looked upon sneezing as an abortive attempt of the soul to escape from the body. Being awake and on guard, the body was able to thwart the soul’s attempted escape. Later on, sneezing was always accompanied by some religious expression, such as “God bless you!”
\vs p086 5:10 \pc Early in evolution sleep was regarded as proving that the ghost soul could be absent from the body, and it was believed that it could be called back by speaking or shouting the sleeper’s name. In other forms of unconsciousness the soul was thought to be farther away, perhaps trying to escape for good --- impending death. Dreams were looked upon as the experiences of the soul during sleep while temporarily absent from the body. The savage believes his dreams to be just as real as any part of his waking experience. The ancients made a practice of awaking sleepers gradually so that the soul might have time to get back into the body.
\vs p086 5:11 All down through the ages men have stood in awe of the apparitions of the night season, and the Hebrews were no exception. They truly believed that God spoke to them in dreams, despite the injunctions of Moses against this idea. And Moses was right, for ordinary dreams are not the methods employed by the personalities of the spiritual world when they seek to communicate with material beings.
\vs p086 5:12 The ancients believed that souls could enter animals or even inanimate objects. This culminated in the werewolf ideas of animal identification. A person could be a law\hyp{}abiding citizen by day, but when he fell asleep, his soul could enter a wolf or some other animal to prowl about on nocturnal depredations.
\vs p086 5:13 Primitive men thought that the soul was associated with the breath, and that its qualities could be imparted or transferred by the breath. The brave chief would breathe upon the newborn child, thereby imparting courage. Among early Christians the ceremony of bestowing the Holy Spirit was accompanied by breathing on the candidates. Said the Psalmist: “By the word of the Lord were the heavens made and all the host of them by the breath of his mouth.” It was long the custom of the eldest son to try to catch the last breath of his dying father.
\vs p086 5:14 The shadow came, later on, to be feared and revered equally with the breath. The reflection of oneself in the water was also sometimes looked upon as proof of the double self, and mirrors were regarded with superstitious awe. Even now many civilized persons turn the mirror to the wall in the event of death. Some backward tribes still believe that the making of pictures, drawings, models, or images removes all or a part of the soul from the body; hence such are forbidden.
\vs p086 5:15 The soul was generally thought of as being identified with the breath, but it was also located by various peoples in the head, hair, heart, liver, blood, and fat. The “crying out of Abel’s blood from the ground” is expressive of the onetime belief in the presence of the ghost in the blood. The Semites taught that the soul resided in the bodily fat, and among many the eating of animal fat was taboo. Head hunting was a method of capturing an enemy’s soul, as was scalping. In recent times the eyes have been regarded as the windows of the soul.
\vs p086 5:16 Those who held the doctrine of three or four souls believed that the loss of one soul meant discomfort, two illness, three death. One soul lived in the breath, one in the head, one in the hair, one in the heart. The sick were advised to stroll about in the open air with the hope of recapturing their strayed souls. The greatest of the medicine men were supposed to exchange the sick soul of a diseased person for a new one, the “new birth.”
\vs p086 5:17 The children of Badonan developed a belief in two souls, the breath and the shadow. The early Nodite races regarded man as consisting of two persons, soul and body. This philosophy of human existence was later reflected in the Greek viewpoint. The Greeks themselves believed in three souls; the vegetative resided in the stomach, the animal in the heart, the intellectual in the head. The Eskimos believe that man has three parts: body, soul, and name.\fnc{The children of \bibtextul{Badanon} developed a belief in two souls, \bibexpl{Badonan is the correct spelling; Badanon was, no doubt, the result of an inadvertent key transposition.}}
\usection{6.\bibnobreakspace The Ghost\hyp{}Spirit Environment}
\vs p086 6:1 Man inherited a natural environment, acquired a social environment, and imagined a ghost environment. The state is man’s reaction to his natural environment, the home to his social environment, the church to his illusory ghost environment.
\vs p086 6:2 Very early in the history of mankind the realities of the imaginary world of ghosts and spirits became universally believed, and this newly imagined spirit world became a power in primitive society. The mental and moral life of all mankind was modified for all time by the appearance of this new factor in human thinking and acting.
\vs p086 6:3 Into this major premise of illusion and ignorance, mortal fear has packed all of the subsequent superstition and religion of primitive peoples. This was man’s only religion up to the times of revelation, and today many of the world’s races have only this crude religion of evolution.
\vs p086 6:4 As evolution progressed, good luck became associated with good spirits and bad luck with bad spirits. The discomfort of enforced adaptation to a changing environment was regarded as ill luck, the displeasure of the spirit ghosts. Primitive man slowly evolved religion out of his innate worship urge and his misconception of chance. Civilized man provides schemes of insurance to overcome these chance occurrences; modern science puts an actuary with mathematical reckoning in the place of fictitious spirits and whimsical gods.
\vs p086 6:5 Each passing generation smiles at the foolish superstitions of its ancestors while it goes on entertaining those fallacies of thought and worship which will give cause for further smiling on the part of enlightened posterity.
\vs p086 6:6 \pc But at last the mind of primitive man was occupied with thoughts which transcended all of his inherent biologic urges; at last man was about to evolve an art of living based on something more than response to material stimuli. The beginnings of a primitive philosophic life policy were emerging. A supernatural standard of living was about to appear, for, if the spirit ghost in anger visits ill luck and in pleasure good fortune, then must human conduct be regulated accordingly. The concept of right and wrong had at last evolved; and all of this long before the times of any revelation on earth.
\vs p086 6:7 With the emergence of these concepts, there was initiated the long and wasteful struggle to appease the ever\hyp{}displeased spirits, the slavish bondage to evolutionary religious fear, that long waste of human effort upon tombs, temples, sacrifices, and priesthoods. It was a terrible and frightful price to pay, but it was worth all it cost, for man therein achieved a natural consciousness of relative right and wrong; human ethics was born!
\usection{7.\bibnobreakspace The Function of Primitive Religion}
\vs p086 7:1 The savage felt the need of insurance, and he therefore willingly paid his burdensome premiums of fear, superstition, dread, and priest gifts toward his policy of magic insurance against ill luck. Primitive religion was simply the payment of premiums on insurance against the perils of the forests; civilized man pays material premiums against the accidents of industry and the exigencies of modern modes of living.
\vs p086 7:2 Modern society is removing the business of insurance from the realm of priests and religion, placing it in the domain of economics. Religion is concerning itself increasingly with the insurance of life beyond the grave. Modern men, at least those who think, no longer pay wasteful premiums to control luck. Religion is slowly ascending to higher philosophic levels in contrast with its former function as a scheme of insurance against bad luck.
\vs p086 7:3 But these ancient ideas of religion prevented men from becoming fatalistic and hopelessly pessimistic; they believed they could at least do something to influence fate. The religion of ghost fear impressed upon men that they must \bibemph{regulate their conduct,} that there was a supermaterial world which was in control of human destiny.
\vs p086 7:4 Modern civilized races are just emerging from ghost fear as an explanation of luck and the commonplace inequalities of existence. Mankind is achieving emancipation from the bondage of the ghost\hyp{}spirit explanation of ill luck. But while men are giving up the erroneous doctrine of a spirit cause of the vicissitudes of life, they exhibit a surprising willingness to accept an almost equally fallacious teaching which bids them attribute all human inequalities to political misadaptation, social injustice, and industrial competition. But new legislation, increasing philanthropy, and more industrial reorganization, however good in and of themselves, will not remedy the facts of birth and the accidents of living. Only comprehension of facts and wise manipulation within the laws of nature will enable man to get what he wants and to avoid what he does not want. Scientific knowledge, leading to scientific action, is the only antidote for so\hyp{}called accidental ills.
\vs p086 7:5 \pc Industry, war, slavery, and civil government arose in response to the social evolution of man in his natural environment; religion similarly arose as his response to the illusory environment of the imaginary ghost world. Religion was an evolutionary development of self\hyp{}maintenance, and it has worked, notwithstanding that it was originally erroneous in concept and utterly illogical.
\vs p086 7:6 Primitive religion prepared the soil of the human mind, by the powerful and awesome force of false fear, for the bestowal of a bona fide spiritual force of supernatural origin, the Thought Adjuster. And the divine Adjusters have ever since labored to transmute God\hyp{}fear into God\hyp{}love. Evolution may be slow, but it is unerringly effective.
\vsetoff
\vs p086 7:7 [Presented by an Evening Star of Nebadon.]
