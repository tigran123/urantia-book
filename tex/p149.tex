\upaper{149}{The Second Preaching Tour}
\vs p149 0:1 THE second public preaching tour of Galilee began on Sunday, October 3, A.D.\,28, and continued for almost three months, ending on December 30. Participating in this effort were Jesus and his twelve apostles, assisted by the newly recruited corps of 117 evangelists and by numerous other interested persons. On this tour they visited Gadara, Ptolemais, Japhia, Dabaritta, Megiddo, Jezreel, Scythopolis, Tarichea, Hippos, Gamala, Bethsaida\hyp{}Julias, and many other cities and villages.
\vs p149 0:2 Before the departure on this Sunday morning Andrew and Peter asked Jesus to give the final charge to the new evangelists, but the Master declined, saying that it was not his province to do those things which others could acceptably perform. After due deliberation it was decided that James Zebedee should administer the charge. At the conclusion of James’s remarks Jesus said to the evangelists: \textcolor{ubdarkred}{“Go now forth to do the work as you have been charged, and later on, when you have shown yourselves competent and faithful, I will ordain you to preach the gospel of the kingdom.”}
\vs p149 0:3 On this tour only James and John traveled with Jesus. Peter and the other apostles each took with them about one dozen of the evangelists and maintained close contact with them while they carried on their work of preaching and teaching. As fast as believers were ready to enter the kingdom, the apostles would administer baptism. Jesus and his two companions traveled extensively during these three months, often visiting two cities in one day to observe the work of the evangelists and to encourage them in their efforts to establish the kingdom. This entire second preaching tour was principally an effort to afford practical experience for this corps of 117 newly trained evangelists.
\vs p149 0:4 \pc Throughout this period and subsequently, up to the time of the final departure of Jesus and the twelve for Jerusalem, David Zebedee maintained a permanent headquarters for the work of the kingdom in his father’s house at Bethsaida. This was the clearinghouse for Jesus’ work on earth and the relay station for the messenger service which David carried on between the workers in various parts of Palestine and adjacent regions. He did all of this on his own initiative but with the approval of Andrew. David employed forty to fifty messengers in this intelligence division of the rapidly enlarging and extending work of the kingdom. While thus employed, he partially supported himself by spending some of his time at his old work of fishing.
\usection{1.\bibnobreakspace The Widespread Fame of Jesus}
\vs p149 1:1 By the time the camp at Bethsaida had been broken up, the fame of Jesus, particularly as a healer, had spread to all parts of Palestine and through all of Syria and the surrounding countries. For weeks after they left Bethsaida, the sick continued to arrive, and when they did not find the Master, on learning from David where he was, they would go in search of him. On this tour Jesus did not deliberately perform any so\hyp{}called miracles of healing. Nevertheless, scores of afflicted found restoration of health and happiness as a result of the reconstructive power of the intense faith which impelled them to seek for healing.
\vs p149 1:2 There began to appear about the time of this mission --- and continued throughout the remainder of Jesus’ life on earth --- a peculiar and unexplained series of healing phenomena. In the course of this three months’ tour more than one hundred men, women, and children from Judea, Idumea, Galilee, Syria, Tyre, and Sidon, and from beyond the Jordan were beneficiaries of this unconscious healing by Jesus and, returning to their homes, added to the enlargement of Jesus’ fame. And they did this notwithstanding that Jesus would, every time he observed one of these cases of spontaneous healing, directly charge the beneficiary to \textcolor{ubdarkred}{“tell no man.”}
\vs p149 1:3 \pc It was never revealed to us just what occurred in these cases of spontaneous or unconscious healing. The Master never explained to his apostles how these healings were effected, other than that on several occasions he merely said, \textcolor{ubdarkred}{“I perceive that power has gone forth from me.”} On one occasion he remarked when touched by an ailing child, “I perceive that life has gone forth from me.”
\vs p149 1:4 In the absence of direct word from the Master regarding the nature of these cases of spontaneous healing, it would be presuming on our part to undertake to explain how they were accomplished, but it will be permissible to record our opinion of all such healing phenomena. We believe that many of these apparent miracles of healing, as they occurred in the course of Jesus’ earth ministry, were the result of the coexistence of the following three powerful, potent, and associated influences:
\vs p149 1:5 \ublistelem{1.}\bibnobreakspace The presence of strong, dominant, and living faith in the heart of the human being who persistently sought healing, together with the fact that such healing was desired for its spiritual benefits rather than for purely physical restoration.
\vs p149 1:6 \pc \ublistelem{2.}\bibnobreakspace The existence, concomitant with such human faith, of the great sympathy and compassion of the incarnated and mercy\hyp{}dominated Creator Son of God, who actually possessed in his person almost unlimited and timeless creative healing powers and prerogatives.
\vs p149 1:7 \pc \ublistelem{3.}\bibnobreakspace Along with the faith of the creature and the life of the Creator it should also be noted that this God\hyp{}man was the personified expression of the Father’s will. If, in the contact of the human need and the divine power to meet it, the Father did not will otherwise, the two became one, and the healing occurred unconsciously to the human Jesus but was immediately recognized by his divine nature. The explanation, then, of many of these cases of healing must be found in a great law which has long been known to us, namely, What the Creator Son desires and the eternal Father wills IS.
\vs p149 1:8 \pc It is, then, our opinion that, in the personal presence of Jesus, certain forms of profound human faith were literally and truly \bibemph{compelling} in the manifestation of healing by certain creative forces and personalities of the universe who were at that time so intimately associated with the Son of Man. It therefore becomes a fact of record that Jesus did frequently suffer men to heal themselves in his presence by their powerful, personal faith.
\vs p149 1:9 Many others sought healing for wholly selfish purposes. A rich widow of Tyre, with her retinue, came seeking to be healed of her infirmities, which were many; and as she followed Jesus about through Galilee, she continued to offer more and more money, as if the power of God were something to be purchased by the highest bidder. But never would she become interested in the gospel of the kingdom; it was only the cure of her physical ailments that she sought.
\usection{2.\bibnobreakspace Attitude of the People}
\vs p149 2:1 Jesus understood the minds of men. He knew what was in the heart of man, and had his teachings been left as he presented them, the only commentary being the inspired interpretation afforded by his earth life, all nations and all religions of the world would speedily have embraced the gospel of the kingdom. The well\hyp{}meant efforts of Jesus’ early followers to restate his teachings so as to make them the more acceptable to certain nations, races, and religions, only resulted in making such teachings the less acceptable to all other nations, races, and religions.
\vs p149 2:2 The Apostle Paul, in his efforts to bring the teachings of Jesus to the favorable notice of certain groups in his day, wrote many letters of instruction and admonition. Other teachers of Jesus’ gospel did likewise, but none of them realized that some of these writings would subsequently be brought together by those who would set them forth as the embodiment of the teachings of Jesus. And so, while so\hyp{}called Christianity does contain more of the Master’s gospel than any other religion, it does also contain much that Jesus did not teach. Aside from the incorporation of many teachings from the Persian mysteries and much of the Greek philosophy into early Christianity, two great mistakes were made:
\vs p149 2:3 \ublistelem{1.}\bibnobreakspace The effort to connect the gospel teaching directly onto the Jewish theology, as illustrated by the Christian doctrines of the atonement --- the teaching that Jesus was the sacrificed Son who would satisfy the Father’s stern justice and appease the divine wrath. These teachings originated in a praiseworthy effort to make the gospel of the kingdom more acceptable to disbelieving Jews. Though these efforts failed as far as winning the Jews was concerned, they did not fail to confuse and alienate many honest souls in all subsequent generations.
\vs p149 2:4 \pc \ublistelem{2.}\bibnobreakspace The second great blunder of the Master’s early followers, and one which all subsequent generations have persisted in perpetuating, was to organize the Christian teaching so completely about the \bibemph{person} of Jesus. This overemphasis of the personality of Jesus in the theology of Christianity has worked to obscure his teachings, and all of this has made it increasingly difficult for Jews, Mohammedans, Hindus, and other Eastern religionists to accept the teachings of Jesus. We would not belittle the place of the person of Jesus in a religion which might bear his name, but we would not permit such consideration to eclipse his inspired life or to supplant his saving message: the fatherhood of God and the brotherhood of man.
\vs p149 2:5 \pc The teachers of the religion of Jesus should approach other religions with the recognition of the truths which are held in common (many of which come directly or indirectly from Jesus’ message) while they refrain from placing so much emphasis on the differences.
\vs p149 2:6 \pc While, at that particular time, the fame of Jesus rested chiefly upon his reputation as a healer, it does not follow that it continued so to rest. As time passed, more and more he was sought for spiritual help. But it was the physical cures that made the most direct and immediate appeal to the common people. Jesus was increasingly sought by the victims of moral enslavement and mental harassments, and he invariably taught them the way of deliverance. Fathers sought his advice regarding the management of their sons, and mothers came for help in the guidance of their daughters. Those who sat in darkness came to him, and he revealed to them the light of life. His ear was ever open to the sorrows of mankind, and he always helped those who sought his ministry.
\vs p149 2:7 When the Creator himself was on earth, incarnated in the likeness of mortal flesh, it was inevitable that some extraordinary things should happen. But you should never approach Jesus through these so\hyp{}called miraculous occurrences. Learn to approach the miracle through Jesus, but do not make the mistake of approaching Jesus through the miracle. And this admonition is warranted, notwithstanding that Jesus of Nazareth is the only founder of a religion who performed supermaterial acts on earth.
\vs p149 2:8 \pc The most astonishing and the most revolutionary feature of Michael’s mission on earth was his attitude toward women. In a day and generation when a man was not supposed to salute even his own wife in a public place, Jesus dared to take women along as teachers of the gospel in connection with his third tour of Galilee. And he had the consummate courage to do this in the face of the rabbinic teaching which declared that it was “better that the words of the law should be burned than delivered to women.”
\vs p149 2:9 In one generation Jesus lifted women out of the disrespectful oblivion and the slavish drudgery of the ages. And it is the one shameful thing about the religion that presumed to take Jesus’ name that it lacked the moral courage to follow this noble example in its subsequent attitude toward women.
\vs p149 2:10 \pc As Jesus mingled with the people, they found him entirely free from the superstitions of that day. He was free from religious prejudices; he was never intolerant. He had nothing in his heart resembling social antagonism. While he complied with the good in the religion of his fathers, he did not hesitate to disregard man\hyp{}made traditions of superstition and bondage. He dared to teach that catastrophes of nature, accidents of time, and other calamitous happenings are not visitations of divine judgments or mysterious dispensations of Providence. He denounced slavish devotion to meaningless ceremonials and exposed the fallacy of materialistic worship. He boldly proclaimed man’s spiritual freedom and dared to teach that mortals of the flesh are indeed and in truth sons of the living God.
\vs p149 2:11 Jesus transcended all the teachings of his forebears when he boldly substituted clean hearts for clean hands as the mark of true religion. He put reality in the place of tradition and swept aside all pretensions of vanity and hypocrisy. And yet this fearless man of God did not give vent to destructive criticism or manifest an utter disregard of the religious, social, economic, and political usages of his day. He was not a militant revolutionist; he was a progressive evolutionist. He engaged in the destruction of that which \bibemph{was} only when he simultaneously offered his fellows the superior thing which \bibemph{ought to be.}
\vs p149 2:12 \pc Jesus received the obedience of his followers without exacting it. Only three men who received his personal call refused to accept the invitation to discipleship. He exercised a peculiar drawing power over men, but he was not dictatorial. He commanded confidence, and no man ever resented his giving a command. He assumed absolute authority over his disciples, but no one ever objected. He permitted his followers to call him Master.
\vs p149 2:13 The Master was admired by all who met him except by those who entertained deep\hyp{}seated religious prejudices or those who thought they discerned political dangers in his teachings. Men were astonished at the originality and authoritativeness of his teaching. They marveled at his patience in dealing with backward and troublesome inquirers. He inspired hope and confidence in the hearts of all who came under his ministry. Only those who had not met him feared him, and he was hated only by those who regarded him as the champion of that truth which was destined to overthrow the evil and error which they had determined to hold in their hearts at all cost.
\vs p149 2:14 On both friends and foes he exercised a strong and peculiarly fascinating influence. Multitudes would follow him for weeks, just to hear his gracious words and behold his simple life. Devoted men and women loved Jesus with a well\hyp{}nigh superhuman affection. And the better they knew him the more they loved him. And all this is still true; even today and in all future ages, the more man comes to know this God\hyp{}man, the more he will love and follow after him.
\usection{3.\bibnobreakspace Hostility of the Religious Leaders}
\vs p149 3:1 Notwithstanding the favorable reception of Jesus and his teachings by the common people, the religious leaders at Jerusalem became increasingly alarmed and antagonistic. The Pharisees had formulated a systematic and dogmatic theology. Jesus was a teacher who taught as the occasion served; he was not a systematic teacher. Jesus taught not so much from the law as from life, by parables. (And when he employed a parable for illustrating his message, he designed to utilize just \bibemph{one} feature of the story for that purpose. Many wrong ideas concerning the teachings of Jesus may be secured by attempting to make allegories out of his parables.)
\vs p149 3:2 The religious leaders at Jerusalem were becoming well\hyp{}nigh frantic as a result of the recent conversion of young Abraham and by the desertion of the three spies who had been baptized by Peter, and who were now out with the evangelists on this second preaching tour of Galilee. The Jewish leaders were increasingly blinded by fear and prejudice, while their hearts were hardened by the continued rejection of the appealing truths of the gospel of the kingdom. When men shut off the appeal to the spirit that dwells within them, there is little that can be done to modify their attitude.
\vs p149 3:3 When Jesus first met with the evangelists at the Bethsaida camp, in concluding his address, he said: \textcolor{ubdarkred}{“You should remember that in body and mind --- emotionally --- men react individually. The only \bibemph{uniform} thing about men is the indwelling spirit. Though divine spirits may vary somewhat in the nature and extent of their experience, they react uniformly to all spiritual appeals. Only through, and by appeal to, this spirit can mankind ever attain unity and brotherhood.”} But many of the leaders of the Jews had closed the doors of their hearts to the spiritual appeal of the gospel. From this day on they ceased not to plan and plot for the Master’s destruction. They were convinced that Jesus must be apprehended, convicted, and executed as a religious offender, a violator of the cardinal teachings of the Jewish sacred law.
\usection{4.\bibnobreakspace Progress of the Preaching Tour}
\vs p149 4:1 Jesus did very little public work on this preaching tour, but he conducted many evening classes with the believers in most of the cities and villages where he chanced to sojourn with James and John. At one of these evening sessions one of the younger evangelists asked Jesus a question about anger, and the Master, among other things, said in reply:\fnc{\ldots{}a question about anger, and the \bibtextul{Master} among other \bibtextul{things} said, in reply: \bibexpl{See also note for \bibref[149:7.1]{p0149 7:1}. This sentence required two edits to make it flow correctly: at this location a comma was inserted after “the Master” and per the following item, a pre-existing comma that originally followed “said” was moved in front of it --- to follow “things”}}
\vs p149 4:2 \pc \textcolor{ubdarkred}{“Anger is a material manifestation which represents, in a general way, the measure of the failure of the spiritual nature to gain control of the combined intellectual and physical natures. Anger indicates your lack of tolerant brotherly love plus your lack of self\hyp{}respect and self\hyp{}control. Anger depletes the health, debases the mind, and handicaps the spirit teacher of man’s soul. Have you not read in the Scriptures that ‘wrath kills the foolish man,’ and that man ‘tears himself in his anger’? That ‘he who is slow of wrath is of great understanding,’ while ‘he who is hasty of temper exalts folly’? You all know that ‘a soft answer turns away wrath,’ and how ‘grievous words stir up anger.’ ‘Discretion defers anger,’ while ‘he who has no control over his own self is like a defenseless city without walls.’ ‘Wrath is cruel and anger is outrageous.’ ‘Angry men stir up strife, while the furious multiply their transgressions.’ ‘Be not hasty in spirit, for anger rests in the bosom of fools.’”} Before Jesus ceased speaking, he said further: \textcolor{ubdarkred}{“Let your hearts be so dominated by love that your spirit guide will have little trouble in delivering you from the tendency to give vent to those outbursts of animal anger which are inconsistent with the status of divine sonship.”}
\vs p149 4:3 \pc On this same occasion the Master talked to the group about the desirability of possessing well\hyp{}balanced characters. He recognized that it was necessary for most men to devote themselves to the mastery of some vocation, but he deplored all tendency toward overspecialization, toward becoming narrow\hyp{}minded and circumscribed in life’s activities. He called attention to the fact that any virtue, if carried to extremes, may become a vice. Jesus always preached temperance and taught consistency --- proportionate adjustment of life problems. He pointed out that overmuch sympathy and pity may degenerate into serious emotional instability; that enthusiasm may drive on into fanaticism. He discussed one of their former associates whose imagination had led him off into visionary and impractical undertakings. At the same time he warned them against the dangers of the dullness of overconservative mediocrity.
\vs p149 4:4 And then Jesus discoursed on the dangers of courage and faith, how they sometimes lead unthinking souls on to recklessness and presumption. He also showed how prudence and discretion, when carried too far, lead to cowardice and failure. He exhorted his hearers to strive for originality while they shunned all tendency toward eccentricity. He pleaded for sympathy without sentimentality, piety without sanctimoniousness. He taught reverence free from fear and superstition.
\vs p149 4:5 It was not so much what Jesus taught about the balanced character that impressed his associates as the fact that his own life was such an eloquent exemplification of his teaching. He lived in the midst of stress and storm, but he never wavered. His enemies continually laid snares for him, but they never entrapped him. The wise and learned endeavored to trip him, but he did not stumble. They sought to embroil him in debate, but his answers were always enlightening, dignified, and final. When he was interrupted in his discourses with multitudinous questions, his answers were always significant and conclusive. Never did he resort to ignoble tactics in meeting the continuous pressure of his enemies, who did not hesitate to employ every sort of false, unfair, and unrighteous mode of attack upon him.
\vs p149 4:6 While it is true that many men and women must assiduously apply themselves to some definite pursuit as a livelihood vocation, it is nevertheless wholly desirable that human beings should cultivate a wide range of cultural familiarity with life as it is lived on earth. Truly educated persons are not satisfied with remaining in ignorance of the lives and doings of their fellows.
\usection{5.\bibnobreakspace Lesson Regarding Contentment}
\vs p149 5:1 When Jesus was visiting the group of evangelists working under the supervision of Simon Zelotes, during their evening conference Simon asked the Master: “Why are some persons so much more happy and contented than others? Is contentment a matter of religious experience?” Among other things, Jesus said in answer to Simon’s question:
\vs p149 5:2 \pc \textcolor{ubdarkred}{“Simon, some persons are naturally more happy than others. Much, very much, depends upon the willingness of man to be led and directed by the Father’s spirit which lives within him. Have you not read in the Scriptures the words of the wise man, ‘The spirit of man is the candle of the Lord, searching all the inward parts’? And also that such spirit\hyp{}led mortals say: ‘The lines are fallen to me in pleasant places; yes, I have a goodly heritage.’ ‘A little that a righteous man has is better than the riches of many wicked,’ for ‘a good man shall be satisfied from within himself.’ ‘A merry heart makes a cheerful countenance and is a continual feast. Better is a little with the reverence of the Lord than great treasure and trouble therewith. Better is a dinner of herbs where love is than a fatted ox and hatred therewith. Better is a little with righteousness than great revenues without rectitude.’ ‘A merry heart does good like a medicine.’ ‘Better is a handful with composure than a superabundance with sorrow and vexation of spirit.’}
\vs p149 5:3 \textcolor{ubdarkred}{“Much of man’s sorrow is born of the disappointment of his ambitions and the wounding of his pride. Although men owe a duty to themselves to make the best of their lives on earth, having thus sincerely exerted themselves, they should cheerfully accept their lot and exercise ingenuity in making the most of that which has fallen to their hands. All too many of man’s troubles take origin in the fear soil of his own natural heart. ‘The wicked flee when no man pursues.’ ‘The wicked are like the troubled sea, for it cannot rest, but its waters cast up mire and dirt; there is no peace, says God, for the wicked.’}
\vs p149 5:4 \textcolor{ubdarkred}{“Seek not, then, for false peace and transient joy but rather for the assurance of faith and the sureties of divine sonship which yield composure, contentment, and supreme joy in the spirit.”}
\vs p149 5:5 \pc Jesus hardly regarded this world as a “vale of tears.” He rather looked upon it as the birth sphere of the eternal and immortal spirits of Paradise ascension, the “vale of soul making.”
\usection{6.\bibnobreakspace The “Fear of the Lord”}
\vs p149 6:1 It was at Gamala, during the evening conference, that Philip said to Jesus: “Master, why is it that the Scriptures instruct us to ‘fear the Lord,’ while you would have us look to the Father in heaven without fear? How are we to harmonize these teachings?” And Jesus replied to Philip, saying:
\vs p149 6:2 \pc \textcolor{ubdarkred}{“My children, I am not surprised that you ask such questions. In the beginning it was only through fear that man could learn reverence, but I have come to reveal the Father’s love so that you will be attracted to the worship of the Eternal by the drawing of a son’s affectionate recognition and reciprocation of the Father’s profound and perfect love. I would deliver you from the bondage of driving yourselves through slavish fear to the irksome service of a jealous and wrathful King\hyp{}God. I would instruct you in the Father\hyp{}son relationship of God and man so that you may be joyfully led into that sublime and supernal free worship of a loving, just, and merciful Father\hyp{}God.}
\vs p149 6:3 \textcolor{ubdarkred}{“The ‘fear of the Lord’ has had different meanings in the successive ages, coming up from fear, through anguish and dread, to awe and reverence. And now from reverence I would lead you up, through recognition, realization, and appreciation, to \bibemph{love.} When man recognizes only the works of God, he is led to fear the Supreme; but when man begins to understand and experience the personality and character of the living God, he is led increasingly to love such a good and perfect, universal and eternal Father. And it is just this changing of the relation of man to God that constitutes the mission of the Son of Man on earth.}
\vs p149 6:4 \textcolor{ubdarkred}{“Intelligent children do not fear their father in order that they may receive good gifts from his hand; but having already received the abundance of good things bestowed by the dictates of the father’s affection for his sons and daughters, these much loved children are led to love their father in responsive recognition and appreciation of such munificent beneficence. The goodness of God leads to repentance; the beneficence of God leads to service; the mercy of God leads to salvation; while the love of God leads to intelligent and freehearted worship.}
\vs p149 6:5 \textcolor{ubdarkred}{“Your forebears feared God because he was mighty and mysterious. You shall adore him because he is magnificent in love, plenteous in mercy, and glorious in truth. The power of God engenders fear in the heart of man, but the nobility and righteousness of his personality beget reverence, love, and willing worship. A dutiful and affectionate son does not fear or dread even a mighty and noble father. I have come into the world to put love in the place of fear, joy in the place of sorrow, confidence in the place of dread, loving service and appreciative worship in the place of slavish bondage and meaningless ceremonies. But it is still true of those who sit in darkness that ‘the fear of the Lord is the beginning of wisdom.’ But when the light has more fully come, the sons of God are led to praise the Infinite for what he \bibemph{is} rather than to fear him for what he \bibemph{does.}}
\vs p149 6:6 \textcolor{ubdarkred}{“When children are young and unthinking, they must necessarily be admonished to honor their parents; but when they grow older and become somewhat more appreciative of the benefits of the parental ministry and protection, they are led up, through understanding respect and increasing affection, to that level of experience where they actually love their parents for what they are more than for what they have done. The father naturally loves his child, but the child must develop his love for the father from the fear of what the father can do, through awe, dread, dependence, and reverence, to the appreciative and affectionate regard of love.}
\vs p149 6:7 \textcolor{ubdarkred}{“You have been taught that you should ‘fear God and keep his commandments, for that is the whole duty of man.’ But I have come to give you a new and higher commandment. I would teach you to ‘love God and learn to do his will, for that is the highest privilege of the liberated sons of God.’ Your fathers were taught to ‘fear God --- the Almighty King.’ I teach you, ‘Love God --- the all\hyp{}merciful Father.’}
\vs p149 6:8 \textcolor{ubdarkred}{“In the kingdom of heaven, which I have come to declare, there is no high and mighty king; this kingdom is a divine family. The universally recognized and unreservedly worshiped center and head of this far\hyp{}flung brotherhood of intelligent beings is my Father and your Father. I am his Son, and you are also his sons. Therefore it is eternally true that you and I are brethren in the heavenly estate, and all the more so since we have become brethren in the flesh of the earthly life. Cease, then, to fear God as a king or serve him as a master; learn to reverence him as the Creator; honor him as the Father of your spirit youth; love him as a merciful defender; and ultimately worship him as the loving and all\hyp{}wise Father of your more mature spiritual realization and appreciation.}
\vs p149 6:9 \textcolor{ubdarkred}{“Out of your wrong concepts of the Father in heaven grow your false ideas of humility and springs much of your hypocrisy. Man may be a worm of the dust by nature and origin, but when he becomes indwelt by my Father’s spirit, that man becomes divine in his destiny. The bestowal spirit of my Father will surely return to the divine source and universe level of origin, and the human soul of mortal man which shall have become the reborn child of this indwelling spirit shall certainly ascend with the divine spirit to the very presence of the eternal Father.}
\vs p149 6:10 \textcolor{ubdarkred}{“Humility, indeed, becomes mortal man who receives all these gifts from the Father in heaven, albeit there is a divine dignity attached to all such faith candidates for the eternal ascent of the heavenly kingdom. The meaningless and menial practices of an ostentatious and false humility are incompatible with the appreciation of the source of your salvation and the recognition of the destiny of your spirit\hyp{}born souls. Humility before God is altogether appropriate in the depths of your hearts; meekness before men is commendable; but the hypocrisy of self\hyp{}conscious and attention\hyp{}craving humility is childish and unworthy of the enlightened sons of the kingdom.}
\vs p149 6:11 \textcolor{ubdarkred}{“You do well to be meek before God and self\hyp{}controlled before men, but let your meekness be of spiritual origin and not the self\hyp{}deceptive display of a self\hyp{}conscious sense of self\hyp{}righteous superiority. The prophet spoke advisedly when he said, ‘Walk humbly with God,’ for, while the Father in heaven is the Infinite and the Eternal, he also dwells ‘with him who is of a contrite mind and a humble spirit.’ My Father disdains pride, loathes hypocrisy, and abhors iniquity. And it was to emphasize the value of sincerity and perfect trust in the loving support and faithful guidance of the heavenly Father that I have so often referred to the little child as illustrative of the attitude of mind and the response of spirit which are so essential to the entrance of mortal man into the spirit realities of the kingdom of heaven.}
\vs p149 6:12 \textcolor{ubdarkred}{“Well did the Prophet Jeremiah describe many mortals when he said: ‘You are near God in the mouth but far from him in the heart.’ And have you not also read that direful warning of the prophet who said: ‘The priests thereof teach for hire, and the prophets thereof divine for money. At the same time they profess piety and proclaim that the Lord is with them.’ Have you not been well warned against those who ‘speak peace to their neighbors when mischief is in their hearts,’ those who ‘flatter with the lips while the heart is given to double\hyp{}dealing’? Of all the sorrows of a trusting man, none are so terrible as to be ‘wounded in the house of a trusted friend.’”}
\usection{7.\bibnobreakspace Returning to Bethsaida}
\vs p149 7:1 Andrew, in consultation with Simon Peter and with the approval of Jesus, had instructed David at Bethsaida to dispatch messengers to the various preaching groups with instructions to terminate the tour and return to Bethsaida sometime on Thursday, December 30. By supper time on that rainy day all of the apostolic party and the teaching evangelists had arrived at the Zebedee home.\fnc{\ldots{}and return to Bethsaida \bibtextul{some time} on Thursday, December 30. \bibexpl{See note for \bibref[60:3.20]{p060 3:2}. “sometime” is correct.}}
\vs p149 7:2 The group remained together over the Sabbath day, being accommodated in the homes of Bethsaida and near\hyp{}by Capernaum, after which the entire party was granted a two weeks’ recess to go home to their families, visit their friends, or go fishing. The two or three days they were together in Bethsaida were, indeed, exhilarating and inspiring; even the older teachers were edified by the young preachers as they narrated their experiences.
\vs p149 7:3 Of the 117 evangelists who participated in this second preaching tour of Galilee, only about seventy\hyp{}five survived the test of actual experience and were on hand to be assigned to service at the end of the two weeks’ recess. Jesus, with Andrew, Peter, James, and John, remained at the Zebedee home and spent much time in conference regarding the welfare and extension of the kingdom.
