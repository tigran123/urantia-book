\upaper{184}{Before the Sanhedrin Court}
\vs p184 0:1 REPRESENTATIVES of Annas had secretly instructed the captain of the Roman soldiers to bring Jesus immediately to the palace of Annas after he had been arrested. The former high priest desired to maintain his prestige as the chief ecclesiastical authority of the Jews. He also had another purpose in detaining Jesus at his house for several hours, and that was to allow time for legally calling together the court of the Sanhedrin. It was not lawful to convene the Sanhedrin court before the time of the offering of the morning sacrifice in the temple, and this sacrifice was offered about three o’clock in the morning.
\vs p184 0:2 Annas knew that a court of Sanhedrists was in waiting at the palace of his son\hyp{}in\hyp{}law, Caiaphas. Some thirty members of the Sanhedrin had gathered at the home of the high priest by midnight so that they would be ready to sit in judgment on Jesus when he might be brought before them. Only those members were assembled who were strongly and openly opposed to Jesus and his teaching since it required only twenty\hyp{}three to constitute a trial court.
\vs p184 0:3 Jesus spent about three hours at the palace of Annas on Mount Olivet, not far from the garden of Gethsemane, where they arrested him. John Zebedee was free and safe in the palace of Annas not only because of the word of the Roman captain, but also because he and his brother James were well known to the older servants, having many times been guests at the palace as the former high priest was a distant relative of their mother, Salome.
\usection{1.\bibnobreakspace Examination by Annas}
\vs p184 1:1 Annas, enriched by the temple revenues, his son\hyp{}in\hyp{}law the acting high priest, and with his relations to the Roman authorities, was indeed the most powerful single individual in all Jewry. He was a suave and politic planner and plotter. He desired to direct the matter of disposing of Jesus; he feared to trust such an important undertaking wholly to his brusque and aggressive son\hyp{}in\hyp{}law. Annas wanted to make sure that the Master’s trial was kept in the hands of the Sadducees; he feared the possible sympathy of some of the Pharisees, seeing that practically all of those members of the Sanhedrin who had espoused the cause of Jesus were Pharisees.
\vs p184 1:2 Annas had not seen Jesus for several years, not since the time when the Master called at his house and immediately left upon observing his coldness and reserve in receiving him. Annas had thought to presume on this early acquaintance and thereby attempt to persuade Jesus to abandon his claims and leave Palestine. He was reluctant to participate in the murder of a good man and had reasoned that Jesus might choose to leave the country rather than to suffer death. But when Annas stood before the stalwart and determined Galilean, he knew at once that it would be useless to make such proposals. Jesus was even more majestic and well poised than Annas remembered him.
\vs p184 1:3 When Jesus was young, Annas had taken a great interest in him, but now his revenues were threatened by what Jesus had so recently done in driving the money\hyp{}changers and other commercial traders out of the temple. This act had aroused the enmity of the former high priest far more than had Jesus’ teachings.
\vs p184 1:4 Annas entered his spacious audience chamber, seated himself in a large chair, and commanded that Jesus be brought before him. After a few moments spent in silently surveying the Master, he said: “You realize that something must be done about your teaching since you are disturbing the peace and order of our country.” As Annas looked inquiringly at Jesus, the Master looked full into his eyes but made no reply. Again Annas spoke, “What are the names of your disciples, besides Simon Zelotes, the agitator?” Again Jesus looked down upon him, but he did not answer.
\vs p184 1:5 Annas was considerably disturbed by Jesus’ refusal to answer his questions, so much so that he said to him: “Do you have no care as to whether I am friendly to you or not? Do you have no regard for the power I have in determining the issues of your coming trial?” When Jesus heard this, he said: \textcolor{ubdarkred}{“Annas, you know that you could have no power over me unless it were permitted by my Father. Some would destroy the Son of Man because they are ignorant; they know no better, but you, friend, know what you are doing. How can you, therefore, reject the light of God?”}
\vs p184 1:6 The kindly manner in which Jesus spoke to Annas almost bewildered him. But he had already determined in his mind that Jesus must either leave Palestine or die; so he summoned up his courage and asked: “Just what is it you are trying to teach the people? What do you claim to be?” Jesus answered: \textcolor{ubdarkred}{“You know full well that I have spoken openly to the world. I have taught in the synagogues and many times in the temple, where all the Jews and many of the gentiles have heard me. In secret I have spoken nothing; why, then, do you ask me about my teaching? Why do you not summon those who have heard me and inquire of them? Behold, all Jerusalem has heard that which I have spoken even if you have not yourself heard these teachings.”} But before Annas could make reply, the chief steward of the palace, who was standing near, struck Jesus in the face with his hand, saying, “How dare you answer the high priest with such words?” Annas spoke no words of rebuke to his steward, but Jesus addressed him, saying, \textcolor{ubdarkred}{“My friend, if I have spoken evil, bear witness against the evil; but if I have spoken the truth, why, then, should you smite me?”}
\vs p184 1:7 Although Annas regretted that his steward had struck Jesus, he was too proud to take notice of the matter. In his confusion he went into another room, leaving Jesus alone with the household attendants and the temple guards for almost an hour.
\vs p184 1:8 When he returned, going up to the Master’s side, he said, “Do you claim to be the Messiah, the deliverer of Israel?” Said Jesus: \textcolor{ubdarkred}{“Annas, you have known me from the times of my youth. You know that I claim to be nothing except that which my Father has appointed, and that I have been sent to all men, gentile as well as Jew.”} Then said Annas: “I have been told that you have claimed to be the Messiah; is that true?” Jesus looked upon Annas but only replied, \textcolor{ubdarkred}{“So you have said.”}
\vs p184 1:9 About this time messengers arrived from the palace of Caiaphas to inquire what time Jesus would be brought before the court of the Sanhedrin, and since it was nearing the break of day, Annas thought best to send Jesus bound and in the custody of the temple guards to Caiaphas. He himself followed after them shortly.
\usection{2.\bibnobreakspace Peter in the Courtyard}
\vs p184 2:1 As the band of guards and soldiers approached the entrance to the palace of Annas, John Zebedee was marching by the side of the captain of the Roman soldiers. Judas had dropped some distance behind, and Simon Peter followed afar off. After John had entered the palace courtyard with Jesus and the guards, Judas came up to the gate but, seeing Jesus and John, went on over to the home of Caiaphas, where he knew the real trial of the Master would later take place. Soon after Judas had left, Simon Peter arrived, and as he stood before the gate, John saw him just as they were about to take Jesus into the palace. The portress who kept the gate knew John, and when he spoke to her, requesting that she let Peter in, she gladly assented.
\vs p184 2:2 Peter, upon entering the courtyard, went over to the charcoal fire and sought to warm himself, for the night was chilly. He felt very much out of place here among the enemies of Jesus, and indeed he was out of place. The Master had not instructed him to keep near at hand as he had admonished John. Peter belonged with the other apostles, who had been specifically warned not to endanger their lives during these times of the trial and crucifixion of their Master.
\vs p184 2:3 Peter threw away his sword shortly before he came up to the palace gate so that he entered the courtyard of Annas unarmed. His mind was in a whirl of confusion; he could scarcely realize that Jesus had been arrested. He could not grasp the reality of the situation --- that he was here in the courtyard of Annas, warming himself beside the servants of the high priest. He wondered what the other apostles were doing and, in turning over in his mind as to how John came to be admitted to the palace, concluded that it was because he was known to the servants, since he had bidden the gate\hyp{}keeper admit him.
\vs p184 2:4 Shortly after the portress let Peter in, and while he was warming himself by the fire, she went over to him and mischievously said, “Are you not also one of this man’s disciples?” Now Peter should not have been surprised at this recognition, for it was John who had requested that the girl let him pass through the palace gates; but he was in such a tense nervous state that this identification as a disciple threw him off his balance, and with only one thought uppermost in his mind --- the thought of escaping with his life --- he promptly answered the maid’s question by saying, “I am not.”
\vs p184 2:5 Very soon another servant came up to Peter and asked: “Did I not see you in the garden when they arrested this fellow? Are you not also one of his followers?” Peter was now thoroughly alarmed; he saw no way of safely escaping from these accusers; so he vehemently denied all connection with Jesus, saying, “I know not this man, neither am I one of his followers.”
\vs p184 2:6 About this time the portress of the gate drew Peter to one side and said: “I am sure you are a disciple of this Jesus, not only because one of his followers bade me let you in the courtyard, but my sister here has seen you in the temple with this man. Why do you deny this?” When Peter heard the maid accuse him, he denied all knowledge of Jesus with much cursing and swearing, again saying, “I am not this man’s follower; I do not even know him; I never heard of him before.”
\vs p184 2:7 Peter left the fireside for a time while he walked about the courtyard. He would have liked to have escaped, but he feared to attract attention to himself. Getting cold, he returned to the fireside, and one of the men standing near him said: “Surely you are one of this man’s disciples. This Jesus is a Galilean, and your speech betrays you, for you also speak as a Galilean.” And again Peter denied all connection with his Master.
\vs p184 2:8 Peter was so perturbed that he sought to escape contact with his accusers by going away from the fire and remaining by himself on the porch. After more than an hour of this isolation, the gate\hyp{}keeper and her sister chanced to meet him, and both of them again teasingly charged him with being a follower of Jesus. And again he denied the accusation. Just as he had once more denied all connection with Jesus, the cock crowed, and Peter remembered the words of warning spoken to him by his Master earlier that same night. As he stood there, heavy of heart and crushed with the sense of guilt, the palace doors opened, and the guards led Jesus past on the way to Caiaphas. As the Master passed Peter, he saw, by the light of the torches, the look of despair on the face of his former self\hyp{}confident and superficially brave apostle, and he turned and looked upon Peter. Peter never forgot that look as long as he lived. It was such a glance of commingled pity and love as mortal man had never beheld in the face of the Master.
\vs p184 2:9 After Jesus and the guards passed out of the palace gates, Peter followed them, but only for a short distance. He could not go farther. He sat down by the side of the road and wept bitterly. And when he had shed these tears of agony, he turned his steps back toward the camp, hoping to find his brother, Andrew. On arriving at the camp, he found only David Zebedee, who sent a messenger to direct him to where his brother had gone to hide in Jerusalem.
\vs p184 2:10 \pc Peter’s entire experience occurred in the courtyard of the palace of Annas on Mount Olivet. He did not follow Jesus to the palace of the high priest, Caiaphas. That Peter was brought to the realization that he had repeatedly denied his Master by the crowing of a cock indicates that this all occurred outside of Jerusalem since it was against the law to keep poultry within the city proper.
\vs p184 2:11 \pc Until the crowing of the cock brought Peter to his better senses, he had only thought, as he walked up and down the porch to keep warm, how cleverly he had eluded the accusations of the servants, and how he had frustrated their purpose to identify him with Jesus. For the time being, he had only considered that these servants had no moral or legal right thus to question him, and he really congratulated himself over the manner in which he thought he had avoided being identified and possibly subjected to arrest and imprisonment. Not until the cock crowed did it occur to Peter that he had denied his Master. Not until Jesus looked upon him, did he realize that he had failed to live up to his privileges as an ambassador of the kingdom.
\vs p184 2:12 Having taken the first step along the path of compromise and least resistance, there was nothing apparent to Peter but to go on with the course of conduct decided upon. It requires a great and noble character, having started out wrong, to turn about and go right. All too often one’s own mind tends to justify continuance in the path of error when once it is entered upon.
\vs p184 2:13 Peter never fully believed that he could be forgiven until he met his Master after the resurrection and saw that he was received just as before the experiences of this tragic night of the denials.
\usection{3.\bibnobreakspace Before the Court of Sanhedrists}
\vs p184 3:1 It was about half past three o’clock this Friday morning when the chief priest, Caiaphas, called the Sanhedrist court of inquiry to order and asked that Jesus be brought before them for his formal trial. On three previous occasions the Sanhedrin, by a large majority vote, had decreed the death of Jesus, had decided that he was worthy of death on informal charges of lawbreaking, blasphemy, and flouting the traditions of the fathers of Israel.\fnc{\ldots{}on informal charges of \bibtextul{law-breaking,} blasphemy\ldots{} \bibexpl{Of the five occurrences of lawbreak[er] [-ing] in the text, three are closed and two are hyphenated. There is no differential in meaning indicated by the two forms, so database standardization could be appropriate.}}
\vs p184 3:2 This was not a regularly called meeting of the Sanhedrin and was not held in the usual place, the chamber of hewn stone in the temple. This was a special trial court of some thirty Sanhedrists and was convened in the palace of the high priest. John Zebedee was present with Jesus throughout this so\hyp{}called trial.
\vs p184 3:3 How these chief priests, scribes, Sadducees, and some of the Pharisees flattered themselves that Jesus, the disturber of their position and the challenger of their authority, was now securely in their hands! And they were resolved that he should never live to escape their vengeful clutches.
\vs p184 3:4 Ordinarily, the Jews, when trying a man on a capital charge, proceeded with great caution and provided every safeguard of fairness in the selection of witnesses and the entire conduct of the trial. But on this occasion, Caiaphas was more of a prosecutor than an unbiased judge.
\vs p184 3:5 \pc Jesus appeared before this court clothed in his usual garments and with his hands bound together behind his back. The entire court was startled and somewhat confused by his majestic appearance. Never had they gazed upon such a prisoner nor witnessed such composure in a man on trial for his life.
\vs p184 3:6 \pc The Jewish law required that at least two witnesses must agree upon any point before a charge could be laid against the prisoner. Judas could not be used as a witness against Jesus because the Jewish law specifically forbade the testimony of a traitor. More than a score of false witnesses were on hand to testify against Jesus, but their testimony was so contradictory and so evidently trumped up that the Sanhedrists themselves were very much ashamed of the performance. Jesus stood there, looking down benignly upon these perjurers, and his very countenance disconcerted the lying witnesses. Throughout all this false testimony the Master never said a word; he made no reply to their many false accusations.
\vs p184 3:7 The first time any two of their witnesses approached even the semblance of an agreement was when two men testified that they had heard Jesus say in the course of one of his temple discourses that he would “destroy this temple made with hands and in three days make another temple without hands.” That was not exactly what Jesus said, regardless of the fact that he pointed to his own body when he made the remark referred to.
\vs p184 3:8 Although the high priest shouted at Jesus, “Do you not answer any of these charges?” Jesus opened not his mouth. He stood there in silence while all of these false witnesses gave their testimony. Hatred, fanaticism, and unscrupulous exaggeration so characterized the words of these perjurers that their testimony fell in its own entanglements. The very best refutation of their false accusations was the Master’s calm and majestic silence.
\vs p184 3:9 Shortly after the beginning of the testimony of the false witnesses, Annas arrived and took his seat beside Caiaphas. Annas now arose and argued that this threat of Jesus to destroy the temple was sufficient to warrant three charges against him:
\vs p184 3:10 \ublistelem{1.}\bibnobreakspace That he was a dangerous traducer of the people. That he taught them impossible things and otherwise deceived them.
\vs p184 3:11 \ublistelem{2.}\bibnobreakspace That he was a fanatical revolutionist in that he advocated laying violent hands on the sacred temple, else how could he destroy it?
\vs p184 3:12 \ublistelem{3.}\bibnobreakspace That he taught magic inasmuch as he promised to build a new temple, and that without hands.
\vs p184 3:13 \pc Already had the full Sanhedrin agreed that Jesus was guilty of death\hyp{}deserving transgressions of the Jewish laws, but they were now more concerned with developing charges regarding his conduct and teachings which would justify Pilate in pronouncing the death sentence upon their prisoner. They knew that they must secure the consent of the Roman governor before Jesus could legally be put to death. And Annas was minded to proceed along the line of making it appear that Jesus was a dangerous teacher to be abroad among the people.
\vs p184 3:14 But Caiaphas could not longer endure the sight of the Master standing there in perfect composure and unbroken silence. He thought he knew at least one way in which the prisoner might be induced to speak. Accordingly, he rushed over to the side of Jesus and, shaking his accusing finger in the Master’s face, said: “I adjure you, in the name of the living God, that you tell us whether you are the Deliverer, the Son of God.” Jesus answered Caiaphas: \textcolor{ubdarkred}{“I am. Soon I go to the Father, and presently shall the Son of Man be clothed with power and once more reign over the hosts of heaven.”}
\vs p184 3:15 When the high priest heard Jesus utter these words, he was exceedingly angry, and rending his outer garments, he exclaimed: “What further need have we of witnesses? Behold, now have you all heard this man’s blasphemy. What do you now think should be done with this lawbreaker and blasphemer?” And they all answered in unison, “He is worthy of death; let him be crucified.”\fnc{\ldots{}be done with this \bibtextul{law-breaker} and blasphemer\ldots{} \bibexpl{See note for \bibref[184:3.1]{p0184 3:1}.}}
\vs p184 3:16 Jesus manifested no interest in any question asked him when before Annas or the Sanhedrists except the one question relative to his bestowal mission. When asked if he were the Son of God, he instantly and unequivocally answered in the affirmative.
\vs p184 3:17 Annas desired that the trial proceed further, and that charges of a definite nature regarding Jesus’ relation to the Roman law and Roman institutions be formulated for subsequent presentation to Pilate. The councilors were anxious to carry these matters to a speedy termination, not only because it was the preparation day for the Passover and no secular work should be done after noon, but also because they feared Pilate might any time return to the Roman capital of Judea, Caesarea, since he was in Jerusalem only for the Passover celebration.
\vs p184 3:18 But Annas did not succeed in keeping control of the court. After Jesus had so unexpectedly answered Caiaphas, the high priest stepped forward and smote him in the face with his hand. Annas was truly shocked as the other members of the court, in passing out of the room, spit in Jesus’ face, and many of them mockingly slapped him with the palms of their hands. And thus in disorder and with such unheard\hyp{}of confusion this first session of the Sanhedrist trial of Jesus ended at half past four o’clock.
\vs p184 3:19 \pc Thirty prejudiced and tradition\hyp{}blinded false judges, with their false witnesses, are presuming to sit in judgment on the righteous Creator of a universe. And these impassioned accusers are exasperated by the majestic silence and superb bearing of this God\hyp{}man. His silence is terrible to endure; his speech is fearlessly defiant. He is unmoved by their threats and undaunted by their assaults. Man sits in judgment on God, but even then he loves them and would save them if he could.
\usection{4.\bibnobreakspace The Hour of Humiliation}
\vs p184 4:1 The Jewish law required that, in the matter of passing the death sentence, there should be two sessions of the court. This second session was to be held on the day following the first, and the intervening time was to be spent in fasting and mourning by the members of the court. But these men could not await the next day for the confirmation of their decision that Jesus must die. They waited only one hour. In the meantime Jesus was left in the audience chamber in the custody of the temple guards, who, with the servants of the high priest, amused themselves by heaping every sort of indignity upon the Son of Man. They mocked him, spit upon him, and cruelly buffeted him. They would strike him in the face with a rod and then say, “Prophesy to us, you the Deliverer, who it was that struck you.” And thus they went on for one full hour, reviling and mistreating this unresisting man of Galilee.
\vs p184 4:2 During this tragic hour of suffering and mock trials before the ignorant and unfeeling guards and servants, John Zebedee waited in lonely terror in an adjoining room. When these abuses first started, Jesus indicated to John, by a nod of his head, that he should retire. The Master well knew that, if he permitted his apostle to remain in the room to witness these indignities, John’s resentment would be so aroused as to produce such an outbreak of protesting indignation as would probably result in his death.
\vs p184 4:3 Throughout this awful hour Jesus uttered no word. To this gentle and sensitive soul of humankind, joined in personality relationship with the God of all this universe, there was no more bitter portion of his cup of humiliation than this terrible hour at the mercy of these ignorant and cruel guards and servants, who had been stimulated to abuse him by the example of the members of this so\hyp{}called Sanhedrist court.
\vs p184 4:4 \pc The human heart cannot possibly conceive of the shudder of indignation that swept out over a vast universe as the celestial intelligences witnessed this sight of their beloved Sovereign submitting himself to the will of his ignorant and misguided creatures on the sin\hyp{}darkened sphere of unfortunate Urantia.
\vs p184 4:5 What is this trait of the animal in man which leads him to want to insult and physically assault that which he cannot spiritually attain or intellectually achieve? In the half\hyp{}civilized man there still lurks an evil brutality which seeks to vent itself upon those who are superior in wisdom and spiritual attainment. Witness the evil coarseness and the brutal ferocity of these supposedly civilized men as they derived a certain form of animal pleasure from this physical attack upon the unresisting Son of Man. As these insults, taunts, and blows fell upon Jesus, he was undefending but not defenseless. Jesus was not vanquished, merely uncontending in the material sense.
\vs p184 4:6 These are the moments of the Master’s greatest victories in all his long and eventful career as maker, upholder, and savior of a vast and far\hyp{}flung universe. Having lived to the full a life of revealing God to man, Jesus is now engaged in making a new and unprecedented revelation of man to God. Jesus is now revealing to the worlds the final triumph over all fears of creature personality isolation. The Son of Man has finally achieved the realization of identity as the Son of God. Jesus does not hesitate to assert that he and the Father are one; and on the basis of the fact and truth of that supreme and supernal experience, he admonishes every kingdom believer to become one with him even as he and his Father are one. The living experience in the religion of Jesus thus becomes the sure and certain technique whereby the spiritually isolated and cosmically lonely mortals of earth are enabled to escape personality isolation, with all its consequences of fear and associated feelings of helplessness. In the fraternal realities of the kingdom of heaven the faith sons of God find final deliverance from the isolation of the self, both personal and planetary. The God\hyp{}knowing believer increasingly experiences the ecstasy and grandeur of spiritual socialization on a universe scale --- citizenship on high in association with the eternal realization of the divine destiny of perfection attainment.
\usection{5.\bibnobreakspace The Second Meeting of the Court}
\vs p184 5:1 At five\hyp{}thirty o’clock the court reassembled, and Jesus was led into the adjoining room, where John was waiting. Here the Roman soldier and the temple guards watched over Jesus while the court began the formulation of the charges which were to be presented to Pilate. Annas made it clear to his associates that the charge of blasphemy would carry no weight with Pilate. Judas was present during this second meeting of the court, but he gave no testimony.
\vs p184 5:2 This session of the court lasted only a half hour, and when they adjourned to go before Pilate, they had drawn up the indictment of Jesus, as being worthy of death, under three heads:
\vs p184 5:3 \ublistelem{1.}\bibnobreakspace That he was a perverter of the Jewish nation; he deceived the people and incited them to rebellion.
\vs p184 5:4 \ublistelem{2.}\bibnobreakspace That he taught the people to refuse to pay tribute to Caesar.
\vs p184 5:5 \ublistelem{3.}\bibnobreakspace That, by claiming to be a king and the founder of a new sort of kingdom, he incited treason against the emperor.
\vs p184 5:6 \pc This entire procedure was irregular and wholly contrary to the Jewish laws. No two witnesses had agreed on any matter except those who testified regarding Jesus’ statement about destroying the temple and raising it again in three days. And even concerning that point, no witnesses spoke for the defense, and neither was Jesus asked to explain his intended meaning.
\vs p184 5:7 The only point the court could have consistently judged him on was that of blasphemy, and that would have rested entirely on his own testimony. Even concerning blasphemy, they failed to cast a formal ballot for the death sentence.
\vs p184 5:8 And now they presumed to formulate three charges, with which to go before Pilate, on which no witnesses had been heard, and which were agreed upon while the accused prisoner was absent. When this was done, three of the Pharisees took their leave; they wanted to see Jesus destroyed, but they would not formulate charges against him without witnesses and in his absence.
\vs p184 5:9 Jesus did not again appear before the Sanhedrist court. They did not want again to look upon his face as they sat in judgment upon his innocent life. Jesus did not know (as a man) of their formal charges until he heard them recited by Pilate.
\vs p184 5:10 \pc While Jesus was in the room with John and the guards, and while the court was in its second session, some of the women about the high priest’s palace, together with their friends, came to look upon the strange prisoner, and one of them asked him, “Are you the Messiah, the Son of God?” And Jesus answered: \textcolor{ubdarkred}{“If I tell you, you will not believe me; and if I ask you, you will not answer.”}
\vs p184 5:11 At six o’clock that morning Jesus was led forth from the home of Caiaphas to appear before Pilate for confirmation of the sentence of death which this Sanhedrist court had so unjustly and irregularly decreed.
