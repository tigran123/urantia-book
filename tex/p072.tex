\upaper{72}{Government on a Neighboring Planet}
\vs p072 0:1 BY PERMISSION of Lanaforge and with the approval of the Most Highs of Edentia, I am authorized to narrate something of the social, moral, and political life of the most advanced human race living on a not far\hyp{}distant planet belonging to the Satania system.
\vs p072 0:2 Of all the Satania worlds which became isolated because of participation in the Lucifer rebellion, this planet has experienced a history most like that of Urantia. The similarity of the two spheres undoubtedly explains why permission to make this extraordinary presentation was granted, for it is most unusual for the system rulers to consent to the narration on one planet of the affairs of another.
\vs p072 0:3 This planet, like Urantia, was led astray by the disloyalty of its Planetary Prince in connection with the Lucifer rebellion. It received a Material Son shortly after Adam came to Urantia, and this Son also defaulted, leaving the sphere isolated, since a Magisterial Son has never been bestowed upon its mortal races.
\usection{1.\bibnobreakspace The Continental Nation}
\vs p072 1:1 Notwithstanding all these planetary handicaps a very superior civilization is evolving on an isolated continent about the size of Australia. This nation numbers about 140 million. Its people are a mixed race, predominantly blue and yellow, having a slightly greater proportion of violet than the so\hyp{}called white race of Urantia. These different races are not yet fully blended, but they fraternize and socialize very acceptably. The average length of life on this continent is now ninety years, fifteen per cent higher than that of any other people on the planet.
\vs p072 1:2 The industrial mechanism of this nation enjoys a certain great advantage derived from the unique topography of the continent. The high mountains, on which heavy rains fall eight months in the year, are situated at the very center of the country. This natural arrangement favors the utilization of water power and greatly facilitates the irrigation of the more arid western quarter of the continent.
\vs p072 1:3 These people are self\hyp{}sustaining, that is, they can live indefinitely without importing anything from the surrounding nations. Their natural resources are replete, and by scientific techniques they have learned how to compensate for their deficiencies in the essentials of life. They enjoy a brisk domestic commerce but have little foreign trade owing to the universal hostility of their less progressive neighbors.
\vs p072 1:4 \pc This continental nation, in general, followed the evolutionary trend of the planet: The development from the tribal stage to the appearance of strong rulers and kings occupied thousands of years. The unconditional monarchs were succeeded by many different orders of government --- abortive republics, communal states, and dictators came and went in endless profusion. This growth continued until about five hundred years ago when, during a politically fermenting period, one of the nation’s powerful dictator\hyp{}triumvirs had a change of heart. He volunteered to abdicate upon condition that one of the other rulers, the baser of the remaining two, also vacate his dictatorship. Thus was the sovereignty of the continent placed in the hands of one ruler. The unified state progressed under strong monarchial rule for over one hundred years, during which there evolved a masterful charter of liberty.
\vs p072 1:5 The subsequent transition from monarchy to a representative form of government was gradual, the kings remaining as mere social or sentimental figureheads, finally disappearing when the male line of descent ran out. The present republic has now been in existence just two hundred years, during which time there has been a continuous progression toward the governmental techniques about to be narrated, the last developments in industrial and political realms having been made within the past decade.
\usection{2.\bibnobreakspace Political Organization}
\vs p072 2:1 This continental nation now has a representative government with a centrally located national capital. The central government consists of a strong federation of one hundred comparatively free states. These states elect their governors and legislators for ten years, and none are eligible for re\hyp{}election. State judges are appointed for life by the governors and confirmed by their legislatures, which consist of one representative for each one hundred thousand citizens.
\vs p072 2:2 There are five different types of metropolitan government, depending on the size of the city, but no city is permitted to have more than one million inhabitants. On the whole, these municipal governing schemes are very simple, direct, and economical. The few offices of city administration are keenly sought by the highest types of citizens.
\vs p072 2:3 The federal government embraces three co\hyp{}ordinate divisions: executive, legislative, and judicial. The federal chief executive is elected every six years by universal territorial suffrage. He is not eligible for re\hyp{}election except upon the petition of at least seventy\hyp{}five state legislatures concurred in by the respective state governors, and then but for one term. He is advised by a supercabinet composed of all living ex\hyp{}chief executives.
\vs p072 2:4 \pc The legislative division embraces three houses:
\vs p072 2:5 \ublistelem{1.}\bibnobreakspace The \bibemph{upper house} is elected by industrial, professional, agricultural, and other groups of workers, balloting in accordance with economic function.
\vs p072 2:6 \pc \ublistelem{2.}\bibnobreakspace The \bibemph{lower house} is elected by certain organizations of society embracing the social, political, and philosophic groups not included in industry or the professions. All citizens in good standing participate in the election of both classes of representatives, but they are differently grouped, depending on whether the election pertains to the upper or lower house.
\vs p072 2:7 \pc \ublistelem{3.}\bibnobreakspace The \bibemph{third house ---} the elder statesmen --- embraces the veterans of civic service and includes many distinguished persons nominated by the chief executive, by the regional (subfederal) executives, by the chief of the supreme tribunal, and by the presiding officers of either of the other legislative houses. This group is limited to one hundred, and its members are elected by the majority action of the elder statesmen themselves. Membership is for life, and when vacancies occur, the person receiving the largest ballot among the list of nominees is thereby duly elected. The scope of this body is purely advisory, but it is a mighty regulator of public opinion and exerts a powerful influence upon all branches of the government.
\vs p072 2:8 \pc Very much of the federal administrative work is carried on by the ten regional (subfederal) authorities, each consisting of the association of ten states. These regional divisions are wholly executive and administrative, having neither legislative nor judicial functions. The ten regional executives are the personal appointees of the federal chief executive, and their term of office is concurrent with his --- six years. The federal supreme tribunal approves the appointment of these ten regional executives, and while they may not be reappointed, the retiring executive automatically becomes the associate and adviser of his successor. Otherwise, these regional chiefs choose their own cabinets of administrative officials.
\vs p072 2:9 \pc This nation is adjudicated by two major court systems --- the law courts and the socioeconomic courts. The law courts function on the following three levels:
\vs p072 2:10 \ublistelem{1.}\bibnobreakspace \bibemph{Minor courts} of municipal and local jurisdiction, whose decisions may be appealed to the high state tribunals.
\vs p072 2:11 \pc \ublistelem{2.}\bibnobreakspace \bibemph{State supreme courts,} whose decisions are final in all matters not involving the federal government or jeopardy of citizenship rights and liberties. The regional executives are empowered to bring any case at once to the bar of the federal supreme court.
\vs p072 2:12 \pc \ublistelem{3.}\bibnobreakspace \bibemph{Federal supreme court ---} the high tribunal for the adjudication of national contentions and the appellate cases coming up from the state courts. This supreme tribunal consists of twelve men over forty and under seventy\hyp{}five years of age who have served two or more years on some state tribunal, and who have been appointed to this high position by the chief executive with the majority approval of the supercabinet and the third house of the legislative assembly. All decisions of this supreme judicial body are by at least a two\hyp{}thirds vote.
\vs p072 2:13 \pc The socioeconomic courts function in the following three divisions:
\vs p072 2:14 \ublistelem{1.}\bibnobreakspace \bibemph{Parental courts,} associated with the legislative and executive divisions of the home and social system.
\vs p072 2:15 \pc \ublistelem{2.}\bibnobreakspace \bibemph{Educational courts ---} the juridical bodies connected with the state and regional school systems and associated with the executive and legislative branches of the educational administrative mechanism.
\vs p072 2:16 \pc \ublistelem{3.}\bibnobreakspace \bibemph{Industrial courts ---} the jurisdictional tribunals vested with full authority for the settlement of all economic misunderstandings.
\vs p072 2:17 \pc The federal supreme court does not pass upon socioeconomic cases except upon the three\hyp{}quarters vote of the third legislative branch of the national government, the house of elder statesmen. Otherwise, all decisions of the parental, educational, and industrial high courts are final.
\usection{3.\bibnobreakspace The Home Life}
\vs p072 3:1 On this continent it is against the law for two families to live under the same roof. And since group dwellings have been outlawed, most of the tenement type of buildings have been demolished. But the unmarried still live in clubs, hotels, and other group dwellings. The smallest homesite permitted must provide fifty thousand square feet of land. All land and other property used for home purposes are free from taxation up to ten times the minimum homesite allotment.
\vs p072 3:2 The home life of this people has greatly improved during the last century. Attendance of parents, both fathers and mothers, at the parental schools of child culture is compulsory. Even the agriculturists who reside in small country settlements carry on this work by correspondence, going to the near\hyp{}by centers for oral instruction once in ten days --- every two weeks, for they maintain a five\hyp{}day week.
\vs p072 3:3 The average number of children in each family is five, and they are under the full control of their parents or, in case of the demise of one or both, under that of the guardians designated by the parental courts. It is considered a great honor for any family to be awarded the guardianship of a full orphan. Competitive examinations are held among parents, and the orphan is awarded to the home of those displaying the best parental qualifications.
\vs p072 3:4 \pc These people regard the home as the basic institution of their civilization. It is expected that the most valuable part of a child’s education and character training will be secured from his parents and at home, and fathers devote almost as much attention to child culture as do mothers.
\vs p072 3:5 All sex instruction is administered in the home by parents or by legal guardians. Moral instruction is offered by teachers during the rest periods in the school shops, but not so with religious training, which is deemed to be the exclusive privilege of parents, religion being looked upon as an integral part of home life. Purely religious instruction is given publicly only in the temples of philosophy, no such exclusively religious institutions as the Urantia churches having developed among this people. In their philosophy, religion is the striving to know God and to manifest love for one’s fellows through service for them, but this is not typical of the religious status of the other nations on this planet. Religion is so entirely a family matter among these people that there are no public places devoted exclusively to religious assembly. Politically, church and state, as Urantians are wont to say, are entirely separate, but there is a strange overlapping of religion and philosophy.
\vs p072 3:6 Until twenty years ago the spiritual teachers (comparable to Urantia pastors), who visit each family periodically to examine the children to ascertain if they have been properly instructed by their parents, were under governmental supervision. These spiritual advisers and examiners are now under the direction of the newly created Foundation of Spiritual Progress, an institution supported by voluntary contributions. Possibly this institution may not further evolve until after the arrival of a Paradise Magisterial Son.
\vs p072 3:7 \pc Children remain legally subject to their parents until they are fifteen, when the first initiation into civic responsibility is held. Thereafter, every five years for five successive periods similar public exercises are held for such age groups at which their obligations to parents are lessened, while new civic and social responsibilities to the state are assumed. Suffrage is conferred at twenty, the right to marry without parental consent is not bestowed until twenty\hyp{}five, and children must leave home on reaching the age of thirty.
\vs p072 3:8 Marriage and divorce laws are uniform throughout the nation. Marriage before twenty --- the age of civil enfranchisement --- is not permitted. Permission to marry is only granted after one year’s notice of intention, and after both bride and groom present certificates showing that they have been duly instructed in the parental schools regarding the responsibilities of married life.
\vs p072 3:9 Divorce regulations are somewhat lax, but decrees of separation, issued by the parental courts, may not be had until one year after application therefor has been recorded, and the year on this planet is considerably longer than on Urantia. Notwithstanding their easy divorce laws, the present rate of divorces is only one tenth that of the civilized races of Urantia.
\usection{4.\bibnobreakspace The Educational System}
\vs p072 4:1 The educational system of this nation is compulsory and coeducational in the precollege schools that the student attends from the ages of five to eighteen. These schools are vastly different from those of Urantia. There are no classrooms, only one study is pursued at a time, and after the first three years all pupils become assistant teachers, instructing those below them. Books are used only to secure information that will assist in solving the problems arising in the school shops and on the school farms. Much of the furniture used on the continent and the many mechanical contrivances --- this is a great age of invention and mechanization --- are produced in these shops. Adjacent to each shop is a working library where the student may consult the necessary reference books. Agriculture and horticulture are also taught throughout the entire educational period on the extensive farms adjoining every local school.
\vs p072 4:2 \pc The feeble\hyp{}minded are trained only in agriculture and animal husbandry, and are committed for life to special custodial colonies where they are segregated by sex to prevent parenthood, which is denied all subnormals. These restrictive measures have been in operation for seventy\hyp{}five years; the commitment decrees are handed down by the parental courts.
\vs p072 4:3 \pc Everyone takes one month’s vacation each year. The precollege schools are conducted for nine months out of the year of ten, the vacation being spent with parents or friends in travel. This travel is a part of the adult\hyp{}education program and is continued throughout a lifetime, the funds for meeting such expenses being accumulated by the same methods as those employed in old\hyp{}age insurance.
\vs p072 4:4 One quarter of the school time is devoted to play --- competitive athletics --- the pupils progressing in these contests from the local, through the state and regional, and on to the national trials of skill and prowess. Likewise, the oratorical and musical contests, as well as those in science and philosophy, occupy the attention of students from the lower social divisions on up to the contests for national honors.
\vs p072 4:5 The school government is a replica of the national government with its three correlated branches, the teaching staff functioning as the third or advisory legislative division. The chief object of education on this continent is to make every pupil a self\hyp{}supporting citizen.
\vs p072 4:6 Every child graduating from the precollege school system at eighteen is a skilled artisan. Then begins the study of books and the pursuit of special knowledge, either in the adult schools or in the colleges. When a brilliant student completes his work ahead of schedule, he is granted an award of time and means wherewith he may execute some pet project of his own devising. The entire educational system is designed to adequately train the individual.
\usection{5.\bibnobreakspace Industrial Organization}
\vs p072 5:1 The industrial situation among this people is far from their ideals; capital and labor still have their troubles, but both are becoming adjusted to the plan of sincere co\hyp{}operation. On this unique continent the workers are increasingly becoming shareholders in all industrial concerns; every intelligent laborer is slowly becoming a small capitalist.
\vs p072 5:2 Social antagonisms are lessening, and good will is growing apace. No grave economic problems have arisen out of the abolition of slavery (over one hundred years ago) since this adjustment was effected gradually by the liberation of two per cent each year. Those slaves who satisfactorily passed mental, moral, and physical tests were granted citizenship; many of these superior slaves were war captives or children of such captives. Some fifty years ago they deported the last of their inferior slaves, and still more recently they are addressing themselves to the task of reducing the numbers of their degenerate and vicious classes.
\vs p072 5:3 \pc These people have recently developed new techniques for the adjustment of industrial misunderstandings and for the correction of economic abuses which are marked improvements over their older methods of settling such problems. Violence has been outlawed as a procedure in adjusting either personal or industrial differences. Wages, profits, and other economic problems are not rigidly regulated, but they are in general controlled by the industrial legislatures, while all disputes arising out of industry are passed upon by the industrial courts.
\vs p072 5:4 The industrial courts are only thirty years old but are functioning very satisfactorily. The most recent development provides that hereafter the industrial courts shall recognize legal compensation as falling in three divisions:
\vs p072 5:5 \ublistelem{1.}\bibnobreakspace Legal rates of interest on invested capital.
\vs p072 5:6 \ublistelem{2.}\bibnobreakspace Reasonable salary for skill employed in industrial operations.
\vs p072 5:7 \ublistelem{3.}\bibnobreakspace Fair and equitable wages for labor.
\vs p072 5:8 \pc These shall first be met in accordance with contract, or in the face of decreased earnings they shall share proportionally in transient reduction. And thereafter all earnings in excess of these fixed charges shall be regarded as dividends and shall be prorated to all three divisions: capital, skill, and labor.
\vs p072 5:9 \pc Every ten years the regional executives adjust and decree the lawful hours of daily gainful toil. Industry now operates on a five\hyp{}day week, working four and playing one. These people labor six hours each working day and, like students, nine months in the year of ten. Vacation is usually spent in travel, and new methods of transportation having been so recently developed, the whole nation is travel bent. The climate favors travel about eight months in the year, and they are making the most of their opportunities.
\vs p072 5:10 \pc Two hundred years ago the profit motive was wholly dominant in industry, but today it is being rapidly displaced by other and higher driving forces. Competition is keen on this continent, but much of it has been transferred from industry to play, skill, scientific achievement, and intellectual attainment. It is most active in social service and governmental loyalty. Among this people public service is rapidly becoming the chief goal of ambition. The richest man on the continent works six hours a day in the office of his machine shop and then hastens over to the local branch of the school of statesmanship, where he seeks to qualify for public service.
\vs p072 5:11 Labor is becoming more honorable on this continent, and all able\hyp{}bodied citizens over eighteen work either at home and on farms, at some recognized industry, on the public works where the temporarily unemployed are absorbed, or else in the corps of compulsory laborers in the mines.
\vs p072 5:12 These people are also beginning to foster a new form of social disgust --- disgust for both idleness and unearned wealth. Slowly but certainly they are conquering their machines. Once they, too, struggled for political liberty and subsequently for economic freedom. Now are they entering upon the enjoyment of both while in addition they are beginning to appreciate their well\hyp{}earned leisure, which can be devoted to increased self\hyp{}realization.
\usection{6.\bibnobreakspace Old\hyp{}Age Insurance}
\vs p072 6:1 This nation is making a determined effort to replace the self\hyp{}respect\hyp{}destroying type of charity by dignified government\hyp{}insurance guarantees of security in old age. This nation provides every child an education and every man a job; therefore can it successfully carry out such an insurance scheme for the protection of the infirm and aged.
\vs p072 6:2 Among this people all persons must retire from gainful pursuit at sixty\hyp{}five unless they secure a permit from the state labor commissioner which will entitle them to remain at work until the age of seventy. This age limit does not apply to government servants or philosophers. The physically disabled or permanently crippled can be placed on the retired list at any age by court order countersigned by the pension commissioner of the regional government.
\vs p072 6:3 \pc The funds for old\hyp{}age pensions are derived from four sources:
\vs p072 6:4 \ublistelem{1.}\bibnobreakspace One day’s earnings each month are requisitioned by the federal government for this purpose, and in this country everybody works.
\vs p072 6:5 \ublistelem{2.}\bibnobreakspace Bequests --- many wealthy citizens leave funds for this purpose.
\vs p072 6:6 \ublistelem{3.}\bibnobreakspace The earnings of compulsory labor in the state mines. After the conscript workers support themselves and set aside their own retirement contributions, all excess profits on their labor are turned over to this pension fund.
\vs p072 6:7 \ublistelem{4.}\bibnobreakspace The income from natural resources. All natural wealth on the continent is held as a social trust by the federal government, and the income therefrom is utilized for social purposes, such as disease prevention, education of geniuses, and expenses of especially promising individuals in the statesmanship schools. One half of the income from natural resources goes to the old\hyp{}age pension fund.
\vs p072 6:8 \pc Although state and regional actuarial foundations supply many forms of protective insurance, old\hyp{}age pensions are solely administered by the federal government through the ten regional departments.
\vs p072 6:9 These government funds have long been honestly administered. Next to treason and murder, the heaviest penalties meted out by the courts are attached to betrayal of public trust. Social and political disloyalty are now looked upon as being the most heinous of all crimes.
\usection{7.\bibnobreakspace Taxation}
\vs p072 7:1 The federal government is paternalistic only in the administration of old\hyp{}age pensions and in the fostering of genius and creative originality; the state governments are slightly more concerned with the individual citizen, while the local governments are much more paternalistic or socialistic. The city (or some subdivision thereof) concerns itself with such matters as health, sanitation, building regulations, beautification, water supply, lighting, heating, recreation, music, and communication.
\vs p072 7:2 In all industry first attention is paid to health; certain phases of physical well\hyp{}being are regarded as industrial and community prerogatives, but individual and family health problems are matters of personal concern only. In medicine, as in all other purely personal matters, it is increasingly the plan of government to refrain from interfering.
\vs p072 7:3 \pc Cities have no taxing power, neither can they go in debt. They receive per capita allowances from the state treasury and must supplement such revenue from the earnings of their socialistic enterprises and by licensing various commercial activities.
\vs p072 7:4 The rapid\hyp{}transit facilities, which make it practical greatly to extend the city boundaries, are under municipal control. The city fire departments are supported by the fire\hyp{}prevention and insurance foundations, and all buildings, in city or country, are fireproof --- have been for over seventy\hyp{}five years.
\vs p072 7:5 There are no municipally appointed peace officers; the police forces are maintained by the state governments. This department is recruited almost entirely from the unmarried men between twenty\hyp{}five and fifty. Most of the states assess a rather heavy bachelor tax, which is remitted to all men joining the state police. In the average state the police force is now only one tenth as large as it was fifty years ago.
\vs p072 7:6 \pc There is little or no uniformity among the taxation schemes of the one hundred comparatively free and sovereign states as economic and other conditions vary greatly in different sections of the continent. Every state has ten basic constitutional provisions which cannot be modified except by consent of the federal supreme court, and one of these articles prevents levying a tax of more than one per cent on the value of any property in any one year, homesites, whether in city or country, being exempted.
\vs p072 7:7 The federal government cannot go in debt, and a three\hyp{}fourths referendum is required before any state can borrow except for purposes of war. Since the federal government cannot incur debt, in the event of war the National Council of Defense is empowered to assess the states for money, as well as for men and materials, as it may be required. But no debt may run for more than twenty\hyp{}five years.
\vs p072 7:8 \pc Income to support the federal government is derived from the following five sources:
\vs p072 7:9 \ublistelem{1.}\bibnobreakspace \bibemph{Import duties.} All imports are subject to a tariff designed to protect the standard of living on this continent, which is far above that of any other nation on the planet. These tariffs are set by the highest industrial court after both houses of the industrial congress have ratified the recommendations of the chief executive of economic affairs, who is the joint appointee of these two legislative bodies. The upper industrial house is elected by labor, the lower by capital.
\vs p072 7:10 \pc \ublistelem{2.}\bibnobreakspace \bibemph{Royalties.} The federal government encourages invention and original creations in the ten regional laboratories, assisting all types of geniuses --- artists, authors, and scientists --- and protecting their patents. In return the government takes one half the profits realized from all such inventions and creations, whether pertaining to machines, books, artistry, plants, or animals.
\vs p072 7:11 \pc \ublistelem{3.}\bibnobreakspace \bibemph{Inheritance tax.} The federal government levies a graduated inheritance tax ranging from one to fifty per cent, depending on the size of an estate as well as on other conditions.
\vs p072 7:12 \pc \ublistelem{4.}\bibnobreakspace \bibemph{Military equipment.} The government earns a considerable sum from the leasing of military and naval equipment for commercial and recreational usages.
\vs p072 7:13 \pc \ublistelem{5.}\bibnobreakspace \bibemph{Natural resources.} The income from natural resources, when not fully required for the specific purposes designated in the charter of federal statehood, is turned into the national treasury.
\vs p072 7:14 \pc Federal appropriations, except war funds assessed by the National Council of Defense, are originated in the upper legislative house, concurred in by the lower house, approved by the chief executive, and finally validated by the federal budget commission of one hundred. The members of this commission are nominated by the state governors and elected by the state legislatures to serve for twenty\hyp{}four years, one quarter being elected every six years. Every six years this body, by a three\hyp{}fourths ballot, chooses one of its number as chief, and he thereby becomes director\hyp{}controller of the federal treasury.
\usection{8.\bibnobreakspace The Special Colleges}
\vs p072 8:1 In addition to the basic compulsory education program extending from the ages of five to eighteen, special schools are maintained as follows:
\vs p072 8:2 \ublistelem{1.}\bibnobreakspace \bibemph{Statesmanship schools.} These schools are of three classes: national, regional, and state. The public offices of the nation are grouped in four divisions. The first division of public trust pertains principally to the national administration, and all officeholders of this group must be graduates of both regional and national schools of statesmanship. Individuals may accept political, elective, or appointive office in the second division upon graduating from any one of the ten regional schools of statesmanship; their trusts concern responsibilities in the regional administration and the state governments. Division three includes state responsibilities, and such officials are only required to have state degrees of statesmanship. The fourth and last division of officeholders are not required to hold statesmanship degrees, such offices being wholly appointive. They represent minor positions of assistantship, secretaryships, and technical trusts which are discharged by the various learned professions functioning in governmental administrative capacities.
\vs p072 8:3 Judges of the minor and state courts hold degrees from the state schools of statesmanship. Judges of the jurisdictional tribunals of social, educational, and industrial matters hold degrees from the regional schools. Judges of the federal supreme court must hold degrees from all these schools of statesmanship.
\vs p072 8:4 \pc \ublistelem{2.}\bibnobreakspace \bibemph{Schools of philosophy.} These schools are affiliated with the temples of philosophy and are more or less associated with religion as a public function.
\vs p072 8:5 \pc \ublistelem{3.}\bibnobreakspace \bibemph{Institutions of science.} These technical schools are co\hyp{}ordinated with industry rather than with the educational system and are administered under fifteen divisions.
\vs p072 8:6 \pc \ublistelem{4.}\bibnobreakspace \bibemph{Professional training schools.} These special institutions provide the technical training for the various learned professions, twelve in number.
\vs p072 8:7 \pc \ublistelem{5.}\bibnobreakspace \bibemph{Military and naval schools.} Near the national headquarters and at the twenty\hyp{}five coastal military centers are maintained those institutions devoted to the military training of volunteer citizens from eighteen to thirty years of age. Parental consent is required before twenty\hyp{}five in order to gain entrance to these schools.
\usection{9.\bibnobreakspace The Plan of Universal Suffrage}
\vs p072 9:1 Although candidates for all public offices are restricted to graduates of the state, regional, or federal schools of statesmanship, the progressive leaders of this nation discovered a serious weakness in their plan of universal suffrage and about fifty years ago made constitutional provision for a modified scheme of voting which embraces the following features:
\vs p072 9:2 \ublistelem{1.}\bibnobreakspace Every man and woman of twenty years and over has one vote. Upon attaining this age, all citizens must accept membership in two voting groups: They will join the first in accordance with their economic function --- industrial, professional, agricultural, or trade; they will enter the second group according to their political, philosophic, and social inclinations. All workers thus belong to some economic franchise group, and these guilds, like the noneconomic associations, are regulated much as is the national government with its threefold division of powers. Registration in these groups cannot be changed for twelve years.
\vs p072 9:3 \pc \ublistelem{2.}\bibnobreakspace Upon nomination by the state governors or by the regional executives and by the mandate of the regional supreme councils, individuals who have rendered great service to society, or who have demonstrated extraordinary wisdom in government service, may have additional votes conferred upon them not oftener than every five years and not to exceed nine such superfranchises. The maximum suffrage of any multiple voter is ten. Scientists, inventors, teachers, philosophers, and spiritual leaders are also thus recognized and honored with augmented political power. These advanced civic privileges are conferred by the state and regional supreme councils much as degrees are bestowed by the special colleges, and the recipients are proud to attach the symbols of such civic recognition, along with their other degrees, to their lists of personal achievements.
\vs p072 9:4 \pc \ublistelem{3.}\bibnobreakspace All individuals sentenced to compulsory labor in the mines and all governmental servants supported by tax funds are, for the periods of such services, disenfranchised. This does not apply to aged persons who may be retired on pensions at sixty\hyp{}five.
\vs p072 9:5 \pc \ublistelem{4.}\bibnobreakspace There are five brackets of suffrage reflecting the average yearly taxes paid for each half\hyp{}decade period. Heavy taxpayers are permitted extra votes up to five. This grant is independent of all other recognition, but in no case can any person cast over ten ballots.
\vs p072 9:6 \pc \ublistelem{5.}\bibnobreakspace At the time this franchise plan was adopted, the territorial method of voting was abandoned in favor of the economic or functional system. All citizens now vote as members of industrial, social, or professional groups, regardless of their residence. Thus the electorate consists of solidified, unified, and intelligent groups who elect only their best members to positions of governmental trust and responsibility. There is one exception to this scheme of functional or group suffrage: The election of a federal chief executive every six years is by nation\hyp{}wide ballot, and no citizen casts over one vote.
\vs p072 9:7 \pc Thus, except in the election of the chief executive, suffrage is exercised by economic, professional, intellectual, and social groupings of the citizenry. The ideal state is organic, and every free and intelligent group of citizens represents a vital and functioning organ within the larger governmental organism.
\vs p072 9:8 The schools of statesmanship have power to start proceedings in the state courts looking toward the disenfranchisement of any defective, idle, indifferent, or criminal individual. These people recognize that, when fifty per cent of a nation is inferior or defective and possesses the ballot, such a nation is doomed. They believe the dominance of mediocrity spells the downfall of any nation. Voting is compulsory, heavy fines being assessed against all who fail to cast their ballots.
\usection{10.\bibnobreakspace Dealing with Crime}
\vs p072 10:1 The methods of this people in dealing with crime, insanity, and degeneracy, while in some ways pleasing, will, no doubt, in others prove shocking to most Urantians. Ordinary criminals and the defectives are placed, by sexes, in different agricultural colonies and are more than self\hyp{}supporting. The more serious habitual criminals and the incurably insane are sentenced to death in the lethal gas chambers by the courts. Numerous crimes aside from murder, including betrayal of governmental trust, also carry the death penalty, and the visitation of justice is sure and swift.
\vs p072 10:2 These people are passing out of the negative into the positive era of law. Recently they have gone so far as to attempt the prevention of crime by sentencing those who are believed to be potential murderers and major criminals to life service in the detention colonies. If such convicts subsequently demonstrate that they have become more normal, they may be either paroled or pardoned. The homicide rate on this continent is only one per cent of that among the other nations.
\vs p072 10:3 Efforts to prevent the breeding of criminals and defectives were begun over one hundred years ago and have already yielded gratifying results. There are no prisons or hospitals for the insane. For one reason, there are only about ten per cent as many of these groups as are found on Urantia.
\usection{11.\bibnobreakspace Military Preparedness}
\vs p072 11:1 Graduates of the federal military schools may be commissioned as “guardians of civilization” in seven ranks, in accordance with ability and experience, by the president of the National Council of Defense. This council consists of twenty\hyp{}five members, nominated by the highest parental, educational, and industrial tribunals, confirmed by the federal supreme court, and presided over ex officio by the chief of staff of co\hyp{}ordinated military affairs. Such members serve until they are seventy years of age.
\vs p072 11:2 The courses pursued by such commissioned officers are four years in length and are invariably correlated with the mastery of some trade or profession. Military training is never given without this associated industrial, scientific, or professional schooling. When military training is finished, the individual has, during his four years’ course, received one half of the education imparted in any of the special schools where the courses are likewise four years in length. In this way the creation of a professional military class is avoided by providing this opportunity for a large number of men to support themselves while securing the first half of a technical or professional training.
\vs p072 11:3 Military service during peacetime is purely voluntary, and the enlistments in all branches of the service are for four years, during which every man pursues some special line of study in addition to the mastery of military tactics. Training in music is one of the chief pursuits of the central military schools and of the twenty\hyp{}five training camps distributed about the periphery of the continent. During periods of industrial slackness many thousands of unemployed are automatically utilized in upbuilding the military defenses of the continent on land and sea and in the air.
\vs p072 11:4 \pc Although these people maintain a powerful war establishment as a defense against invasion by the surrounding hostile peoples, it may be recorded to their credit that they have not in over one hundred years employed these military resources in an offensive war. They have become civilized to that point where they can vigorously defend civilization without yielding to the temptation to utilize their war powers in aggression. There have been no civil wars since the establishment of the united continental state, but during the last two centuries these people have been called upon to wage nine fierce defensive conflicts, three of which were against mighty confederations of world powers. Although this nation maintains adequate defense against attack by hostile neighbors, it pays far more attention to the training of statesmen, scientists, and philosophers.
\vs p072 11:5 When at peace with the world, all mobile defense mechanisms are quite fully employed in trade, commerce, and recreation. When war is declared, the entire nation is mobilized. Throughout the period of hostilities military pay obtains in all industries, and the chiefs of all military departments become members of the chief executive’s cabinet.
\usection{12.\bibnobreakspace The Other Nations}
\vs p072 12:1 Although the society and government of this unique people are in many respects superior to those of the Urantia nations, it should be stated that on the other continents (there are eleven on this planet) the governments are decidedly inferior to the more advanced nations of Urantia.
\vs p072 12:2 Just now this superior government is planning to establish ambassadorial relations with the inferior peoples, and for the first time a great religious leader has arisen who advocates the sending of missionaries to these surrounding nations. We fear they are about to make the mistake that so many others have made when they have endeavored to force a superior culture and religion upon other races. What a wonderful thing could be done on this world if this continental nation of advanced culture would only go out and bring to itself the best of the neighboring peoples and then, after educating them, send them back as emissaries of culture to their benighted brethren! Of course, if a Magisterial Son should soon come to this advanced nation, great things could quickly happen on this world.
\vs p072 12:3 \pc This recital of the affairs of a neighboring planet is made by special permission with the intent of advancing civilization and augmenting governmental evolution on Urantia. Much more could be narrated that would no doubt interest and intrigue Urantians, but this disclosure covers the limits of our permissive mandate.
\vs p072 12:4 \pc Urantians should, however, take note that their sister sphere in the Satania family has benefited by neither magisterial nor bestowal missions of the Paradise Sons. Neither are the various peoples of Urantia set off from each other by such disparity of culture as separates the continental nation from its planetary fellows.
\vs p072 12:5 The pouring out of the Spirit of Truth provides the spiritual foundation for the realization of great achievements in the interests of the human race of the bestowal world. Urantia is therefore far better prepared for the more immediate realization of a planetary government with its laws, mechanisms, symbols, conventions, and language --- all of which could contribute so mightily to the establishment of world\hyp{}wide peace under law and could lead to the sometime dawning of a real age of spiritual striving; and such an age is the planetary threshold to the utopian ages of light and life.
\vsetoff
\vs p072 12:6 [Presented by a Melchizedek of Nebadon.]
