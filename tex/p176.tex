\upaper{176}{Tuesday Evening on Mount Olivet}
\vs p176 0:1 THIS Tuesday afternoon, as Jesus and the apostles passed out of the temple on their way to the Gethsemane camp, Matthew, calling attention to the temple construction, said: “Master, observe what manner of buildings these are. See the massive stones and the beautiful adornment; can it be that these buildings are to be destroyed?” As they went on toward Olivet, Jesus said: \textcolor{ubdarkred}{“You see these stones and this massive temple; verily, verily, I say to you: In the days soon to come there shall not be left one stone upon another. They shall all be thrown down.”} These remarks depicting the destruction of the sacred temple aroused the curiosity of the apostles as they walked along behind the Master; they could conceive of no event short of the end of the world which would occasion the destruction of the temple.
\vs p176 0:2 In order to avoid the crowds passing along the Kidron valley toward Gethsemane, Jesus and his associates were minded to climb up the western slope of Olivet for a short distance and then follow a trail over to their private camp near Gethsemane located a short distance above the public camping ground. As they turned to leave the road leading on to Bethany, they observed the temple, glorified by the rays of the setting sun; and while they tarried on the mount, they saw the lights of the city appear and beheld the beauty of the illuminated temple; and there, under the mellow light of the full moon, Jesus and the twelve sat down. The Master talked with them, and presently Nathaniel asked this question: “Tell us, Master, how shall we know when these events are about to come to pass?”
\usection{1.\bibnobreakspace The Destruction of Jerusalem}
\vs p176 1:1 In answering Nathaniel’s question, Jesus said: \textcolor{ubdarkred}{“Yes, I will tell you about the times when this people shall have filled up the cup of their iniquity; when justice shall swiftly descend upon this city of our fathers. I am about to leave you; I go to the Father. After I leave you, take heed that no man deceive you, for many will come as deliverers and will lead many astray. When you hear of wars and rumors of wars, be not troubled, for though all these things will happen, the end of Jerusalem is not yet at hand. You should not be perturbed by famines or earthquakes; neither should you be concerned when you are delivered up to the civil authorities and are persecuted for the sake of the gospel. You will be thrown out of the synagogue and put in prison for my sake, and some of you will be killed. When you are brought up before governors and rulers, it shall be for a testimony of your faith and to show your steadfastness in the gospel of the kingdom. And when you stand before judges, be not anxious beforehand as to what you should say, for the spirit will teach you in that very hour what you should answer your adversaries. In these days of travail, even your own kinsfolk, under the leadership of those who have rejected the Son of Man, will deliver you up to prison and death. For a time you may be hated by all men for my sake, but even in these persecutions I will not forsake you; my spirit will not desert you. Be patient! doubt not that this gospel of the kingdom will triumph over all enemies and, eventually, be proclaimed to all nations.”}
\vs p176 1:2 Jesus paused while he looked down upon the city. The Master realized that the rejection of the spiritual concept of the Messiah, the determination to cling persistently and blindly to the material mission of the expected deliverer, would presently bring the Jews in direct conflict with the powerful Roman armies, and that such a contest could only result in the final and complete overthrow of the Jewish nation. When his people rejected his spiritual bestowal and refused to receive the light of heaven as it so mercifully shone upon them, they thereby sealed their doom as an independent people with a special spiritual mission on earth. Even the Jewish leaders subsequently recognized that it was this secular idea of the Messiah which directly led to the turbulence which eventually brought about their destruction.
\vs p176 1:3 Since Jerusalem was to become the cradle of the early gospel movement, Jesus did not want its teachers and preachers to perish in the terrible overthrow of the Jewish people in connection with the destruction of Jerusalem; wherefore did he give these instructions to his followers. Jesus was much concerned lest some of his disciples become involved in these soon\hyp{}coming revolts and so perish in the downfall of Jerusalem.
\vs p176 1:4 Then Andrew inquired: “But, Master, if the Holy City and the temple are to be destroyed, and if you are not here to direct us, when should we forsake Jerusalem?” Said Jesus: \textcolor{ubdarkred}{“You may remain in the city after I have gone, even through these times of travail and bitter persecution, but when you finally see Jerusalem being encompassed by the Roman armies after the revolt of the false prophets, then will you know that her desolation is at hand; then must you flee to the mountains. Let none who are in the city and around about tarry to save aught, neither let those who are outside dare to enter therein. There will be great tribulation, for these will be the days of gentile vengeance. And after you have deserted the city, this disobedient people will fall by the edge of the sword and will be led captive into all nations; and so shall Jerusalem be trodden down by the gentiles. In the meantime, I warn you, be not deceived. If any man comes to you, saying, ‘Behold, here is the Deliverer,’ or ‘Behold, there is he,’ believe it not, for many false teachers will arise and many will be led astray; but you should not be deceived, for I have told you all this beforehand.”}
\vs p176 1:5 The apostles sat in silence in the moonlight for a considerable time while these astounding predictions of the Master sank into their bewildered minds. And it was in conformity with this very warning that practically the entire group of believers and disciples fled from Jerusalem upon the first appearance of the Roman troops, finding a safe shelter in Pella to the north.
\vs p176 1:6 Even after this explicit warning, many of Jesus’ followers interpreted these predictions as referring to the changes which would obviously occur in Jerusalem when the reappearing of the Messiah would result in the establishment of the New Jerusalem and in the enlargement of the city to become the world’s capital. In their minds these Jews were determined to connect the destruction of the temple with the “end of the world.” They believed this New Jerusalem would fill all Palestine; that the end of the world would be followed by the immediate appearance of the “new heavens and the new earth.” And so it was not strange that Peter should say: “Master, we know that all things will pass away when the new heavens and the new earth appear, but how shall we know when you will return to bring all this about?”
\vs p176 1:7 When Jesus heard this, he was thoughtful for some time and then said: \textcolor{ubdarkred}{“You ever err since you always try to attach the new teaching to the old; you are determined to misunderstand all my teaching; you insist on interpreting the gospel in accordance with your established beliefs. Nevertheless, I will try to enlighten you.”}
\usection{2.\bibnobreakspace The Master’s Second Coming}
\vs p176 2:1 On several occasions Jesus had made statements which led his hearers to infer that, while he intended presently to leave this world, he would most certainly return to consummate the work of the heavenly kingdom. As the conviction grew on his followers that he was going to leave them, and after he had departed from this world, it was only natural for all believers to lay fast hold upon these promises to return. The doctrine of the second coming of Christ thus became early incorporated into the teachings of the Christians, and almost every subsequent generation of disciples has devoutly believed this truth and has confidently looked forward to his sometime coming.
\vs p176 2:2 If they were to part with their Master and Teacher, how much more did these first disciples and the apostles grasp at this promise to return, and they lost no time in associating the predicted destruction of Jerusalem with this promised second coming. And they continued thus to interpret his words notwithstanding that, throughout this evening of instruction on Mount Olivet, the Master took particular pains to prevent just such a mistake.
\vs p176 2:3 \pc In further answer to Peter’s question, Jesus said: \textcolor{ubdarkred}{“Why do you still look for the Son of Man to sit upon the throne of David and expect that the material dreams of the Jews will be fulfilled? Have I not told you all these years that my kingdom is not of this world? The things which you now look down upon are coming to an end, but this will be a new beginning out of which the gospel of the kingdom will go to all the world and this salvation will spread to all peoples. And when the kingdom shall have come to its full fruition, be assured that the Father in heaven will not fail to visit you with an enlarged revelation of truth and an enhanced demonstration of righteousness, even as he has already bestowed upon this world him who became the prince of darkness, and then Adam, who was followed by Melchizedek, and in these days, the Son of Man. And so will my Father continue to manifest his mercy and show forth his love, even to this dark and evil world. So also will I, after my Father has invested me with all power and authority, continue to follow your fortunes and to guide in the affairs of the kingdom by the presence of my spirit, who shall shortly be poured out upon all flesh. Even though I shall thus be present with you in spirit, I also promise that I will sometime return to this world, where I have lived this life in the flesh and achieved the experience of simultaneously revealing God to man and leading man to God. Very soon must I leave you and take up the work the Father has intrusted to my hands, but be of good courage, for I will sometime return. In the meantime, my Spirit of the Truth of a universe shall comfort and guide you.}
\vs p176 2:4 \textcolor{ubdarkred}{“You behold me now in weakness and in the flesh, but when I return, it shall be with power and in the spirit. The eye of flesh beholds the Son of Man in the flesh, but only the eye of the spirit will behold the Son of Man glorified by the Father and appearing on earth in his own name.}
\vs p176 2:5 \textcolor{ubdarkred}{“But the times of the reappearing of the Son of Man are known only in the councils of Paradise; not even the angels of heaven know when this will occur. However, you should understand that, when this gospel of the kingdom shall have been proclaimed to all the world for the salvation of all peoples, and when the fullness of the age has come to pass, the Father will send you another dispensational bestowal, or else the Son of Man will return to adjudge the age.}
\vs p176 2:6 \textcolor{ubdarkred}{“And now concerning the travail of Jerusalem, about which I have spoken to you, even this generation will not pass away until my words are fulfilled; but concerning the times of the coming again of the Son of Man, no one in heaven or on earth may presume to speak. But you should be wise regarding the ripening of an age; you should be alert to discern the signs of the times. You know when the fig tree shows its tender branches and puts forth its leaves that summer is near. Likewise, when the world has passed through the long winter of material\hyp{}mindedness and you discern the coming of the spiritual springtime of a new dispensation, should you know that the summertime of a new visitation draws near.}
\vs p176 2:7 \textcolor{ubdarkred}{“But what is the significance of this teaching having to do with the coming of the Sons of God? Do you not perceive that, when each of you is called to lay down his life struggle and pass through the portal of death, you stand in the immediate presence of judgment, and that you are face to face with the facts of a new dispensation of service in the eternal plan of the infinite Father? What the whole world must face as a literal fact at the end of an age, you, as individuals, must each most certainly face as a personal experience when you reach the end of your natural life and thereby pass on to be confronted with the conditions and demands inherent in the next revelation of the eternal progression of the Father’s kingdom.”}
\vs p176 2:8 Of all the discourses which the Master gave his apostles, none ever became so confused in their minds as this one, given this Tuesday evening on the Mount of Olives, regarding the twofold subject of the destruction of Jerusalem and his own second coming. There was, therefore, little agreement between the subsequent written accounts based on the memories of what the Master said on this extraordinary occasion. Consequently, when the records were left blank concerning much that was said that Tuesday evening, there grew up many traditions; and very early in the second century a Jewish apocalyptic about the Messiah written by one Selta, who was attached to the court of the Emperor Caligula, was bodily copied into the Matthew Gospel and subsequently added (in part) to the Mark and Luke records. It was in these writings of Selta that the parable of the ten virgins appeared. No part of the gospel record ever suffered such confusing misconstruction as this evening’s teaching. But the Apostle John never became thus confused.
\vs p176 2:9 As these thirteen men resumed their journey toward the camp, they were speechless and under great emotional tension. Judas had finally confirmed his decision to abandon his associates. It was a late hour when David Zebedee, John Mark, and a number of the leading disciples welcomed Jesus and the twelve to the new camp, but the apostles did not want to sleep; they wanted to know more about the destruction of Jerusalem, the Master’s departure, and the end of the world.
\usection{3.\bibnobreakspace Later Discussion at the Camp}
\vs p176 3:1 As they gathered about the campfire, some twenty of them, Thomas asked: “Since you are to return to finish the work of the kingdom, what should be our attitude while you are away on the Father’s business?” As Jesus looked them over by the firelight, he answered:
\vs p176 3:2 \pc \textcolor{ubdarkred}{“And even you, Thomas, fail to comprehend what I have been saying. Have I not all this time taught you that your connection with the kingdom is spiritual and individual, wholly a matter of personal experience in the spirit by the faith\hyp{}realization that you are a son of God? What more shall I say? The downfall of nations, the crash of empires, the destruction of the unbelieving Jews, the end of an age, even the end of the world, what have these things to do with one who believes this gospel, and who has hid his life in the surety of the eternal kingdom? You who are God\hyp{}knowing and gospel\hyp{}believing have already received the assurances of eternal life. Since your lives have been lived in the spirit and for the Father, nothing can be of serious concern to you. Kingdom builders, the accredited citizens of the heavenly worlds, are not to be disturbed by temporal upheavals or perturbed by terrestrial cataclysms. What does it matter to you who believe this gospel of the kingdom if nations overturn, the age ends, or all things visible crash, since you know that your life is the gift of the Son, and that it is eternally secure in the Father? Having lived the temporal life by faith and having yielded the fruits of the spirit as the righteousness of loving service for your fellows, you can confidently look forward to the next step in the eternal career with the same survival faith that has carried you through your first and earthly adventure in sonship with God.}
\vs p176 3:3 \textcolor{ubdarkred}{“Each generation of believers should carry on their work, in view of the possible return of the Son of Man, exactly as each individual believer carries forward his lifework in view of inevitable and ever\hyp{}impending natural death. When you have by faith once established yourself as a son of God, nothing else matters as regards the surety of survival. But make no mistake! this survival faith is a living faith, and it increasingly manifests the fruits of that divine spirit which first inspired it in the human heart. That you have once accepted sonship in the heavenly kingdom will not save you in the face of the knowing and persistent rejection of those truths which have to do with the progressive spiritual fruit\hyp{}bearing of the sons of God in the flesh. You who have been with me in the Father’s business on earth can even now desert the kingdom if you find that you love not the way of the Father’s service for mankind.}
\vs p176 3:4 “As individuals, and as a generation of believers, hear me while I speak a parable: There was a certain great man who, before starting out on a long journey to another country, called all his trusted servants before him and delivered into their hands all his goods. To one he gave five talents, to another two, and to another one. And so on down through the entire group of honored stewards, to each he intrusted his goods according to their several abilities; and then he set out on his journey. When their lord had departed, his servants set themselves at work to gain profits from the wealth intrusted to them. Immediately he who had received five talents began to trade with them and very soon had made a profit of another five talents. In like manner he who had received two talents soon had gained two more. And so did all of these servants make gains for their master except him who received but one talent. He went away by himself and dug a hole in the earth where he hid his lord’s money. Presently the lord of those servants unexpectedly returned and called upon his stewards for a reckoning. And when they had all been called before their master, he who had received the five talents came forward with the money which had been intrusted to him and brought five additional talents, saying, ‘Lord, you gave me five talents to invest, and I am glad to present five other talents as my gain.’ And then his lord said to him: ‘Well done, good and faithful servant, you have been faithful over a few things; I will now set you as steward over many; enter forthwith into the joy of your lord.’ And then he who had received the two talents came forward, saying: ‘Lord, you delivered into my hands two talents; behold, I have gained these other two talents.’ And his lord then said to him: ‘Well done, good and faithful steward; you also have been faithful over a few things, and I will now set you over many; enter you into the joy of your lord.’ And then there came to the accounting he who had received the one talent. This servant came forward, saying, ‘Lord, I knew you and realized that you were a shrewd man in that you expected gains where you had not personally labored; therefore was I afraid to risk aught of that which was intrusted to me. I safely hid your talent in the earth; here it is; you now have what belongs to you.’ But his lord answered: ‘You are an indolent and slothful steward. By your own words you confess that you knew I would require of you an accounting with reasonable profit, such as your diligent fellow servants have this day rendered. Knowing this, you ought, therefore, to have at least put my money into the hands of the bankers that on my return I might have received my own with interest.’ And then to the chief steward this lord said: ‘Take away this one talent from this unprofitable servant and give it to him who has the ten talents.’\fnc{And so did all of these servants make gains for their master except \bibtextul{he} who received but one talent. \bibexpl{The pronoun is the object of the preposition “except” therefore “him” is correct. See last sentence in subject paragraph for parallel usage where “him” is object of “to” also creating a “him who” phrase.}}
\vs p176 3:5 \textcolor{ubdarkred}{“To every one who has, more shall be given, and he shall have abundance; but from him who has not, even that which he has shall be taken away. You cannot stand still in the affairs of the eternal kingdom. My Father requires all his children to grow in grace and in a knowledge of the truth. You who know these truths must yield the increase of the fruits of the spirit and manifest a growing devotion to the unselfish service of your fellow servants. And remember that, inasmuch as you minister to one of the least of my brethren, you have done this service to me.}
\vs p176 3:6 \textcolor{ubdarkred}{“And so should you go about the work of the Father’s business, now and henceforth, even forevermore. Carry on until I come. In faithfulness do that which is intrusted to you, and thereby shall you be ready for the reckoning call of death. And having thus lived for the glory of the Father and the satisfaction of the Son, you shall enter with joy and exceedingly great pleasure into the eternal service of the everlasting kingdom.”}
\vs p176 3:7 \pc Truth is living; the Spirit of Truth is ever leading the children of light into new realms of spiritual reality and divine service. You are not given truth to crystallize into settled, safe, and honored forms. Your revelation of truth must be so enhanced by passing through your personal experience that new beauty and actual spiritual gains will be disclosed to all who behold your spiritual fruits and in consequence thereof are led to glorify the Father who is in heaven. Only those faithful servants who thus grow in the knowledge of the truth, and who thereby develop the capacity for divine appreciation of spiritual realities, can ever hope to “enter fully into the joy of their Lord.” What a sorry sight for successive generations of the professed followers of Jesus to say, regarding their stewardship of divine truth: “Here, Master, is the truth you committed to us a hundred or a thousand years ago. We have lost nothing; we have faithfully preserved all you gave us; we have allowed no changes to be made in that which you taught us; here is the truth you gave us.” But such a plea concerning spiritual indolence will not justify the barren steward of truth in the presence of the Master. In accordance with the truth committed to your hands will the Master of truth require a reckoning.
\vs p176 3:8 In the next world you will be asked to give an account of the endowments and stewardships of this world. Whether inherent talents are few or many, a just and merciful reckoning must be faced. If endowments are used only in selfish pursuits and no thought is bestowed upon the higher duty of obtaining increased yield of the fruits of the spirit, as they are manifested in the ever\hyp{}expanding service of men and the worship of God, such selfish stewards must accept the consequences of their deliberate choosing.
\vs p176 3:9 And how much like all selfish mortals was this unfaithful servant with the one talent in that he blamed his slothfulness directly upon his lord. How prone is man, when he is confronted with the failures of his own making, to put the blame upon others, oftentimes upon those who least deserve it!
\vs p176 3:10 Said Jesus that night as they went to their rest: \textcolor{ubdarkred}{“Freely have you received; therefore freely should you give of the truth of heaven, and in the giving will this truth multiply and show forth the increasing light of saving grace, even as you minister it.”}
\usection{4.\bibnobreakspace The Return of Michael}
\vs p176 4:1 Of all the Master’s teachings no one phase has been so misunderstood as his promise sometime to come back in person to this world. It is not strange that Michael should be interested in sometime returning to the planet whereon he experienced his seventh and last bestowal, as a mortal of the realm. It is only natural to believe that Jesus of Nazareth, now sovereign ruler of a vast universe, would be interested in coming back, not only once but even many times, to the world whereon he lived such a unique life and finally won for himself the Father’s unlimited bestowal of universe power and authority. Urantia will eternally be one of the seven nativity spheres of Michael in the winning of universe sovereignty.
\vs p176 4:2 Jesus did, on numerous occasions and to many individuals, declare his intention of returning to this world. As his followers awakened to the fact that their Master was not going to function as a temporal deliverer, and as they listened to his predictions of the overthrow of Jerusalem and the downfall of the Jewish nation, they most naturally began to associate his promised return with these catastrophic events. But when the Roman armies leveled the walls of Jerusalem, destroyed the temple, and dispersed the Judean Jews, and still the Master did not reveal himself in power and glory, his followers began the formulation of that belief which eventually associated the second coming of Christ with the end of the age, even with the end of the world.
\vs p176 4:3 Jesus promised to do two things after he had ascended to the Father, and after all power in heaven and on earth had been placed in his hands. He promised, first, to send into the world, and in his stead, another teacher, the Spirit of Truth; and this he did on the day of Pentecost. Second, he most certainly promised his followers that he would sometime personally return to this world. But he did not say how, where, or when he would revisit this planet of his bestowal experience in the flesh. On one occasion he intimated that, whereas the eye of flesh had beheld him when he lived here in the flesh, on his return (at least on one of his possible visits) he would be discerned only by the eye of spiritual faith.
\vs p176 4:4 Many of us are inclined to believe that Jesus will return to Urantia many times during the ages to come. We do not have his specific promise to make these plural visits, but it seems most probable that he who carries among his universe titles that of Planetary Prince of Urantia will many times visit the world whose conquest conferred such a unique title upon him.
\vs p176 4:5 We most positively believe that Michael will again come in person to Urantia, but we have not the slightest idea as to when or in what manner he may choose to come. Will his second advent on earth be timed to occur in connection with the terminal judgment of this present age, either with or without the associated appearance of a Magisterial Son? Will he come in connection with the termination of some subsequent Urantian age? Will he come unannounced and as an isolated event? We do not know. Only one thing we are certain of, that is, when he does return, all the world will likely know about it, for he must come as the supreme ruler of a universe and not as the obscure babe of Bethlehem. But if every eye is to behold him, and if only spiritual eyes are to discern his presence, then must his advent be long deferred.
\vs p176 4:6 You would do well, therefore, to disassociate the Master’s personal return to earth from any and all set events or settled epochs. We are sure of only one thing: He has promised to come back. We have no idea as to when he will fulfill this promise or in what connection. As far as we know, he may appear on earth any day, and he may not come until age after age has passed and been duly adjudicated by his associated Sons of the Paradise corps.
\vs p176 4:7 The second advent of Michael on earth is an event of tremendous sentimental value to both midwayers and humans; but otherwise it is of no immediate moment to midwayers and of no more practical importance to human beings than the common event of natural death, which so suddenly precipitates mortal man into the immediate grasp of that succession of universe events which leads directly to the presence of this same Jesus, the sovereign ruler of our universe. The children of light are all destined to see him, and it is of no serious concern whether we go to him or whether he should chance first to come to us. Be you therefore ever ready to welcome him on earth as he stands ready to welcome you in heaven. We confidently look for his glorious appearing, even for repeated comings, but we are wholly ignorant as to how, when, or in what connection he is destined to appear.
