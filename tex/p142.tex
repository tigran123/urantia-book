\upaper{142}{The Passover at Jerusalem}
\vs p142 0:1 THE month of April Jesus and the apostles worked in Jerusalem, going out of the city each evening to spend the night at Bethany. Jesus himself spent one or two nights each week in Jerusalem at the home of Flavius, a Greek Jew, where many prominent Jews came in secret to interview him.
\vs p142 0:2 \pc The first day in Jerusalem Jesus called upon his friend of former years, Annas, the onetime high priest and relative of Salome, Zebedee’s wife. Annas had been hearing about Jesus and his teachings, and when Jesus called at the high priest’s home, he was received with much reserve. When Jesus perceived Annas’s coldness, he took immediate leave, saying as he departed: \textcolor{ubdarkred}{“Fear is man’s chief enslaver and pride his great weakness; will you betray yourself into bondage to both of these destroyers of joy and liberty?”} But Annas made no reply. The Master did not again see Annas until the time when he sat with his son\hyp{}in\hyp{}law in judgment on the Son of Man.
\usection{1.\bibnobreakspace Teaching in the Temple}
\vs p142 1:1 Throughout this month Jesus or one of the apostles taught daily in the temple. When the Passover crowds were too great to find entrance to the temple teaching, the apostles conducted many teaching groups outside the sacred precincts. The burden of their message was:
\vs p142 1:2 \ublistelem{1.}\bibnobreakspace The kingdom of heaven is at hand.
\vs p142 1:3 \ublistelem{2.}\bibnobreakspace By faith in the fatherhood of God you may enter the kingdom of heaven, thus becoming the sons of God.
\vs p142 1:4 \ublistelem{3.}\bibnobreakspace Love is the rule of living within the kingdom --- supreme devotion to God while loving your neighbor as yourself.
\vs p142 1:5 \ublistelem{4.}\bibnobreakspace Obedience to the will of the Father, yielding the fruits of the spirit in one’s personal life, is the law of the kingdom.
\vs p142 1:6 \pc The multitudes who came to celebrate the Passover heard this teaching of Jesus, and hundreds of them rejoiced in the good news. The chief priests and rulers of the Jews became much concerned about Jesus and his apostles and debated among themselves as to what should be done with them.
\vs p142 1:7 Besides teaching in and about the temple, the apostles and other believers were engaged in doing much personal work among the Passover throngs. These interested men and women carried the news of Jesus’ message from this Passover celebration to the uttermost parts of the Roman Empire and also to the East. This was the beginning of the spread of the gospel of the kingdom to the outside world. No longer was the work of Jesus to be confined to Palestine.
\usection{2.\bibnobreakspace God’s Wrath}
\vs p142 2:1 There was in Jerusalem in attendance upon the Passover festivities one Jacob, a wealthy Jewish trader from Crete, and he came to Andrew making request to see Jesus privately. Andrew arranged this secret meeting with Jesus at Flavius’s home the evening of the next day. This man could not comprehend the Master’s teachings, and he came because he desired to inquire more fully about the kingdom of God. Said Jacob to Jesus: “But, Rabbi, Moses and the olden prophets tell us that Yahweh is a jealous God, a God of great wrath and fierce anger. The prophets say he hates evildoers and takes vengeance on those who obey not his law. You and your disciples teach us that God is a kind and compassionate Father who so loves all men that he would welcome them into this new kingdom of heaven, which you proclaim is so near at hand.”
\vs p142 2:2 \pc When Jacob finished speaking, Jesus replied: \textcolor{ubdarkred}{“Jacob, you have well stated the teachings of the olden prophets who taught the children of their generation in accordance with the light of their day. Our Father in Paradise is changeless. But the concept of his nature has enlarged and grown from the days of Moses down through the times of Amos and even to the generation of the prophet Isaiah. And now have I come in the flesh to reveal the Father in new glory and to show forth his love and mercy to all men on all worlds. As the gospel of this kingdom shall spread over the world with its message of good cheer and good will to all men, there will grow up improved and better relations among the families of all nations. As time passes, fathers and their children will love each other more, and thus will be brought about a better understanding of the love of the Father in heaven for his children on earth. Remember, Jacob, that a good and true father not only loves his family as a whole --- as a family --- but he also truly loves and affectionately cares for each \bibemph{individual} member.”}
\vs p142 2:3 After considerable discussion of the heavenly Father’s character, Jesus paused to say: \textcolor{ubdarkred}{“You, Jacob, being a father of many, know well the truth of my words.”} And Jacob said: “But, Master, who told you I was the father of six children? How did you know this about me?” And the Master replied: \textcolor{ubdarkred}{“Suffice it to say that the Father and the Son know all things, for indeed they see all. Loving your children as a father on earth, you must now accept as a reality the love of the heavenly Father for \bibemph{you ---} not just for all the children of Abraham, but for you, your individual soul.”}
\vs p142 2:4 \pc Then Jesus went on to say: \textcolor{ubdarkred}{“When your children are very young and immature, and when you must chastise them, they may reflect that their father is angry and filled with resentful wrath. Their immaturity cannot penetrate beyond the punishment to discern the father’s farseeing and corrective affection. But when these same children become grown\hyp{}up men and women, would it not be folly for them to cling to these earlier and misconceived notions regarding their father? As men and women they should now discern their father’s love in all these early disciplines. And should not mankind, as the centuries pass, come the better to understand the true nature and loving character of the Father in heaven? What profit have you from successive generations of spiritual illumination if you persist in viewing God as Moses and the prophets saw him? I say to you, Jacob, under the bright light of this hour you should see the Father as none of those who have gone before ever beheld him. And thus seeing him, you should rejoice to enter the kingdom wherein such a merciful Father rules, and you should seek to have his will of love dominate your life henceforth.”}
\vs p142 2:5 And Jacob answered: “Rabbi, I believe; I desire that you lead me into the Father’s kingdom.”
\usection{3.\bibnobreakspace The Concept of God}
\vs p142 3:1 The twelve apostles, most of whom had listened to this discussion of the character of God, that night asked Jesus many questions about the Father in heaven. The Master’s answers to these questions can best be presented by the following summary in modern phraseology:
\vs p142 3:2 Jesus mildly upbraided the twelve, in substance saying: Do you not know the traditions of Israel relating to the growth of the idea of Yahweh, and are you ignorant of the teaching of the Scriptures concerning the doctrine of God? And then did the Master proceed to instruct the apostles about the evolution of the concept of Deity throughout the course of the development of the Jewish people. He called attention to the following phases of the growth of the God idea:
\vs p142 3:3 \ublistelem{1.}\bibnobreakspace \bibemph{Yahweh ---} the god of the Sinai clans. This was the primitive concept of Deity which Moses exalted to the higher level of the Lord God of Israel. The Father in heaven never fails to accept the sincere worship of his children on earth, no matter how crude their concept of Deity or by what name they symbolize his divine nature.
\vs p142 3:4 \pc \ublistelem{2.}\bibnobreakspace \bibemph{The Most High.} This concept of the Father in heaven was proclaimed by Melchizedek to Abraham and was carried far from Salem by those who subsequently believed in this enlarged and expanded idea of Deity. Abraham and his brother left Ur because of the establishment of sun worship, and they became believers in Melchizedek’s teaching of El Elyon --- the Most High God. Theirs was a composite concept of God, consisting in a blending of their older Mesopotamian ideas and the Most High doctrine.
\vs p142 3:5 \pc \ublistelem{3.}\bibnobreakspace \bibemph{El Shaddai.} During these early days many of the Hebrews worshiped El Shaddai, the Egyptian concept of the God of heaven, which they learned about during their captivity in the land of the Nile. Long after the times of Melchizedek all three of these concepts of God became joined together to form the doctrine of the creator Deity, the Lord God of Israel.
\vs p142 3:6 \pc \ublistelem{4.}\bibnobreakspace \bibemph{Elohim.} From the times of Adam the teaching of the Paradise Trinity has persisted. Do you not recall how the Scriptures begin by asserting that “In the beginning the Gods created the heavens and the earth”? This indicates that when that record was made the Trinity concept of three Gods in one had found lodgment in the religion of our forebears.
\vs p142 3:7 \pc \ublistelem{5.}\bibnobreakspace \bibemph{The Supreme Yahweh.} By the times of Isaiah these beliefs about God had expanded into the concept of a Universal Creator who was simultaneously all\hyp{}powerful and all\hyp{}merciful. And this evolving and enlarging concept of God virtually supplanted all previous ideas of Deity in our fathers’ religion.
\vs p142 3:8 \pc \ublistelem{6.}\bibnobreakspace \bibemph{The Father in heaven.} And now do we know God as our Father in heaven. Our teaching provides a religion wherein the believer \bibemph{is} a son of God. That is the good news of the gospel of the kingdom of heaven. Coexistent with the Father are the Son and the Spirit, and the revelation of the nature and ministry of these Paradise Deities will continue to enlarge and brighten throughout the endless ages of the eternal spiritual progression of the ascending sons of God. At all times and during all ages the true worship of any human being --- as concerns individual spiritual progress --- is recognized by the indwelling spirit as homage rendered to the Father in heaven.
\vs p142 3:9 \pc Never before had the apostles been so shocked as they were upon hearing this recounting of the growth of the concept of God in the Jewish minds of previous generations; they were too bewildered to ask questions. As they sat before Jesus in silence, the Master continued: \textcolor{ubdarkred}{“And you would have known these truths had you read the Scriptures. Have you not read in Samuel where it says: ‘And the anger of the Lord was kindled against Israel, so much so that he moved David against them, saying, go number Israel and Judah’? And this was not strange because in the days of Samuel the children of Abraham really believed that Yahweh created both good and evil. But when a later writer narrated these events, subsequent to the enlargement of the Jewish concept of the nature of God, he did not dare attribute evil to Yahweh; therefore he said: ‘And Satan stood up against Israel and provoked David to number Israel.’ Cannot you discern that such records in the Scriptures clearly show how the concept of the nature of God continued to grow from one generation to another?}
\vs p142 3:10 \textcolor{ubdarkred}{“Again should you have discerned the growth of the understanding of divine law in perfect keeping with these enlarging concepts of divinity. When the children of Israel came out of Egypt in the days before the enlarged revelation of Yahweh, they had ten commandments which served as their law right up to the times when they were encamped before Sinai. And these ten commandments were:}
\vs p142 3:11 \textcolor{ubdarkred}{“\ublistelem{1.}\bibnobreakspace You shall worship no other god, for the Lord is a jealous God.}
\vs p142 3:12 \textcolor{ubdarkred}{“\ublistelem{2.}\bibnobreakspace You shall not make molten gods.}
\vs p142 3:13 \textcolor{ubdarkred}{“\ublistelem{3.}\bibnobreakspace You shall not neglect to keep the feast of unleavened bread.}
\vs p142 3:14 \textcolor{ubdarkred}{“\ublistelem{4.}\bibnobreakspace Of all the males of men or cattle, the first\hyp{}born are mine, says the Lord.}
\vs p142 3:15 \textcolor{ubdarkred}{“\ublistelem{5.}\bibnobreakspace Six days you may work, but on the seventh day you shall rest.}
\vs p142 3:16 \textcolor{ubdarkred}{“\ublistelem{6.}\bibnobreakspace You shall not fail to observe the feast of the first fruits and the feast of the ingathering at the end of the year.}
\vs p142 3:17 \textcolor{ubdarkred}{“\ublistelem{7.}\bibnobreakspace You shall not offer the blood of any sacrifice with leavened bread.}
\vs p142 3:18 \textcolor{ubdarkred}{“\ublistelem{8.}\bibnobreakspace The sacrifice of the feast of the Passover shall not be left until morning.}
\vs p142 3:19 \textcolor{ubdarkred}{“\ublistelem{9.}\bibnobreakspace The first of the first fruits of the ground you shall bring to the house of the Lord your God.}
\vs p142 3:20 \textcolor{ubdarkred}{“\ublistelem{10.}\bibnobreakspace You shall not seethe a kid in its mother’s milk.}
\vs p142 3:21 \pc \textcolor{ubdarkred}{“And then, amidst the thunders and lightnings of Sinai, Moses gave them the new ten commandments, which you will all allow are more worthy utterances to accompany the enlarging Yahweh concepts of Deity. And did you never take notice of these commandments as twice recorded in the Scriptures, that in the first case deliverance from Egypt is assigned as the reason for Sabbath keeping, while in a later record the advancing religious beliefs of our forefathers demanded that this be changed to the recognition of the fact of creation as the reason for Sabbath observance?}
\vs p142 3:22 \textcolor{ubdarkred}{“And then will you remember that once again --- in the greater spiritual enlightenment of Isaiah’s day --- these ten negative commandments were changed into the great and positive law of love, the injunction to love God supremely and your neighbor as yourself. And it is this supreme law of love for God and for man that I also declare to you as constituting the whole duty of man.”}
\vs p142 3:23 \pc And when he had finished speaking, no man asked him a question. They went, each one to his sleep.
\usection{4.\bibnobreakspace Flavius and Greek Culture}
\vs p142 4:1 Flavius, the Greek Jew, was a proselyte of the gate, having been neither circumcised nor baptized; and since he was a great lover of the beautiful in art and sculpture, the house which he occupied when sojourning in Jerusalem was a beautiful edifice. This home was exquisitely adorned with priceless treasures which he had gathered up here and there on his world travels. When he first thought of inviting Jesus to his home, he feared that the Master might take offense at the sight of these so\hyp{}called images. But Flavius was agreeably surprised when Jesus entered the home that, instead of rebuking him for having these supposedly idolatrous objects scattered about the house, he manifested great interest in the entire collection and asked many appreciative questions about each object as Flavius escorted him from room to room, showing him all of his favorite statues.
\vs p142 4:2 The Master saw that his host was bewildered at his friendly attitude toward art; therefore, when they had finished the survey of the entire collection, Jesus said: \textcolor{ubdarkred}{“Because you appreciate the beauty of things created by my Father and fashioned by the artistic hands of man, why should you expect to be rebuked? Because Moses onetime sought to combat idolatry and the worship of false gods, why should all men frown upon the reproduction of grace and beauty? I say to you, Flavius, Moses’ children have misunderstood him, and now do they make false gods of even his prohibitions of images and the likeness of things in heaven and on earth. But even if Moses taught such restrictions to the darkened minds of those days, what has that to do with this day when the Father in heaven is revealed as the universal Spirit Ruler over all? And, Flavius, I declare that in the coming kingdom they shall no longer teach, ‘Do not worship this and do not worship that’; no longer shall they concern themselves with commands to refrain from this and take care not to do that, but rather shall all be concerned with one supreme duty. And this duty of man is expressed in two great privileges: sincere worship of the infinite Creator, the Paradise Father, and loving service bestowed upon one’s fellow men. If you love your neighbor as you love yourself, you really know that you are a son of God.}
\vs p142 4:3 \textcolor{ubdarkred}{“In an age when my Father was not well understood, Moses was justified in his attempts to withstand idolatry, but in the coming age the Father will have been revealed in the life of the Son; and this new revelation of God will make it forever unnecessary to confuse the Creator Father with idols of stone or images of gold and silver. Henceforth, intelligent men may enjoy the treasures of art without confusing such material appreciation of beauty with the worship and service of the Father in Paradise, the God of all things and all beings.”}
\vs p142 4:4 \pc Flavius believed all that Jesus taught him. The next day he went to Bethany beyond the Jordan and was baptized by the disciples of John. And this he did because the apostles of Jesus did not yet baptize believers. When Flavius returned to Jerusalem, he made a great feast for Jesus and invited sixty of his friends. And many of these guests also became believers in the message of the coming kingdom.
\usection{5.\bibnobreakspace The Discourse on Assurance}
\vs p142 5:1 One of the great sermons which Jesus preached in the temple this Passover week was in answer to a question asked by one of his hearers, a man from Damascus. This man asked Jesus: “But, Rabbi, how shall we know of a certainty that you are sent by God, and that we may truly enter into this kingdom which you and your disciples declare is near at hand?” And Jesus answered:
\vs p142 5:2 \pc \textcolor{ubdarkred}{“As to my message and the teaching of my disciples, you should judge them by their fruits. If we proclaim to you the truths of the spirit, the spirit will witness in your hearts that our message is genuine. Concerning the kingdom and your assurance of acceptance by the heavenly Father, let me ask what father among you who is a worthy and kindhearted father would keep his son in anxiety or suspense regarding his status in the family or his place of security in the affections of his father’s heart? Do you earth fathers take pleasure in torturing your children with uncertainty about their place of abiding love in your human hearts? Neither does your Father in heaven leave his faith children of the spirit in doubtful uncertainty as to their position in the kingdom. If you receive God as your Father, then indeed and in truth are you the sons of God. And if you are sons, then are you secure in the position and standing of all that concerns eternal and divine sonship. If you believe my words, you thereby believe in Him who sent me, and by thus believing in the Father, you have made your status in heavenly citizenship sure. If you do the will of the Father in heaven, you shall never fail in the attainment of the eternal life of progress in the divine kingdom.}
\vs p142 5:3 \textcolor{ubdarkred}{“The Supreme Spirit shall bear witness with your spirits that you are truly the children of God. And if you are the sons of God, then have you been born of the spirit of God; and whosoever has been born of the spirit has in himself the power to overcome all doubt, and this is the victory that overcomes all uncertainty, even your faith.}
\vs p142 5:4 \textcolor{ubdarkred}{“Said the Prophet Isaiah, speaking of these times: ‘When the spirit is poured upon us from on high, then shall the work of righteousness become peace, quietness, and assurance forever.’ And for all who truly believe this gospel, I will become surety for their reception into the eternal mercies and the everlasting life of my Father’s kingdom. You, then, who hear this message and believe this gospel of the kingdom are the sons of God, and you have life everlasting; and the evidence to all the world that you have been born of the spirit is that you sincerely love one another.”}
\vs p142 5:5 \pc The throng of listeners remained many hours with Jesus, asking him questions and listening attentively to his comforting answers. Even the apostles were emboldened by Jesus’ teaching to preach the gospel of the kingdom with more power and assurance. This experience at Jerusalem was a great inspiration to the twelve. It was their first contact with such enormous crowds, and they learned many valuable lessons which proved of great assistance in their later work.
\usection{6.\bibnobreakspace The Visit with Nicodemus}
\vs p142 6:1 One evening at the home of Flavius there came to see Jesus one Nicodemus, a wealthy and elderly member of the Jewish Sanhedrin. He had heard much about the teachings of this Galilean, and so he went one afternoon to hear him as he taught in the temple courts. He would have gone often to hear Jesus teach, but he feared to be seen by the people in attendance upon his teaching, for already were the rulers of the Jews so at variance with Jesus that no member of the Sanhedrin would want to be identified in any open manner with him. Accordingly, Nicodemus had arranged with Andrew to see Jesus privately and after nightfall on this particular evening. Peter, James, and John were in Flavius’s garden when the interview began, but later they all went into the house where the discourse continued.
\vs p142 6:2 In receiving Nicodemus, Jesus showed no particular deference; in talking with him, there was no compromise or undue persuasiveness. The Master made no attempt to repulse his secretive caller, nor did he employ sarcasm. In all his dealings with the distinguished visitor, Jesus was calm, earnest, and dignified. Nicodemus was not an official delegate of the Sanhedrin; he came to see Jesus wholly because of his personal and sincere interest in the Master’s teachings.
\vs p142 6:3 Upon being presented by Flavius, Nicodemus said: “Rabbi, we know that you are a teacher sent by God, for no mere man could so teach unless God were with him. And I am desirous of knowing more about your teachings regarding the coming kingdom.”
\vs p142 6:4 Jesus answered Nicodemus: \textcolor{ubdarkred}{“Verily, verily, I say to you, Nicodemus, except a man be born from above, he cannot see the kingdom of God.”} Then replied Nicodemus: “But how can a man be born again when he is old? He cannot enter a second time into his mother’s womb to be born.”
\vs p142 6:5 Jesus said: \textcolor{ubdarkred}{“Nevertheless, I declare to you, except a man be born of the spirit, he cannot enter into the kingdom of God. That which is born of the flesh is flesh, and that which is born of the spirit is spirit. But you should not marvel that I said you must be born from above. When the wind blows, you hear the rustle of the leaves, but you do not see the wind --- whence it comes or whither it goes --- and so it is with everyone born of the spirit. With the eyes of the flesh you can behold the manifestations of the spirit, but you cannot actually discern the spirit.”}
\vs p142 6:6 Nicodemus replied: “But I do not understand --- how can that be?” Said Jesus: \textcolor{ubdarkred}{“Can it be that you are a teacher in Israel and yet ignorant of all this? It becomes, then, the duty of those who know about the realities of the spirit to reveal these things to those who discern only the manifestations of the material world. But will you believe us if we tell you of the heavenly truths? Do you have the courage, Nicodemus, to believe in one who has descended from heaven, even the Son of Man?”}
\vs p142 6:7 And Nicodemus said: “But how can I begin to lay hold upon this spirit which is to remake me in preparation for entering into the kingdom?” Jesus answered: \textcolor{ubdarkred}{“Already does the spirit of the Father in heaven indwell you. If you would be led by this spirit from above, very soon would you begin to see with the eyes of the spirit, and then by the wholehearted choice of spirit guidance would you be born of the spirit since your only purpose in living would be to do the will of your Father who is in heaven. And so finding yourself born of the spirit and happily in the kingdom of God, you would begin to bear in your daily life the abundant fruits of the spirit.”}
\vs p142 6:8 Nicodemus was thoroughly sincere. He was deeply impressed but went away bewildered. Nicodemus was accomplished in self\hyp{}development, in self\hyp{}restraint, and even in high moral qualities. He was refined, egoistic, and altruistic; but he did not know how to \bibemph{submit} his will to the will of the divine Father as a little child is willing to submit to the guidance and leading of a wise and loving earthly father, thereby becoming in reality a son of God, a progressive heir of the eternal kingdom.
\vs p142 6:9 But Nicodemus did summon faith enough to lay hold of the kingdom. He faintly protested when his colleagues of the Sanhedrin sought to condemn Jesus without a hearing; and with Joseph of Arimathea, he later boldly acknowledged his faith and claimed the body of Jesus, even when most of the disciples had fled in fear from the scenes of their Master’s final suffering and death.
\usection{7.\bibnobreakspace The Lesson on the Family}
\vs p142 7:1 After the busy period of teaching and personal work of Passover week in Jerusalem, Jesus spent the next Wednesday at Bethany with his apostles, resting. That afternoon, Thomas asked a question which elicited a long and instructive answer. Said Thomas: “Master, on the day we were set apart as ambassadors of the kingdom, you told us many things, instructed us regarding our personal mode of life, but what shall we teach the multitude? How are these people to live after the kingdom more fully comes? Shall your disciples own slaves? Shall your believers court poverty and shun property? Shall mercy alone prevail so that we shall have no more law and justice?” Jesus and the twelve spent all afternoon and all that evening, after supper, discussing Thomas’s questions. For the purposes of this record we present the following summary of the Master’s instruction:
\vs p142 7:2 Jesus sought first to make plain to his apostles that he himself was on earth living a unique life in the flesh, and that they, the twelve, had been called to participate in this bestowal experience of the Son of Man; and as such coworkers, they, too, must share in many of the special restrictions and obligations of the entire bestowal experience. There was a veiled intimation that the Son of Man was the only person who had ever lived on earth who could simultaneously see into the very heart of God and into the very depths of man’s soul.
\vs p142 7:3 Very plainly Jesus explained that the kingdom of heaven was an evolutionary experience, beginning here on earth and progressing up through successive life stations to Paradise. In the course of the evening he definitely stated that at some future stage of kingdom development he would revisit this world in spiritual power and divine glory.
\vs p142 7:4 He next explained that the “kingdom idea” was not the best way to illustrate man’s relation to God; that he employed such figures of speech because the Jewish people were expecting the kingdom, and because John had preached in terms of the coming kingdom. Jesus said: \textcolor{ubdarkred}{“The people of another age will better understand the gospel of the kingdom when it is presented in terms expressive of the family relationship --- when man understands religion as the teaching of the fatherhood of God and the brotherhood of man, sonship with God.”} Then the Master discoursed at some length on the earthly family as an illustration of the heavenly family, restating the two fundamental laws of living: the first commandment of love for the father, the head of the family, and the second commandment of mutual love among the children, to love your brother as yourself. And then he explained that such a quality of brotherly affection would invariably manifest itself in unselfish and loving social service.
\vs p142 7:5 Following that, came the memorable discussion of the fundamental characteristics of family life and their application to the relationship existing between God and man. Jesus stated that a true family is founded on the following seven facts:
\vs p142 7:6 \ublistelem{1.}\bibnobreakspace \bibemph{The fact of existence.} The relationships of nature and the phenomena of mortal likenesses are bound up in the family: Children inherit certain parental traits. The children take origin in the parents; personality existence depends on the act of the parent. The relationship of father and child is inherent in all nature and pervades all living existences.
\vs p142 7:7 \pc \ublistelem{2.}\bibnobreakspace \bibemph{Security and pleasure.} True fathers take great pleasure in providing for the needs of their children. Many fathers are not content with supplying the mere wants of their children but enjoy making provision for their pleasures also.
\vs p142 7:8 \pc \ublistelem{3.}\bibnobreakspace \bibemph{Education and training.} Wise fathers carefully plan for the education and adequate training of their sons and daughters. When young they are prepared for the greater responsibilities of later life.
\vs p142 7:9 \pc \ublistelem{4.}\bibnobreakspace \bibemph{Discipline and restraint.} Farseeing fathers also make provision for the necessary discipline, guidance, correction, and sometimes restraint of their young and immature offspring.
\vs p142 7:10 \pc \ublistelem{5.}\bibnobreakspace \bibemph{Companionship and loyalty.} The affectionate father holds intimate and loving intercourse with his children. Always is his ear open to their petitions; he is ever ready to share their hardships and assist them over their difficulties. The father is supremely interested in the progressive welfare of his progeny.
\vs p142 7:11 \pc \ublistelem{6.}\bibnobreakspace \bibemph{Love and mercy.} A compassionate father is freely forgiving; fathers do not hold vengeful memories against their children. Fathers are not like judges, enemies, or creditors. Real families are built upon tolerance, patience, and forgiveness.
\vs p142 7:12 \pc \ublistelem{7.}\bibnobreakspace \bibemph{Provision for the future.} Temporal fathers like to leave an inheritance for their sons. The family continues from one generation to another. Death only ends one generation to mark the beginning of another. Death terminates an individual life but not necessarily the family.
\vs p142 7:13 \pc For hours the Master discussed the application of these features of family life to the relations of man, the earth child, to God, the Paradise Father. And this was his conclusion: \textcolor{ubdarkred}{“This entire relationship of a son to the Father, I know in perfection, for all that you must attain of sonship in the eternal future I have now already attained. The Son of Man is prepared to ascend to the right hand of the Father, so that in me is the way now open still wider for all of you to see God and, ere you have finished the glorious progression, to become perfect, even as your Father in heaven is perfect.”}
\vs p142 7:14 When the apostles heard these startling words, they recalled the pronouncements which John made at the time of Jesus’ baptism, and they also vividly recalled this experience in connection with their preaching and teaching subsequent to the Master’s death and resurrection.
\vs p142 7:15 Jesus is a divine Son, one in the Universal Father’s full confidence. He had been with the Father and comprehended him fully. He had now lived his earth life to the full satisfaction of the Father, and this incarnation in the flesh had enabled him fully to comprehend man. Jesus was the perfection of man; he had attained just such perfection as all true believers are destined to attain in him and through him. Jesus revealed a God of perfection to man and presented in himself the perfected son of the realms to God.
\vs p142 7:16 Although Jesus discoursed for several hours, Thomas was not yet satisfied, for he said: “But, Master, we do not find that the Father in heaven always deals kindly and mercifully with us. Many times we grievously suffer on earth, and not always are our prayers answered. Where do we fail to grasp the meaning of your teaching?”
\vs p142 7:17 Jesus replied: \textcolor{ubdarkred}{“Thomas, Thomas, how long before you will acquire the ability to listen with the ear of the spirit? How long will it be before you discern that this kingdom is a spiritual kingdom, and that my Father is also a spiritual being? Do you not understand that I am teaching you as spiritual children in the spirit family of heaven, of which the fatherhead is an infinite and eternal spirit? Will you not allow me to use the earth family as an illustration of divine relationships without so literally applying my teaching to material affairs? In your minds cannot you separate the spiritual realities of the kingdom from the material, social, economic, and political problems of the age? When I speak the language of the spirit, why do you insist on translating my meaning into the language of the flesh just because I presume to employ commonplace and literal relationships for purposes of illustration? My children, I implore that you cease to apply the teaching of the kingdom of the spirit to the sordid affairs of slavery, poverty, houses, and lands, and to the material problems of human equity and justice. These temporal matters are the concern of the men of this world, and while in a way they affect all men, you have been called to represent me in the world, even as I represent my Father. You are spiritual ambassadors of a spiritual kingdom, special representatives of the spirit Father. By this time it should be possible for me to instruct you as full\hyp{}grown men of the spirit kingdom. Must I ever address you only as children? Will you never grow up in spirit perception? Nevertheless, I love you and will bear with you, even to the very end of our association in the flesh. And even then shall my spirit go before you into all the world.”}
\usection{8.\bibnobreakspace In Southern Judea}
\vs p142 8:1 By the end of April the opposition to Jesus among the Pharisees and Sadducees had become so pronounced that the Master and his apostles decided to leave Jerusalem for a while, going south to work in Bethlehem and Hebron. The entire month of May was spent in doing personal work in these cities and among the people of the surrounding villages. No public preaching was done on this trip, only house\hyp{}to\hyp{}house visitation. A part of this time, while the apostles taught the gospel and ministered to the sick, Jesus and Abner spent at Engedi, visiting the Nazarite colony. John the Baptist had gone forth from this place, and Abner had been head of this group. Many of the Nazarite brotherhood became believers in Jesus, but the majority of these ascetic and eccentric men refused to accept him as a teacher sent from heaven because he did not teach fasting and other forms of self\hyp{}denial.
\vs p142 8:2 The people living in this region did not know that Jesus had been born in Bethlehem. They always supposed the Master had been born at Nazareth, as did the vast majority of his disciples, but the twelve knew the facts.
\vs p142 8:3 This sojourn in the south of Judea was a restful and fruitful season of labor; many souls were added to the kingdom. By the first days of June the agitation against Jesus had so quieted down in Jerusalem that the Master and the apostles returned to instruct and comfort believers.
\vs p142 8:4 Although Jesus and the apostles spent the entire month of June in or near Jerusalem, they did no public teaching during this period. They lived for the most part in tents, which they pitched in a shaded park, or garden, known in that day as Gethsemane. This park was situated on the western slope of the Mount of Olives not far from the brook Kidron. The Sabbath weekends they usually spent with Lazarus and his sisters at Bethany. Jesus entered within the walls of Jerusalem only a few times, but a large number of interested inquirers came out to Gethsemane to visit with him. One Friday evening Nicodemus and one Joseph of Arimathea ventured out to see Jesus but turned back through fear even after they were standing before the entrance to the Master’s tent. And, of course, they did not perceive that Jesus knew all about their doings.\fnc{The Sabbath \bibtextul{week ends} they usually spent with Lazarus\ldots{} \bibexpl{The two-word form is supported by Webster’s; the hyphenated form (week-end) by the OED, but the closed form is not found in any of the contemporary sources. However, the closed form has become the standard usage since that time, as has the related “weekday,” therefore the committee decided to adopt the closed form for both words.}}
\vs p142 8:5 When the rulers of the Jews learned that Jesus had returned to Jerusalem, they prepared to arrest him; but when they observed that he did no public preaching, they concluded that he had become frightened by their previous agitation and decided to allow him to carry on his teaching in this private manner without further molestation. And thus affairs moved along quietly until the last days of June, when one Simon, a member of the Sanhedrin, publicly espoused the teachings of Jesus, after so declaring himself before the rulers of the Jews. Immediately a new agitation for Jesus’ apprehension sprang up and grew so strong that the Master decided to retire into the cities of Samaria and the Decapolis.
