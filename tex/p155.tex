\upaper{155}{Fleeing Through Northern Galilee}
\vs p155 0:1 SOON after landing near Kheresa on this eventful Sunday, Jesus and the twenty\hyp{}four went a little way to the north, where they spent the night in a beautiful park south of Bethsaida\hyp{}Julias. They were familiar with this camping place, having stopped there in days gone by. Before retiring for the night, the Master called his followers around him and discussed with them the plans for their projected tour through Batanea and northern Galilee to the Phoenician coast.
\usection{1.\bibnobreakspace Why Do the Heathen Rage?}
\vs p155 1:1 Said Jesus: \textcolor{ubdarkred}{“You should all recall how the Psalmist spoke of these times, saying, ‘Why do the heathen rage and the peoples plot in vain? The kings of the earth set themselves, and the rulers of the people take counsel together, against the Lord and against his anointed, saying, Let us break the bonds of mercy asunder and let us cast away the cords of love.’}
\vs p155 1:2 \textcolor{ubdarkred}{“Today you see this fulfilled before your eyes. But you shall not see the remainder of the Psalmist’s prophecy fulfilled, for he entertained erroneous ideas about the Son of Man and his mission on earth. My kingdom is founded on love, proclaimed in mercy, and established by unselfish service. My Father does not sit in heaven laughing in derision at the heathen. He is not wrathful in his great displeasure. True is the promise that the Son shall have these so\hyp{}called heathen (in reality his ignorant and untaught brethren) for an inheritance. And I will receive these gentiles with open arms of mercy and affection. All this loving\hyp{}kindness shall be shown the so\hyp{}called heathen, notwithstanding the unfortunate declaration of the record which intimates that the triumphant Son ‘shall break them with a rod of iron and dash them to pieces like a potter’s vessel.’ The Psalmist exhorted you to ‘serve the Lord with fear’ --- I bid you enter into the exalted privileges of divine sonship by faith; he commands you to rejoice with trembling; I bid you rejoice with assurance. He says, ‘Kiss the Son, lest he be angry, and you perish when his wrath is kindled.’ But you who have lived with me well know that anger and wrath are not a part of the establishment of the kingdom of heaven in the hearts of men. But the Psalmist did glimpse the true light when, in finishing this exhortation, he said: ‘Blessed are they who put their trust in this Son.’”}
\vs p155 1:3 Jesus continued to teach the twenty\hyp{}four, saying: \textcolor{ubdarkred}{“The heathen are not without excuse when they rage at us. Because their outlook is small and narrow, they are able to concentrate their energies enthusiastically. Their goal is near and more or less visible; wherefore do they strive with valiant and effective execution. You who have professed entrance into the kingdom of heaven are altogether too vacillating and indefinite in your teaching conduct. The heathen strike directly for their objectives; you are guilty of too much chronic yearning. If you desire to enter the kingdom, why do you not take it by spiritual assault even as the heathen take a city they lay siege to? You are hardly worthy of the kingdom when your service consists so largely in an attitude of regretting the past, whining over the present, and vainly hoping for the future. Why do the heathen rage? Because they know not the truth. Why do you languish in futile yearning? Because you \bibemph{obey} not the truth. Cease your useless yearning and go forth bravely doing that which concerns the establishment of the kingdom.}
\vs p155 1:4 \textcolor{ubdarkred}{“In all that you do, become not one\hyp{}sided and overspecialized. The Pharisees who seek our destruction verily think they are doing God’s service. They have become so narrowed by tradition that they are blinded by prejudice and hardened by fear. Consider the Greeks, who have a science without religion, while the Jews have a religion without science. And when men become thus misled into accepting a narrow and confused disintegration of truth, their only hope of salvation is to become truth\hyp{}co\hyp{}ordinated --- converted.}
\vs p155 1:5 \textcolor{ubdarkred}{“Let me emphatically state this eternal truth: If you, by truth co\hyp{}ordination, learn to exemplify in your lives this beautiful wholeness of righteousness, your fellow men will then seek after you that they may gain what you have so acquired. The measure wherewith truth seekers are drawn to you represents the measure of your truth endowment, your righteousness. The extent to which you have to go with your message to the people is, in a way, the measure of your failure to live the whole or righteous life, the truth\hyp{}co\hyp{}ordinated life.”}
\vs p155 1:6 And many other things the Master taught his apostles and the evangelists before they bade him good night and sought rest upon their pillows.
\usection{2.\bibnobreakspace The Evangelists in Chorazin}
\vs p155 2:1 On Monday morning, May 23, Jesus directed Peter to go over to Chorazin with the twelve evangelists while he, with the eleven, departed for Caesarea\hyp{}Philippi, going by way of the Jordan to the Damascus\hyp{}Capernaum road, thence northeast to the junction with the road to Caesarea\hyp{}Philippi, and then on into that city, where they tarried and taught for two weeks. They arrived during the afternoon of Tuesday, May 24.
\vs p155 2:2 Peter and the evangelists sojourned in Chorazin for two weeks, preaching the gospel of the kingdom to a small but earnest company of believers. But they were not able to win many new converts. No city of all Galilee yielded so few souls for the kingdom as Chorazin. In accordance with Peter’s instructions the twelve evangelists had less to say about healing --- things physical --- while they preached and taught with increased vigor the spiritual truths of the heavenly kingdom. These two weeks at Chorazin constituted a veritable baptism of adversity for the twelve evangelists in that it was the most difficult and unproductive period in their careers up to this time. Being thus deprived of the satisfaction of winning souls for the kingdom, each of them the more earnestly and honestly took stock of his own soul and its progress in the spiritual paths of the new life.
\vs p155 2:3 When it appeared that no more people were minded to seek entrance into the kingdom, Peter, on Tuesday, June 7, called his associates together and departed for Caesarea\hyp{}Philippi to join Jesus and the apostles. They arrived about noontime on Wednesday and spent the entire evening in rehearsing their experiences among the unbelievers of Chorazin. During the discussions of this evening Jesus made further reference to the parable of the sower and taught them much about the meaning of the apparent failure of life undertakings.
\usection{3.\bibnobreakspace At Caesarea\hyp{}Philippi}
\vs p155 3:1 Although Jesus did no public work during this two weeks’ sojourn near Caesarea\hyp{}Philippi, the apostles held numerous quiet evening meetings in the city, and many of the believers came out to the camp to talk with the Master. Very few were added to the group of believers as a result of this visit. Jesus talked with the apostles each day, and they more clearly discerned that a new phase of the work of preaching the kingdom of heaven was now beginning. They were commencing to comprehend that the “kingdom of heaven is not meat and drink but the realization of the spiritual joy of the acceptance of divine sonship.”
\vs p155 3:2 The sojourn at Caesarea\hyp{}Philippi was a real test to the eleven apostles; it was a difficult two weeks for them to live through. They were well\hyp{}nigh depressed, and they missed the periodic stimulation of Peter’s enthusiastic personality. In these times it was truly a great and testing adventure to believe in Jesus and go forth to follow after him. Though they made few converts during these two weeks, they did learn much that was highly profitable from their daily conferences with the Master.
\vs p155 3:3 The apostles learned that the Jews were spiritually stagnant and dying because they had crystallized truth into a creed; that when truth becomes formulated as a boundary line of self\hyp{}righteous exclusiveness instead of serving as signposts of spiritual guidance and progress, such teachings lose their creative and life\hyp{}giving power and ultimately become merely preservative and fossilizing.
\vs p155 3:4 Increasingly they learned from Jesus to look upon human personalities in terms of their possibilities in time and in eternity. They learned that many souls can best be led to love the unseen God by being first taught to love their brethren whom they can see. And it was in this connection that new meaning became attached to the Master’s pronouncement concerning unselfish service for one’s fellows: \textcolor{ubdarkred}{“Inasmuch as you did it to one of the least of my brethren, you did it to me.”}
\vs p155 3:5 One of the great lessons of this sojourn at Caesarea had to do with the origin of religious traditions, with the grave danger of allowing a sense of sacredness to become attached to nonsacred things, common ideas, or everyday events. From one conference they emerged with the teaching that true religion was man’s heartfelt loyalty to his highest and truest convictions.
\vs p155 3:6 Jesus warned his believers that, if their religious longings were only material, increasing knowledge of nature would, by progressive displacement of the supposed supernatural origin of things, ultimately deprive them of their faith in God. But that, if their religion were spiritual, never could the progress of physical science disturb their faith in eternal realities and divine values.
\vs p155 3:7 They learned that, when religion is wholly spiritual in motive, it makes all life more worth while, filling it with high purposes, dignifying it with transcendent values, inspiring it with superb motives, all the while comforting the human soul with a sublime and sustaining hope. True religion is designed to lessen the strain of existence; it releases faith and courage for daily living and unselfish serving. Faith promotes spiritual vitality and righteous fruitfulness.
\vs p155 3:8 Jesus repeatedly taught his apostles that no civilization could long survive the loss of the best in its religion. And he never grew weary of pointing out to the twelve the great danger of accepting religious symbols and ceremonies in the place of religious experience. His whole earth life was consistently devoted to the mission of thawing out the frozen forms of religion into the liquid liberties of enlightened sonship.
\usection{4.\bibnobreakspace On the Way to Phoenicia}
\vs p155 4:1 On Thursday morning, June 9, after receiving word regarding the progress of the kingdom brought by the messengers of David from Bethsaida, this group of twenty\hyp{}five teachers of truth left Caesarea\hyp{}Philippi to begin their journey to the Phoenician coast. They passed around the marsh country, by way of Luz, to the point of junction with the Magdala\hyp{}Mount Lebanon trail road, thence to the crossing with the road leading to Sidon, arriving there Friday afternoon.
\vs p155 4:2 While pausing for lunch under the shadow of an overhanging ledge of rock, near Luz, Jesus delivered one of the most remarkable addresses which his apostles ever listened to throughout all their years of association with him. No sooner had they seated themselves to break bread than Simon Peter asked Jesus: “Master, since the Father in heaven knows all things, and since his spirit is our support in the establishment of the kingdom of heaven on earth, why is it that we flee from the threats of our enemies? Why do we refuse to confront the foes of truth?” But before Jesus had begun to answer Peter’s question, Thomas broke in, asking: “Master, I should really like to know just what is wrong with the religion of our enemies at Jerusalem. What is the real difference between their religion and ours? Why is it we are at such diversity of belief when we all profess to serve the same God?” And when Thomas had finished, Jesus said: \textcolor{ubdarkred}{“While I would not ignore Peter’s question, knowing full well how easy it would be to misunderstand my reasons for avoiding an open clash with the rulers of the Jews at just this time, still it will prove more helpful to all of you if I choose rather to answer Thomas’s question. And that I will proceed to do when you have finished your lunch.”}
\usection{5.\bibnobreakspace The Discourse on True Religion}
\vs p155 5:1 This memorable discourse on religion, summarized and restated in modern phraseology, gave expression to the following truths:
\vs p155 5:2 \pc While the religions of the world have a double origin --- natural and revelatory --- at any one time and among any one people there are to be found three distinct forms of religious devotion. And these three manifestations of the religious urge are:
\vs p155 5:3 \ublistelem{1.}\bibnobreakspace \bibemph{Primitive religion.} The seminatural and instinctive urge to fear mysterious energies and worship superior forces, chiefly a religion of the physical nature, the religion of fear.
\vs p155 5:4 \pc \ublistelem{2.}\bibnobreakspace \bibemph{The religion of civilization.} The advancing religious concepts and practices of the civilizing races --- the religion of the mind --- the intellectual theology of the authority of established religious tradition.
\vs p155 5:5 \pc \ublistelem{3.}\bibnobreakspace \bibemph{True religion --- the religion of revelation.} The revelation of supernatural values, a partial insight into eternal realities, a glimpse of the goodness and beauty of the infinite character of the Father in heaven --- the religion of the spirit as demonstrated in human experience.
\vs p155 5:6 \pc The religion of the physical senses and the superstitious fears of natural man, the Master refused to belittle, though he deplored the fact that so much of this primitive form of worship should persist in the religious forms of the more intelligent races of mankind. Jesus made it clear that the great difference between the religion of the mind and the religion of the spirit is that, while the former is upheld by ecclesiastical authority, the latter is wholly based on human experience.
\vs p155 5:7 \pc And then the Master, in his hour of teaching, went on to make clear these truths:
\vs p155 5:8 \pc Until the races become highly intelligent and more fully civilized, there will persist many of those childlike and superstitious ceremonies which are so characteristic of the evolutionary religious practices of primitive and backward peoples. Until the human race progresses to the level of a higher and more general recognition of the realities of spiritual experience, large numbers of men and women will continue to show a personal preference for those religions of authority which require only intellectual assent, in contrast to the religion of the spirit, which entails active participation of mind and soul in the faith adventure of grappling with the rigorous realities of progressive human experience.
\vs p155 5:9 The acceptance of the traditional religions of authority presents the easy way out for man’s urge to seek satisfaction for the longings of his spiritual nature. The settled, crystallized, and established religions of authority afford a ready refuge to which the distracted and distraught soul of man may flee when harassed by fear and tormented by uncertainty. Such a religion requires of its devotees, as the price to be paid for its satisfactions and assurances, only a passive and purely intellectual assent.
\vs p155 5:10 And for a long time there will live on earth those timid, fearful, and hesitant individuals who will prefer thus to secure their religious consolations, even though, in so casting their lot with the religions of authority, they compromise the sovereignty of personality, debase the dignity of self\hyp{}respect, and utterly surrender the right to participate in that most thrilling and inspiring of all possible human experiences: the personal quest for truth, the exhilaration of facing the perils of intellectual discovery, the determination to explore the realities of personal religious experience, the supreme satisfaction of experiencing the personal triumph of the actual realization of the victory of spiritual faith over intellectual doubt as it is honestly won in the supreme adventure of all human existence --- man seeking God, for himself and as himself, and finding him.
\vs p155 5:11 The religion of the spirit means effort, struggle, conflict, faith, determination, love, loyalty, and progress. The religion of the mind --- the theology of authority --- requires little or none of these exertions from its formal believers. Tradition is a safe refuge and an easy path for those fearful and halfhearted souls who instinctively shun the spirit struggles and mental uncertainties associated with those faith voyages of daring adventure out upon the high seas of unexplored truth in search for the farther shores of spiritual realities as they may be discovered by the progressive human mind and experienced by the evolving human soul.
\vs p155 5:12 \pc And Jesus went on to say: \textcolor{ubdarkred}{“At Jerusalem the religious leaders have formulated the various doctrines of their traditional teachers and the prophets of other days into an established system of intellectual beliefs, a religion of authority. The appeal of all such religions is largely to the mind. And now are we about to enter upon a deadly conflict with such a religion since we will so shortly begin the bold proclamation of a new religion --- a religion which is not a religion in the present\hyp{}day meaning of that word, a religion that makes its chief appeal to the divine spirit of my Father which resides in the mind of man; a religion which shall derive its authority from the fruits of its acceptance that will so certainly appear in the personal experience of all who really and truly become believers in the truths of this higher spiritual communion.”}
\vs p155 5:13 Pointing out each of the twenty\hyp{}four and calling them by name, Jesus said: \textcolor{ubdarkred}{“And now, which one of you would prefer to take this easy path of conformity to an established and fossilized religion, as defended by the Pharisees at Jerusalem, rather than to suffer the difficulties and persecutions attendant upon the mission of proclaiming a better way of salvation to men while you realize the satisfaction of discovering for yourselves the beauties of the realities of a living and personal experience in the eternal truths and supreme grandeurs of the kingdom of heaven? Are you fearful, soft, and ease\hyp{}seeking? Are you afraid to trust your future in the hands of the God of truth, whose sons you are? Are you distrustful of the Father, whose children you are? Will you go back to the easy path of the certainty and intellectual settledness of the religion of traditional authority, or will you gird yourselves to go forward with me into that uncertain and troublous future of proclaiming the new truths of the religion of the spirit, the kingdom of heaven in the hearts of men?”}
\vs p155 5:14 All twenty\hyp{}four of his hearers rose to their feet, intending to signify their united and loyal response to this, one of the few emotional appeals which Jesus ever made to them, but he raised his hand and stopped them, saying: \textcolor{ubdarkred}{“Go now apart by yourselves, each man alone with the Father, and there find the unemotional answer to my question, and having found such a true and sincere attitude of soul, speak that answer freely and boldly to my Father and your Father, whose infinite life of love is the very spirit of the religion we proclaim.”}
\vs p155 5:15 The evangelists and apostles went apart by themselves for a short time. Their spirits were uplifted, their minds were inspired, and their emotions mightily stirred by what Jesus had said. But when Andrew called them together, the Master said only: \textcolor{ubdarkred}{“Let us resume our journey. We go into Phoenicia to tarry for a season, and all of you should pray the Father to transform your emotions of mind and body into the higher loyalties of mind and the more satisfying experiences of the spirit.”}
\vs p155 5:16 As they journeyed on down the road, the twenty\hyp{}four were silent, but presently they began to talk one with another, and by three o’clock that afternoon they could not go farther; they came to a halt, and Peter, going up to Jesus, said: “Master, you have spoken to us the words of life and truth. We would hear more; we beseech you to speak to us further concerning these matters.”
\usection{6.\bibnobreakspace The Second Discourse on Religion}
\vs p155 6:1 And so, while they paused in the shade of the hillside, Jesus continued to teach them regarding the religion of the spirit, in substance saying:
\vs p155 6:2 \pc \textcolor{ubdarkred}{You have come out from among those of your fellows who choose to remain satisfied with a religion of mind, who crave security and prefer conformity. You have elected to exchange your feelings of authoritative certainty for the assurances of the spirit of adventurous and progressive faith. You have dared to protest against the grueling bondage of institutional religion and to reject the authority of the traditions of record which are now regarded as the word of God. Our Father did indeed speak through Moses, Elijah, Isaiah, Amos, and Hosea, but he did not cease to minister words of truth to the world when these prophets of old made an end of their utterances. My Father is no respecter of races or generations in that the word of truth is vouchsafed one age and withheld from another. Commit not the folly of calling that divine which is wholly human, and fail not to discern the words of truth which come not through the traditional oracles of supposed inspiration.}
\vs p155 6:3 \pc \textcolor{ubdarkred}{I have called upon you to be born again, to be born of the spirit. I have called you out of the darkness of authority and the lethargy of tradition into the transcendent light of the realization of the possibility of making for yourselves the greatest discovery possible for the human soul to make --- the supernal experience of finding God for yourself, in yourself, and of yourself, and of doing all this as a fact in your own personal experience. And so may you pass from death to life, from the authority of tradition to the experience of knowing God; thus will you pass from darkness to light, from a racial faith inherited to a personal faith achieved by actual experience; and thereby will you progress from a theology of mind handed down by your ancestors to a true religion of spirit which shall be built up in your souls as an eternal endowment.}
\vs p155 6:4 \textcolor{ubdarkred}{Your religion shall change from the mere intellectual belief in traditional authority to the actual experience of that living faith which is able to grasp the reality of God and all that relates to the divine spirit of the Father. The religion of the mind ties you hopelessly to the past; the religion of the spirit consists in progressive revelation and ever beckons you on toward higher and holier achievements in spiritual ideals and eternal realities.}
\vs p155 6:5 \textcolor{ubdarkred}{While the religion of authority may impart a present feeling of settled security, you pay for such a transient satisfaction the price of the loss of your spiritual freedom and religious liberty. My Father does not require of you as the price of entering the kingdom of heaven that you should force yourself to subscribe to a belief in things which are spiritually repugnant, unholy, and untruthful. It is not required of you that your own sense of mercy, justice, and truth should be outraged by submission to an outworn system of religious forms and ceremonies. The religion of the spirit leaves you forever free to follow the truth wherever the leadings of the spirit may take you. And who can judge --- perhaps this spirit may have something to impart to this generation which other generations have refused to hear?}
\vs p155 6:6 \textcolor{ubdarkred}{Shame on those false religious teachers who would drag hungry souls back into the dim and distant past and there leave them! And so are these unfortunate persons doomed to become frightened by every new discovery, while they are discomfited by every new revelation of truth. The prophet who said, “He will be kept in perfect peace whose mind is stayed on God,” was not a mere intellectual believer in authoritative theology. This truth\hyp{}knowing human had discovered God; he was not merely talking about God.}
\vs p155 6:7 \textcolor{ubdarkred}{I admonish you to give up the practice of always quoting the prophets of old and praising the heroes of Israel, and instead aspire to become living prophets of the Most High and spiritual heroes of the coming kingdom. To honor the God\hyp{}knowing leaders of the past may indeed be worth while, but why, in so doing, should you sacrifice the supreme experience of human existence: finding God for yourselves and knowing him in your own souls?}
\vs p155 6:8 \textcolor{ubdarkred}{Every race of mankind has its own mental outlook upon human existence; therefore must the religion of the mind ever run true to these various racial viewpoints. Never can the religions of authority come to unification. Human unity and mortal brotherhood can be achieved only by and through the superendowment of the religion of the spirit. Racial minds may differ, but all mankind is indwelt by the same divine and eternal spirit. The hope of human brotherhood can only be realized when, and as, the divergent mind religions of authority become impregnated with, and overshadowed by, the unifying and ennobling religion of the spirit --- the religion of personal spiritual experience.}
\vs p155 6:9 \textcolor{ubdarkred}{The religions of authority can only divide men and set them in conscientious array against each other; the religion of the spirit will progressively draw men together and cause them to become understandingly sympathetic with one another. The religions of authority require of men uniformity in belief, but this is impossible of realization in the present state of the world. The religion of the spirit requires only unity of experience --- uniformity of destiny --- making full allowance for diversity of belief. The religion of the spirit requires only uniformity of insight, not uniformity of viewpoint and outlook. The religion of the spirit does not demand uniformity of intellectual views, only unity of spirit feeling. The religions of authority crystallize into lifeless creeds; the religion of the spirit grows into the increasing joy and liberty of ennobling deeds of loving service and merciful ministration.}
\vs p155 6:10 \textcolor{ubdarkred}{But watch, lest any of you look with disdain upon the children of Abraham because they have fallen on these evil days of traditional barrenness. Our forefathers gave themselves up to the persistent and passionate search for God, and they found him as no other whole race of men have ever known him since the times of Adam, who knew much of this as he was himself a Son of God. My Father has not failed to mark the long and untiring struggle of Israel, ever since the days of Moses, to find God and to know God. For weary generations the Jews have not ceased to toil, sweat, groan, travail, and endure the sufferings and experience the sorrows of a misunderstood and despised people, all in order that they might come a little nearer the discovery of the truth about God. And, notwithstanding all the failures and falterings of Israel, our fathers progressively, from Moses to the times of Amos and Hosea, did reveal increasingly to the whole world an ever clearer and more truthful picture of the eternal God. And so was the way prepared for the still greater revelation of the Father which you have been called to share.}
\vs p155 6:11 \textcolor{ubdarkred}{Never forget there is only one adventure which is more satisfying and thrilling than the attempt to discover the will of the living God, and that is the supreme experience of honestly trying to do that divine will. And fail not to remember that the will of God can be done in any earthly occupation. Some callings are not holy and others secular. All things are sacred in the lives of those who are spirit led; that is, subordinated to truth, ennobled by love, dominated by mercy, and restrained by fairness --- justice. The spirit which my Father and I shall send into the world is not only the Spirit of Truth but also the spirit of idealistic beauty.}
\vs p155 6:12 \textcolor{ubdarkred}{You must cease to seek for the word of God only on the pages of the olden records of theologic authority. Those who are born of the spirit of God shall henceforth discern the word of God regardless of whence it appears to take origin. Divine truth must not be discounted because the channel of its bestowal is apparently human. Many of your brethren have minds which accept the theory of God while they spiritually fail to realize the presence of God. And that is just the reason why I have so often taught you that the kingdom of heaven can best be realized by acquiring the spiritual attitude of a sincere child. It is not the mental immaturity of the child that I commend to you but rather the \bibemph{spiritual simplicity} of such an easy\hyp{}believing and fully\hyp{}trusting little one. It is not so important that you should know about the fact of God as that you should increasingly grow in the ability to \bibemph{feel the presence of God.}}
\vs p155 6:13 \textcolor{ubdarkred}{When you once begin to find God in your soul, presently you will begin to discover him in other men’s souls and eventually in all the creatures and creations of a mighty universe. But what chance does the Father have to appear as a God of supreme loyalties and divine ideals in the souls of men who give little or no time to the thoughtful contemplation of such eternal realities? While the mind is not the seat of the spiritual nature, it is indeed the gateway thereto.}
\vs p155 6:14 \textcolor{ubdarkred}{But do not make the mistake of trying to prove to other men that you have found God; you cannot consciously produce such valid proof, albeit there are two positive and powerful demonstrations of the fact that you are God\hyp{}knowing, and they are:}
\vs p155 6:15 \textcolor{ubdarkred}{1. The fruits of the spirit of God showing forth in your daily routine life.}
\vs p155 6:16 \pc \textcolor{ubdarkred}{2. The fact that your entire life plan furnishes positive proof that you have unreservedly risked everything you are and have on the adventure of survival after death in the pursuit of the hope of finding the God of eternity, whose presence you have foretasted in time.}
\vs p155 6:17 \pc \textcolor{ubdarkred}{Now, mistake not, my Father will ever respond to the faintest flicker of faith. He takes note of the physical and superstitious emotions of the primitive man. And with those honest but fearful souls whose faith is so weak that it amounts to little more than an intellectual conformity to a passive attitude of assent to religions of authority, the Father is ever alert to honor and foster even all such feeble attempts to reach out for him. But you who have been called out of darkness into the light are expected to believe with a whole heart; your faith shall dominate the combined attitudes of body, mind, and spirit.}
\vs p155 6:18 \textcolor{ubdarkred}{You are my apostles, and to you religion shall not become a theologic shelter to which you may flee in fear of facing the rugged realities of spiritual progress and idealistic adventure; but rather shall your religion become the fact of real experience which testifies that God has found you, idealized, ennobled, and spiritualized you, and that you have enlisted in the eternal adventure of finding the God who has thus found and sonshipped you.}
\vs p155 6:19 \pc And when Jesus had finished speaking, he beckoned to Andrew and, pointing to the west toward Phoenicia, said: \textcolor{ubdarkred}{“Let us be on our way.”}
